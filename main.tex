\documentclass[14pt,oneside]{extarticle} % Utilizar extarticle para tamaños de letra grandes
\usepackage{tocloft}
\renewcommand{\cftdot}{}
\usepackage{sectsty}
\usepackage{amsmath}
\usepackage{pdfpages}
\usepackage{geometry}
\usepackage[utf8]{inputenc}
\usepackage[spanish,es-nodecimaldot,es-tabla]{babel}
\usepackage[ruled,vlined,linesnumbered,lined,boxed,commentsnumbered,onelanguage]{algorithm2e}
\usepackage{bm}
\usepackage{upgreek}
\usepackage{amsmath}
\usepackage{amssymb}
\usepackage{microtype} 
\usepackage[svgnames]{xcolor}
%\usepackage{titlesec}
\usepackage{enumitem}
\usepackage{caption}
\usepackage{subcaption}
\usepackage{fancyhdr}
\usepackage[colorlinks=true,linkcolor=black,citecolor=black,urlcolor=black,bookmarks=false]{hyperref}
\usepackage{graphicx}
\usepackage{mwe}
\usepackage{cite}
\usepackage{titlesec}
\usepackage{graphicx}
\usepackage{longtable}
\usepackage{tabularx}
\usepackage{adjustbox}
\usepackage{changepage} % Para ajustar el tamaño de la parte específica del texto

% Tipo de letra más formal
\usepackage{mathptmx} % Times New Roman font

% Tamaño de fuente aumentado
\renewcommand{\normalsize}{\fontsize{13.5}{17}\selectfont} % Interlineado más estrecho

% Definir \subsubsubsection
\titleclass{\subsubsubsection}{straight}[\subsubsection]
\newcounter{subsubsubsection}[subsubsection]
\renewcommand\thesubsubsubsection{\thesubsubsection.\arabic{subsubsubsection}}
\titleformat{\subsubsubsection}
  {\normalfont\normalsize\bfseries}{\thesubsubsubsection}{1em}{}
\titlespacing*{\subsubsubsection}
  {0pt}{1.25ex plus 1ex minus .2ex}{0.75ex plus .2ex}

% Aumentar el nivel de profundidad de numeración de secciones
\setcounter{secnumdepth}{4}
% Aumentar el nivel de profundidad de secciones en el índice
\setcounter{tocdepth}{4}

\usepackage{multirow}
\usepackage{xcolor,colortbl}
\usepackage{setspace}
\usepackage{caption}
\usepackage{lipsum}
\usepackage{makecell}
\usepackage{float}
\usepackage{listings}
\usepackage{forloop}
\usepackage{longtable}
\newcounter{loopcntr}
\newcommand{\rpt}[2][1]{%
  \forloop{loopcntr}{0}{\value{loopcntr}<#1}{#2}%
}
\newcommand{\on}[1][1]{
  \forloop{loopcntr}{0}{\value{loopcntr}<#1}{&\cellcolor{blue}}
}
\newcommand{\off}[1][1]{
  \forloop{loopcntr}{0}{\value{loopcntr}<#1}{&}
}
\geometry{
 letterpaper,
 left=20mm,
 right=20mm,
 tmargin=30mm,
 bmargin=30mm,
 }
\definecolor{Custom_color_1}{rgb}{0.56,0.83,0.29}
\definecolor{Custom_color_2}{rgb}{0.2,0.2,0.8}
\definecolor{mygreen}{RGB}{28,172,0} % color values Red, loGreen, Blue
\definecolor{mylilas}{RGB}{170,55,241}
\captionsetup{justification=raggedright,labelfont={color=Custom_color_2,bf}}
\setlength{\headheight}{18pt}
\setlength{\parskip}{5mm}

\lstset{language=Matlab,%
    basicstyle=\small,
    breaklines=true,
    morekeywords={matlab2tikz},
    keywordstyle=\color{blue},
    morekeywords=[2]{1}, keywordstyle=[2]{\color{black}},
    identifierstyle=\color{black},
    stringstyle=\color{mylilas},
    commentstyle=\color{mygreen},
    showstringspaces=false,
    numbers=left,
    numberstyle={\tiny \color{black}},
    numbersep=9pt, 
    emph=[1]{for,end,break},emphstyle=[1]\color{red},  
    xleftmargin=.2\textwidth, 
    xrightmargin=.2\textwidth,
    morecomment=[f][\color{Custom_color_1}][0]{\%},
}

\renewcommand{\rmdefault}{ptm} % Times font
\renewcommand{\sfdefault}{phv} 
\renewcommand{\algorithmcfname}{Algoritmo}
\pagestyle{fancy}

\begin{document}

\fancyhf{}
\renewcommand{\headrulewidth}{0pt}

\begin{adjustwidth}{0cm}{0cm}
\begin{center}
    \includegraphics[width=0.12\textwidth]{Figures/Logo_ipn.png} \hfill
    \includegraphics[width=0.18\textwidth]{Figures/Logoupiita.png}

    \vspace{0.5cm} % Espacio vertical

    {\LARGE % Tamaño de fuente grande para el encabezado principal
    {Instituto Politécnico Nacional}  \\
    \vspace{0.2cm}
    Unidad Profesional Interdisciplinaria en Ingeniería y Tecnologías Avanzadas \\
    \vspace{1.5cm}
    Trabajo Terminal 2
    }

    \vspace{0.5cm} % Espacio vertical

    \rule{\textwidth}{1pt}

    {\LARGE % Tamaño de fuente grande para el título del trabajo
    \textit{\textbf{Sistema semiautomático de mezclado de biopolímeros y polímeros sintéticos y su extrusión para la obtención de filamentos}}
    }

    \rule{\textwidth}{1pt}

    \vspace{0.5cm} % Espacio vertical

    {\LARGE % Tamaño de fuente grande para el encabezado principal
    Realizado por:
    }

    \vspace{0.5cm} % Espacio vertical

    {\large % Tamaño de fuente para el contenido de la tabla
    \begin{center}
        \begin{tabular}{| l | l | l |}
            \hline
            \rowcolor{Custom_color_1}
            \textbf{Boleta} & \textbf{Nombre} & \textbf{Apellidos} \\
            \hline
            2020640005 & Fernando Manuel & Barrera Ramírez \\ 
            \hline
            2020640313 & Marco Antonio & Melchor Solis \\ 
            \hline
            2020640439 & Enrique Maximiliano & Rivera Zúñiga \\ 
            \hline
        \end{tabular}
    \end{center}
    }
\end{center}
\end{adjustwidth}

\newpage

\fancyhead{}
\fancyhead[C]{ \Large{Trabajo Terminal 2}}
\fancyhead[L]{ \Large{UPIITA}}
\fancyhead[R]{ \Large{SSDMyE}}
\fancyfoot[R]{\thepage}
\renewcommand{\headrulewidth}{2pt}
\renewcommand{\footrulewidth}{2pt}
\pagenumbering{arabic}
\normalsize

\newpage
\tableofcontents
\newpage
\listoftables
\listoffigures
\clearpage

\section{Resumen}

El proyecto dirigido por la Dra. Luz Arcelia García Serrano en el CIIEMAD aborda la implementación de una máquina semi-automática para la extrusión de filamentos para impresión 3D, estos filamentos serán a partir de la mezcla de biopolímeros y polímeros sintéticos. Esta mezcla busca reducir el uso de los plásticos sintéticos así como aprovechar la materia orgánica como materia prima, resultando en un producto final más ecológico.

El sistema comprende varias etapas clave. Comienza con la fase de mezclado, donde los diferentes polímeros se combinan en proporciones adecuadas de aproximadamente 80-85\% sintético y 15-20\% orgánico, según la materia prima que se tenga estos valores podrían cambiar. A continuación, el precalentamiento y mezclado asegura que la mezcla alcance una temperatura óptima y se homogeneice la cual según la experiencia de la doctora líder del proyecto es de 50ºC. La etapa de extrusión, donde la mezcla se comprime a un diámetro de 1.5-1.8mm. Finalmente, el enfriamiento del filamento principalmente para la seguridad del usuario y para que conserve su forma.

El resultado esperado de este proyecto es un filamento de 1.5 a 1.8 mm de diámetro, medidas estandarizadas para impresoras 3D comerciales. Asimismo, se almacenarán los datos de mezclas, temperaturas y tiempos para las correspondientes investigaciones  del CIIEEMAD. 

\section{Palabras Clave}

Biopolímeros, extrusora, programable, filamentos, tecnología de mezcla, extrusión de materiales, control.

\newpage
\section{Abstract}

The project led by Dr. Luz Arcelia García Serrano at CIIEMAD addresses the implementation of a semi-automatic machine for the extrusion of filaments for 3D printing, these filaments will be from the mixture of biopolymers and synthetic polymers. This blend seeks to reduce the use of synthetic plastics as well as to take advantage of organic material as raw material, resulting in a more environmentally friendly final product.

The system comprises several key stages. It starts with the blending phase, where the different polymers are combined in appropriate proportions of approximately 80-85\% synthetic and 15-20\% organic, depending on the raw material available these values could change. Next, preheating and mixing ensures that the mixture reaches an optimum temperature and homogenizes, which according to the experience of the project leader is 50ºC. The extrusion stage, where the mixture is compressed to a diameter of 1.5-1.8mm. Finally, the cooling of the filament, mainly for the user's safety and to keep its shape.

The expected result of this project is a filament of 1.5 to 1.8 mm in diameter, standardized measures for commercial 3D printers. Also, the data of mixtures, temperatures and times will be stored for the corresponding investigations of CIIEMAD.

\newpage

\section{Introducción}

La impresión 3D ha revolucionado la fabricación en diversas industrias, ofreciendo flexibilidad, personalización y reducción de costos. Sin embargo, la dependencia de polímeros sintéticos derivados del petróleo en la impresión 3D ha generado preocupaciones ambientales significativas. Los procesos de obtención y el uso de estos polímeros tradicionales tienen implicaciones adversas para el medio ambiente, contribuyendo a la acumulación de desechos plásticos y la contaminación. Los plásticos tradicionales utilizados en la impresión 3D, a menudo se derivan de recursos no renovables y pueden ser difíciles de reciclar. Sin embargo, su popularidad se debe a sus propiedades únicas, tales como su ligereza, flexibilidad, resistencia mecánica y resistencia a altas temperaturas, además de su facilidad de producción. 

Los plásticos comunes derivados de fuentes fósiles suelen ser químicamente estables y biodegradables, aunque su descomposición es extremadamente lenta. En consecuencia, los residuos plásticos han despertado una mayor atención como un punto crítico en cuanto a su impacto potencial en los ecosistemas terrestres, la atmósfera, el agua dulce y el medio marino. Según \cite{dang2022}, en Europa se vierten cada año al océano entre 307 y 925 millones de litros de artículos, de los cuales alrededor del 82\% son productos de plástico. Se trata principalmente de fragmentos y artículos de un solo uso como botellas, envases y bolsas. Además, se prevé que para 2025 se acumularán 11 mil millones de toneladas de plásticos en el medio ambiente. Con una cantidad tan enorme de “contaminación blanca”, el macro, meso, micro y nanoplástico degradado posteriormente podría representar graves amenazas para la salud humana a lo largo de la cadena alimentaria.

A partir de estas estadísticas, se hace evidente que los desechos plásticos producidos por las impresoras 3D están emergiendo como una preocupación ambiental cada vez más relevante. Según el especialista en asuntos medioambientales, Christian Lölkes, como menciona en \cite{nationalgeographic} entre el 5\% y el 10\% de todos los productos impresos se convierten en residuos que finalmente terminan en los vertederos.

Recientemente, la investigación se ha centrado en los bioplásticos como una posible dirección hacia productos sostenibles y un impacto ambiental reducido. Los plásticos sintéticos elaborados a partir de recursos renovables son bioplásticos, como los agropolímeros (cultivos terrestres), los polímeros de algas y los polímeros bacterianos. Como se puede observar en \cite{rajpoot2022}, actualmente, los bioplásticos representan alrededor del 1\% de los más de 368 millones de toneladas de plásticos que se producen anualmente. 

Estos resultados sugieren que la aplicación de bioplásticos está todavía en su infancia, como lo demuestran la estadísticas mostradas en \cite{dang2022}, una capacidad de producción global de 2,11 millones de toneladas, de las cuales sólo el 55,5\% se deriva de materiales biodegradables.

En este contexto, los biopolímeros se han convertido en una alternativa prometedora. Estos materiales derivados de fuentes naturales, ofrecen propiedades sostenibles y biodegradables que pueden mitigar los impactos ambientales negativos asociados con los polímeros sintéticos. No obstante, los biopolímeros por sí solos a menudo no poseen las propiedades mecánicas y térmicas requeridas para reemplazar completamente a los polímeros sintéticos en la impresión 3D.

Debido a la razón mencionada anteriormente, la motivación subyacente de este proyecto radica en la urgente necesidad de desarrollar una solución tecnológica que permita aprovechar las ventajas de los biopolímeros en la impresión 3D, superando las limitaciones inherentes a estos materiales. Dada la necesidad tecnológica del laboratorio del CIIEMAD a cargo de la doctora Luz Arcelia García Serrano, el objetivo general de este proyecto es concebir y desarrollar una extrusora especializada que permita la producción controlada y eficiente de filamentos de biopolímeros, aprovechando la combinación sinérgica de polímeros naturales y sintéticos. 

Para lograr este objetivo, se propone la creación de un sistema mecatrónico integral. Este sistema se encargará de la mezcla homogénea de los polímeros, la extrusión de los filamentos, el control de la temperatura y el enfriamiento adecuado. La producción resultante de filamentos de biopolímeros se considera relevante para promover la adopción de prácticas más responsables en este ámbito. Sin embargo, se reconoce que este proyecto tiene limitaciones inherentes, incluyendo restricciones de recursos y tiempos. Por lo tanto, se centrará en la concepción y desarrollo del sistema mecatrónico, dejando para futuros trabajos la exploración de aplicaciones específicas y pruebas exhaustivas de rendimiento.

\section{Planteamiento del Problema}

Frente al escenario mencionado en la introducción, emerge la necesidad de explorar alternativas sostenibles. La aplicación de biopolímeros en la impresión 3D representa un desafío significativo debido a las limitaciones de materiales como el PLA, que, aunque sostenibles, tienen una resistencia y tolerancia a altas temperaturas inferiores a las de sus contrapartes sintéticas. Esta limitación afecta su uso en sectores como el automotriz, textil, construcción, etc. Para superar estas barreras, el Centro Interdisciplinario de Investigaciones y Estudios sobre Medio Ambiente y Desarrollo (CIIEMAD) ha explorado mezclas de polímeros naturales y sintéticos, buscando desarrollar un material que combine sostenibilidad con funcionalidad.

La justificación para explorar esta combinación de polímeros está respaldada por múltiples estudios académicos. Por ejemplo, se ha examinado el impacto de combinar polímeros sintéticos y naturales en las propiedades reológicas del asfalto [1], así como la aplicación de un modelo mecánico para compuestos poliméricos reforzados con fibras naturales [2]. Además, el CIIEMAD ya ha desarrollado prototipos mediante un proceso manual de mezcla de estos polímeros.

Sin embargo, el principal desafío se encuentra en la naturaleza manual e ineficiente del proceso de mezcla, lo cual se torna aún más problemático debido a la utilización de instrumentos poco ortodoxos para estas operaciones, como una máquina de esquimo. Este método rudimentario no solo complica la exploración de diferentes combinaciones de polímeros y la experimentación con variadas temperaturas, sino que también carece de la precisión necesaria para obtener resultados consistentes y optimizados. La evaluación visual para determinar si la mezcla está lista es altamente subjetiva y propensa a errores, lo cual pone en riesgo la calidad y la reproducibilidad del compuesto resultante.

Adicionalmente, la producción manual de los filamentos y la creación de piezas se realiza en cantidades muy pequeñas, lo que impide la obtención de filamentos extensos y limita severamente la capacidad de escalar el proceso para satisfacer las demandas requeridas para el uso y estudio de estos biopolímeros.

La mezcla actual involucra un polímero natural en forma de polvo, un polímero sintético en pellets y agua, lo cual, dado el método manual empleado, no permite un control preciso sobre las proporciones y las condiciones de mezcla, como la temperatura, que son cruciales para obtener un compuesto híbrido con las propiedades deseadas.

Ante estas problemáticas, la mecatrónica se destaca como una solución viable. La implementación de un sistema mecatrónico permitiría controlar con precisión la temperatura y las proporciones de los materiales involucrados, así como automatizar el proceso de mezcla y la producción de filamentos. 

\newpage
Con un sistema de este tipo, se podría producir un filamento de biopolímero optimizado para la impresión 3D, garantizando consistencia y calidad, mientras se exploran diferentes combinaciones de polímeros y se experimenta con diversas temperaturas de manera controlada y sistemática.

A través de esto, surge la problemática académica del desarrollo de estos sistemas, dado que si bien ya existen trabajos y antecedentes de extrusoras, el hecho de implementar un sistema de mezclado en conjunto con el de extrusión es algo nuevo, así como la inclusión del sistema de enfriamiento, por lo que el reto representa el diseño y desarrollo de cada subsistema y su comunicación entre estos para lograr un sistema que funcione armónicamente con todas las etapas que se plantean.

En resumen, las fases esenciales del proyecto buscan alcanzar las siguientes acciones:

\begin{itemize}
    \item Mezcla del polímero natural y sintético.
    \item Precalentamiento inicial del sistema.
    \item Ajuste de temperatura según las especificaciones del usuario según la mezcla.
    \item Extrusión del compuesto híbrido.
    \item Enfriamiento del filamento resultante.
\end{itemize}


\section{Justificación}

En la actualidad, como se ha comentado, las demandas de la sociedad moderna, orientadas hacia la sostenibilidad ambiental y la reducción de la huella ecológica, han generado una urgencia por encontrar alternativas a los polímeros sintéticos tradicionales, los cuales, aunque poseen propiedades técnicas admirables, están asociados a una cascada de consecuencias adversas para el medio ambiente. 

Es por esto que surge la idea planteada por la doctora Luz Arcelia García Serrano de implementar un porcentaje orgánico a los filamentos para impresión en 3D para reducir el tiempo de degradación. Además, se puede reutilizar el PLA y aprovechar los residuos orgánicos al mismo tiempo.

Este proyecto permitirá un avance para implementar nuevas soluciones en el mundo de la impresión 3D, el cual incrementa cada año y más personas alrededor del mundo están utilizando. Asimismo, permitirá al laboratorio del CIIEMAD experimentar con distintas mezclas creadas por expertos en el área con un sistema que incluya todos los procesos que necesitan a partir de las materias primas ya a su disposición.

Actualmente no existe un sistema que cuente con todas las etapas propuestas, si bien se pueden encontrar extrusoras de plástico desde artesanales hasta industriales, el diferenciador del proyecto es la etapa de mezclado la cual está fuertemente ligada a los requerimientos iniciales del Centro de investigación mencionado anteriormente. Esta brinda mayor rapidez en obtener el filamento como se hace hoy en día, al tener todo junto y aporta una mayor comodidad debido al poco movimiento que tendrá que realizar el operador.

Por otro lado, la sociedad se ve favorecida por la perspectiva de una producción industrial más amigable con el entorno, reduciendo así la huella de carbono y contribuyendo a la lucha contra el cambio climático. La industria local se erige como un actor relevante, al potenciar el cultivo y procesamiento de biopolímeros, promoviendo empleo y consolidando un bastión de investigación y desarrollo en materia de estos.

\section{Objetivos}

\subsection{Objetivo General}

Diseñar y construir un sistema de mezclado y extrusión para la producción de filamentos de biopolímeros de un diámetro semi-uniforme, que combine polímeros naturales y sintéticos, con el propósito de satisfacer las necesidades reales del Centro Interdisciplinario de Investigaciones y Estudios sobre Medio Ambiente y Desarrollo.

\subsection{Objetivos específicos TT1}

\begin{itemize}
    \item Diseñar un sistema de mezclado semi-automatizado controlado por tiempos para que permita la homogeneización de polímeros naturales y sintéticos.
    \item Diseñar un sistema de extrusión de la mezcla.
    \item Diseñar un programa para el sensado de los parámetros y crear una interfaz para su visualización.
    \item Simular el sistema en conjunto en un software CAD.
\end{itemize}

\subsection{Objetivos específicos TT2}

\begin{itemize}
    \item Manufacturar y ensamblar físicamente los módulos de mezclado, precalentado, extrusión y enfriamiento.
    \item Implementar sensores a lo largo del sistema para recibir información y mostrarla en la interfaz.
    \item Realizar pruebas de funcionamiento para verificar la interacción adecuada entre los componentes mecánicos y electrónicos del sistema.
\end{itemize}

\section{Antecedentes}

Las extrusoras a partir de su invención en el siglo XVIII han significado avance tecnológico para el mezclado y moldeo de materias de una forma sencilla pero eficaz. Si bien en la actualidad esta idea no es muy novedosa, siguen siendo utilizadas en industrias como la alimenticia y de formación de plásticos. Es por ello que a continuación, se muestran algunas máquinas ya existentes en la extrusión de materia orgánica y de plásticos.

Como se puede observar en la tabla \ref{Tab:Extrusoras}, existen diversas extrusoras con diferentes funciones, las cuales van desde académicas hasta industriales. El objetivo es realizar un sistema de mezclado, extrusión y enfriamiento especializado para la creación de un filamento con base en biopolímeros y que maneje temperaturas máximas de 250°C.

\begin{table}
\centering
\begin{tabular}{|c|p{3cm}|p{2.5cm}|p{3cm}|p{1.7cm}|p{2.2cm}|c|}
\hline
\textbf{No} & \textbf{Nombre} & \textbf{Descripción} & \textbf{Características} & \textbf{País} & \textbf{Instituto} & \textbf{Tipo} \\
\hline
1 & Extrusora de termoplásticos\cite{TablaAntecedentes_Ref1}\newline \includegraphics[width=3cm,height=2cm]{ImagenesTT1/Extrusora_UPIITA.png} & Extrusora de filamento termoplástico PLA para la impresión 3D & Dimen.: 1.2$m^3$\newline Peso: 50kg\newline Temp. de fusión: 50-350°C\newline Dia. extrusión: 1.5-3 mm\newline Vel. de extrusión: 1-20 rpm & México & UPIITA & Tesis \\
\hline
2 & Máquina extrusora de filamento\cite{TablaAntecedentes_Ref2}\newline \includegraphics[width=3cm,height=2.5cm]{Figures/Extrusora60Mil.png} & Máquina semi automática extrusora de filamento de impresoras 3D & Temp. máxima: 262°C\newline Velocidad de trabajo: 60 rpm\newline Mezcla de trabajo: 250 gr. & México & ESIME & Tesis \\
\hline
3 & Moldeo por extrusión \cite{TablaAntecedentes_Ref3}\newline \includegraphics[width=3cm,height=1.5cm]{Figures/ExtrusoraESIME.jpg} & Máquina para el proceso de moldeo por extrusión a partir de pellets de polietileno & Temperatura máx: 270°C\newline Diámetro del husillo: 32 mm\newline Longitud del husillo: 760 mm\newline Diámetro de salida: 1.75 mm & México & ESIME & Tesis \\
\hline
4 & Máquina FullExtruder\cite{TablaAntecedentes_Ref4}\newline \includegraphics[width=3cm,height=2.5cm]{Figures/TablaAntecedentes_Ref4.png} & Extrusora de filamento, medidor de grosor y bobinadora de filamento & Temp. máxima: 250°C\newline Grosor de filamento: 1.5-1.8mm\newline Vel. máxima del motor: 16 rpm & España & & Producto \\
\hline
\end{tabular}
\caption{Antecedentes de extrusoras similares}
\label{Tab:Extrusoras}
\end{table}

\section{Marco teórico}

\subsection{Extrusión de Polímeros}

La extrusión de polímeros es un proceso fundamental en la fabricación de productos plásticos, ya que permite la transformación de materiales poliméricos en formas específicas mediante la aplicación de calor y presión \cite{white2003}. Este proceso se utiliza en una amplia variedad de industrias, desde la producción de películas plásticas y tuberías hasta la fabricación de perfiles, láminas y filamentos para impresión 3D. Para comprender mejor los principios fundamentales de la extrusión de polímeros, es esencial conocer cómo funciona una extrusora y sus componentes clave.

\subsubsection{Principios Fundamentales de la Extrusión de Polímeros}

La extrusión de polímeros se basa en el uso de materiales plásticos termoplásticos \cite{tadmor2006}. Estos materiales tienen la propiedad de ablandarse y fundirse cuando se calientan y luego solidificarse cuando se enfrían. Los polímeros más comunes utilizados en la extrusión incluyen el polietileno (PE), el polipropileno (PP), el PVC, el poliestireno (PS) y muchos otros \cite{mohanty2002}. 

Por otro lado, la extrusora es la máquina central en el proceso de extrusión. Consiste en un tornillo de extrusión encerrado dentro de un cilindro. El tornillo gira y avanza a través del cilindro, comprimiendo y fundiendo el material polimérico a medida que se desplaza hacia adelante. Esta máquina se compone de distintos elementos y etapas, de manera general en \cite{wagner2009} se mencionan: 

\begin{itemize}
    \item Calentamiento: El cilindro de la extrusora está equipado con zonas de calentamiento a lo largo de su longitud. Estas zonas elevan gradualmente la temperatura del material polimérico a medida que avanza hacia la salida. El control preciso de la temperatura es esencial para obtener un producto de calidad.
    \newpage
    \item Tornillo de Extrusión: El tornillo de extrusión es el componente clave de la extrusora. Está diseñado con un perfil helicoidal que ayuda a comprimir, mezclar y fundir el polímero a medida que avanza. La geometría del tornillo puede variar según la aplicación específica.
    \item Troquel o Dado: En el extremo de la extrusora, se encuentra el troquel o dado. Este componente determina la forma final del producto extruido. Puede haber una amplia variedad de troqueles para producir diferentes perfiles, como tuberías, perfiles, películas, láminas o filamentos.
    \item Enfriamiento y Solidificación: Después de pasar por el troquel, el producto extruido debe enfriarse rápidamente para solidificarse. Esto se logra mediante sistemas de enfriamiento que pueden incluir aire, agua o rodillos enfriados.
\end{itemize}

\subsection{Polímeros naturales (biopolímeros) y polímeros sintéticos}

\subsubsection{Biopolímeros}

Los biopolímeros son polímeros que se derivan de fuentes naturales y renovables, como plantas, animales y microorganismos. Estos polímeros son biodegradables y, en su mayoría, no provienen de recursos fósiles. Ejemplos de biopolímeros incluyen el ácido poliláctico (PLA), que se obtiene del almidón de maíz, y el almidón termoplástico (TPS), derivado del almidón de patata. Los biopolímeros se utilizan en una variedad de aplicaciones, incluyendo envases biodegradables, productos médicos y materiales sostenibles.

Los biopolímeros presentan una serie de propiedades físicas, mecánicas y térmicas que los hacen adecuados para diversas aplicaciones. Sus características incluyen densidad, viscosidad, resistencia y tenacidad, que varían según el tipo de biopolímero y su origen. A continuación, en la tabla \ref{Tab:Propiedades Biopolimeros}, se muestran algunas de las más importantes. 

\begin{table}[H]
\centering
\small
\begin{tabular}{|>{\centering\arraybackslash}m{3.5cm}|>{\centering\arraybackslash}m{2cm}|>{\centering\arraybackslash}m{2cm}|>{\centering\arraybackslash}m{2cm}|>{\centering\arraybackslash}m{2cm}|>{\centering\arraybackslash}m{3cm}|}
\hline
\rowcolor{blue!50}
\textbf{Biopolímero} & \textbf{Temp. de fusión (°C)} & \textbf{Resistencia mecánica} & \textbf{Flexibilidad} & \textbf{Durabilidad} & \textbf{Aplicaciones Comunes} \\
Microalgas & Varía según el tipo & Media & Media & Media & Bioplásticos, alimentos \\
\hline
Alginatos & ~80-120 & Media & Alta & Media & Medicina, alimentos \\
\hline
Carragenanos & ~60-80 & Baja & Alta & Baja & Alimentos, cosméticos \\
\hline
Agar & ~85-95 & Media & Media & Media & Laboratorios, alimentos \\
\hline
Caucho Natural (Hevea brasiliensis) & ~100-110 & Alta & Alta & Alta & Neumáticos, elastómeros \\
\hline
Polihidroxialcanoatos (PHA) & ~160-180 & Alta & Media & Alta & Bioplásticos, envases \\
\hline
Almidón termoplástico & ~100-120 & Media & Baja & Media & Envases, juguetes \\
\hline
Celulosa & ~260-270 & Alta & Baja & Alta & Papel, textiles \\
\hline
Proteínas & Varía según el tipo & Media-Alta & Media & Media-Alta & Alimentos, medicina \\
\hline
Polihidroxibutirato & ~170-180 & Alta & Media & Alta & Bioplásticos, envases \\
\hline
\end{tabular}
\caption{Propiedades de Biopolímero. Recuperado de \cite{ref18}, \cite{ref19}, \cite{ref20}, \cite{ref21}, \cite{ref22}, \cite{ref23}}
\label{Tab:Propiedades Biopolimeros}
\end{table}


\subsubsection{Polímeros sintéticos}

Los polímeros sintéticos son polímeros que se fabrican mediante procesos químicos a partir de materias primas derivadas del petróleo y otras fuentes no renovables. Ejemplos comunes de polímeros sintéticos incluyen el polietileno (PE), el polipropileno (PP) y el policloruro de vinilo (PVC). Estos polímeros son versátiles, duraderos y se utilizan en una amplia gama de aplicaciones, como envases, productos industriales, textiles y más.

Los polímeros presentan una serie de propiedades físicas, mecánicas y térmicas que los hacen adecuados para diversas aplicaciones. Sus características incluyen densidad, viscosidad, resistencia y tenacidad, que varían según el tipo y su origen. A continuación, en la tabla \ref{Tab:Propiedades Polimeros}, se muestran algunas de las más importantes. 

\begin{table}[h]
\centering
\small
\begin{tabular}{|m{3cm}|>{\centering\arraybackslash}m{2.2cm}|>{\centering\arraybackslash}m{2cm}|>{\centering\arraybackslash}m{2cm}|>{\centering\arraybackslash}m{2cm}|>{\centering\arraybackslash}m{3cm}|}
\hline
\rowcolor{blue!50}
\textbf{Nombre del Polímero} & \textbf{Temperatura de Fusión (°C)} & \textbf{Resistencia Mecánica} & \textbf{Flexibilidad} & \textbf{Durabilidad} & \textbf{Aplicaciones Comunes} \\
\hline
PLA (Ácido Poliláctico) & 160-180 & Media & Baja & Media & Prototipos, juguetes, objetos decorativos \\
\hline
ABS (Acrilonitrilo Butadieno Estireno) & 220-250 & Alta & Media & Alta & Piezas mecánicas, herramientas, juguetes \\
\hline
PETG (Tereftalato de polietileno glicol) & 220-250 & Alta & Alta & Alta & Envases, piezas mecánicas, componentes resistentes \\
\hline
TPU (Poliuretano Termoplástico) & 220-250 & Media & Muy Alta & Alta & Piezas flexibles, sellos, conectores \\
\hline
PVA (Alcohol Polivinílico) & 190-210 & Baja & Baja & Baja & Material de soporte soluble \\
\hline
\end{tabular}
\caption{Propiedades de Polímeros. Recuperado de \cite{ref15}, \cite{ref16}, \cite{ref17}}
\label{Tab:Propiedades Polimeros}
\end{table}

\subsubsection{Mezcla de biopolimeros con polimeros sintéticos}

La mezcla de biopolímeros y polímeros sintéticos es una estrategia que combina las ventajas de ambos tipos de materiales. Esta combinación puede tener varios propósitos:

\begin{enumerate}
    \item Mejora de la Sostenibilidad: Al mezclar biopolímeros con polímeros sintéticos, se puede reducir la dependencia de los recursos no renovables y disminuir la huella de carbono de los materiales. Esto contribuye a la sostenibilidad y la reducción de la contaminación.
    \item Mejora de Propiedades: La mezcla de biopolímeros puede mejorar ciertas propiedades de los polímeros sintéticos, como la biodegradabilidad, la resistencia al impacto o la transparencia. Esto puede hacer que los materiales sean más adecuados para aplicaciones específicas.
    \item Control de Costos: La incorporación de biopolímeros puede ayudar a controlar los costos, ya que estos materiales a menudo son más asequibles que los polímeros sintéticos puros.
    \item Cumplimiento de Regulaciones: En algunos casos, las regulaciones ambientales o de salud pueden requerir la incorporación de biopolímeros para reducir el impacto ambiental o mejorar la seguridad de ciertos productos.  
\end{enumerate}

\subsection{Bioplásticos y estado actual de la producción de bioplásticos}

De manera general, los plásticos son materiales poliméricos sintéticos que se caracterizan por su versatilidad y amplias aplicaciones en diversos sectores industriales. La clasificación de los plásticos se basa en su estructura química y propiedades, lo que permite comprender mejor sus características y aplicaciones específicas. 

Por otro lado, los bioplásticos son una clase especial de plásticos que se derivan de fuentes renovables y biodegradables, como plantas, almidón, celulosa y otros materiales orgánicos. A diferencia de los plásticos tradicionales, que se derivan del petróleo y pueden ser perjudiciales para el medio ambiente, los bioplásticos ofrecen una alternativa más sostenible y amigable con el entorno. 

Basandonos de \cite{joogi2020}, para etiquetar un material polimérico como biopolímero o bioplástico, tiene que ser biodegradable, estar elaborado a partir de una fuente renovable (biobase) o ser biocompatible. European Bio plastics describe el material plástico como "bioplástico" si es biodegradable, de base biológica o incluye ambas propiedades. 

Desde el punto de vista medioambiental, la degradabilidad del material y el origen de las materias primas son las propiedades más importantes. Para ofrecer una mejor visión general, todos los polímeros podrían dividirse en cuatro categorías superpuestas, como se muestra en la Figura \ref{fig:Clasificación}.

\begin{figure}[H]
    \centering
    \includegraphics[width=0.7\textwidth]{Figures/Figura1A.png}
    \caption{ Clasificación de plástico. Fuente: \cite{joogi2020}}
    \label{fig:Clasificación}
\end{figure}

Para mostrar una panorama general de la producción de bioplásticos, mencionaremos algunos datos que consideramos importantes, simplemente para conocer y darnos una idea de la facilidad de adquisición de los mismos. Según datos del mercado europeo de bioplásticos mostrados en \cite{joogi2020}, la producción mundial de bioplásticos en  2019 fue de 2,11 millones de toneladas. 

Según el tipo de material, el PP (polipropileno) y los PHA de base biológica han mostrado la tasa  de crecimiento relativo más alta en las cantidades de producción. Como el PP es un material  plástico de uso común con una amplia gama de aplicaciones, se espera que la alternativa de base  biológica tenga un crecimiento continuo también en los próximos años. 

La Tabla \ref{Tab:Ejemplo} ilustra la producción mundial de bioplástico según el tipo de material en 2019 y el  pronóstico para 2024. 

\begin{table}[H]
\begin{center}
{
\setlength\arrayrulewidth{1pt}
\begin{tabular}{|c|c|c|}
\hline
\rowcolor{green!50} \textbf{Tipo de material bioplástico} & \textbf{2019} & \textbf{2024} \\
\hline
\hline
Mezclas de Almidón & 21.3\% & 18.5\%  \\
\hline
PLA & 13.9\% & 13.1\%  \\
\hline
PBAT & 13.4\% & 12.5\%  \\
\hline
PE* & 11.8\% & 12.0\%  \\
\hline
PA* & 11.6\% & 11.6\%  \\
\hline
PET* & 9.8\% & 8.0\%  \\
\hline
PTT* & 9.2\% & 6.6\%  \\
\hline
PBS & 4.3\% & 6.0\%  \\
\hline
Otros (biodegradables) & 1.4\% & 5.3\%  \\
\hline
PHA & 1.2\% & 3.8\%  \\
\hline
Otros (de base biológica/no biodegradables)* & 1.1\% & 1.3\%  \\
\hline
PP* & 0.9\% & 0.9\%  \\
\hline
PEF* & 0.0\% & 0.2\%  \\
\hline
\end{tabular}
}
\end{center}
\caption{Producción mundial de bioplásticos por tipo de material en 2019 y 2024. Los tipos de materiales marcados con * son de base biológica, pero no biodegradables. Fuente: \cite{joogi2020}}
\label{Tab:Ejemplo}
\end{table}

\begin{figure}[H]
    \centering
    \includegraphics[scale=0.6]{Figures/Figura1B.png}
    \caption{ Capacidad de producción de bioplásticos por tipo de material a nivel mundial. Fuente: \cite{joogi2020}}
    \label{fig:Producción}
\end{figure}

Como se ve, las mezclas de almidón fueron el tipo de bioplástico más común en 2019 y continuar ocupando el cargo también para los años siguientes. Como alternativa al PET (tereftalato de polietileno), se espera que el nuevo  polímero PEF (furanoato de polietileno) ingrese al mercado de bioplásticos para  2023. El furanoato de polietileno (PEF) se produce a partir de recursos renovables mediante un proceso de policondensación. Sin embargo, el proceso de producción a escala industrial  todavía tiene algunos desafíos (como la degradación y la decoloración) que superar. 

En 2019, los bioplásticos biodegradables (incluidos PLA, PHA y mezclas de  almidón) representaron el 55,5\% de todos los bioplásticos producidos. Según los  datos de mercado de 2019, por regiones, Europa ocupa el primer lugar en actividades  de investigación y desarrollo relacionadas con los bioplásticos. Sin embargo, Asia  todavía tiene la mayor capacidad de producción: alrededor del 45\% de los bioplásticos  producidos a nivel mundial se fabricaron en Asia, seguida de Europa (25\%), América  del Norte (18\%) y América del Sur (12\%) \cite{dang2022}.

\subsubsection{Métodos de producción de bioplásticos}

Los métodos de producción de bioplásticos se pueden dividir en cinco grupos principales, basándonos en \cite{joogi2020}, se clasifican según el origen de la materia prima y la correspondiente tecnología de producción de polímeros:

\begin{enumerate}
    \item \textbf{Extraído directamente de biomasa} 

Los bioplásticos podrían producirse mediante extracción de biomasa a partir de biopolímeros naturales como polisacáridos (por ejemplo, almidón, celulosa) y proteínas. Por ejemplo, el uso industrial de biomasa de lignocelulosa y almidón se está expandiendo rápidamente debido principalmente a su bajo costo, abundancia y naturaleza renovable. En realidad, la mayoría de los bioplásticos obtenidos de la extracción de biomasa requieren aditivos o mezclarse con otros polímeros para mejorar las propiedades del material. Para superar las malas propiedades de los materiales y las limitaciones de uso, se utilizan recubrimientos de materiales, mezclas, aditivos de nanopartículas y diferentes modificaciones químicas o físicas. 

    \item \textbf{Producidos por organismos naturales o genéticamente modificados:}

Los PHA son poliésteres de base biológica compostables. El poli (3-hidroxibutirato), también conocido como [P(3HB)], es el PHA más común. Las propiedades físicas de los PHA son comparables a las de los polímeros petroquímicos comunes, lo que los convierte en alternativas sostenibles para el creciente mercado mundial de bioplásticos.

La producción de PHA a partir de desechos y subproductos de alimentos ha mostrado un gran potencial, ya que los desechos ofrecerían una fuente de carbono abundante y barata. Cuando se utilizan desechos como materia prima para la biosíntesis de PHA, se debe considerar la pureza de los PHA producidos, ya que las contaminaciones virales, bacterianas, plasmídicas o genéticas pueden transferirse al material final. Si el uso previsto es para el contacto con alimentos o para aplicaciones médicas, es necesario aplicar procedimientos adicionales de lavado y esterilización, lo que también generará un posible aumento en los costos del material final.

    \newpage
    \item \textbf{Sintetizado a partir de monómeros de origen biológico:}

El PLA es un poliéster alifático biodegradable, producido
principalmente por policondensación industrial de ácido láctico y/o polimerización con apertura de anillo de lactida. Convencionalmente, los PLA se producen convirtiendo una fuente de carbohidratos en dextrosa, seguida de una fermentación a ácido láctico que se policondensa aún más.

    \item \textbf{Sintetizado a partir de petroquímicos:} 

Los bioplásticos que se sintetizan a partir de petrorecursos son mucho más caros que los plásticos  petroquímicos convencionales, por esta razón, estos materiales rara vez se utilizan solos para  aplicaciones de embalaje y a menudo se combinan con celulosa o almidón
\end{enumerate}


\subsection{Mezclado y homogeneización}

Al tratarse de la primer etapa de todo el proceso, es importante destacar que el filamento resultante depende en gran medida de esta etapa.

El mezclado a realizar consiste en combinar 85-90\% de PLA, 10-15\% gramos del biopolímero entregado por el CIIEMAD, que posteriormente se mezclan con ayuda de un un motorreductor (capaz de entregar una velocidad máxima de al menos 400 rpm y un torque mayor a 1.1 Kg x cm) durante 1 a 2.5 horas hasta tener una consistencia viscosa y sin grumos. Dicha mezcla se verterá a la entrada de la extrusora que cuenta con un precalentamiento a 50°C.

En la parte de extrusión, se termina de homogeneizar por el cizallamiento ocasionado por cambios de presión, velocidad, forma del tornillo, temperatura aplicada y tipo de boquilla \cite{mezcladora}. Además, funde y homogeneiza los materiales antes de llegar a la matriz.

Es importante mencionar que si se realiza un mezclado incorrecto o se manejan temperaturas inadecuadas, afectará a la calidad del filamento. Por ejemplo, en la imagen \ref{fig:Chocolate} se puede observar fallas por un mal manejo de temperaturas.

\begin{figure}[H]
    \centering
    \includegraphics[width=0.75\textwidth]{Figures/Chocolate.jpg}
    \caption{Fallas por temperatura. Fuente: \cite{chocolate}}
    \label{fig:Chocolate}
\end{figure}

\subsubsection{Estrategias de Control en la Extrusión}

En el contexto de la impresión 3D, la extrusión es un proceso crucial en el que se funde y extruye un filamento de material plástico a través de una boquilla para crear capas sucesivas de un objeto tridimensional. El control preciso de la temperatura del cabezal de extrusión, la velocidad de alimentación del filamento y la velocidad de movimiento de la boquilla son esenciales para obtener impresiones de alta calidad. Para lograrlo, se utilizan diversas estrategias de control, que pueden incluir:

\begin{itemize}
    \item Control de Temperatura: Se utiliza un control PID para mantener la temperatura del cabezal de extrusión constante y dentro de un rango específico. Esto garantiza una extrusión uniforme del material.
    \item Control de Velocidad de Alimentación: Se ajusta la velocidad de avance del filamento para controlar la cantidad de material extruido. Esto es crucial para mantener la precisión dimensional de la impresión.
    \item Control de Velocidad de Movimiento: La velocidad de desplazamiento de la boquilla debe ser controlada para garantizar una deposición precisa de material capa por capa.
\end{itemize}

En este sentido, uno de los elementos esenciales de la impresión 3D es el filamento, el material utilizado para construir objetos tridimensionales capa por capa. En este sección, profundizaremos en las propiedades y requisitos de los filamentos utilizados en la impresión 3D, incluyendo el diámetro, la tolerancia dimensional y las características de impresión.

\newpage
\textbf{Diámetro del Filamento}

El diámetro del filamento es una característica fundamental que afecta directamente la calidad de la impresión 3D. Los filamentos más comunes tienen diámetros de 1.75 mm y 2.85 mm, aunque existen otros tamaños disponibles. La elección del diámetro del filamento depende del tipo de impresora 3D que se utilice, ya que las impresoras están diseñadas para trabajar con un tamaño específico. Es esencial que el diámetro del filamento coincida con la capacidad de la impresora para garantizar una alimentación uniforme y una impresión precisa.

\textbf{Tolerancia Dimensional} 

La tolerancia dimensional se refiere a la variación permitida en el diámetro del filamento en relación con su valor nominal. Por ejemplo, un filamento con una tolerancia dimensional de ±0.02 mm significa que el diámetro real del filamento puede variar en hasta 0.02 mm desde el valor nominal. Es crucial que el filamento tenga una tolerancia dimensional baja para evitar problemas de impresión, como atascos en el extrusor o capas de impresión inconsistentes.

\textbf{Temperatura de impresión}

Cada tipo de filamento tiene una temperatura óptima de impresión. Es importante ajustar la temperatura del extrusor y la cama caliente según las recomendaciones del fabricante para obtener los mejores resultados. En este caso, tomando en cuenta las propiedades de los biopolímeros y polímeros mostradas en las tablas \ref{Tab:Propiedades Biopolimeros} y \ref{Tab:Propiedades Polimeros}, trabajaremos desde los 50 °C hasta los 250 °C, cubriendo así la mayoría de las temperaturas de fusión para ambos.

\subsection{Motores}

La selección de los dos motores que se ocuparán en la máquina es de suma importancia tomando en cuenta, la eficiencia energética para consumir lo menor posible pero sin sacrificar el trabajo.

Para el motor de la etapa de mezclado se establecieron parámetros a cumplir por parte del CIIEMAD con rango grande y sencillo de tener. Se busca un motor DC que gire de 150 a 200 rpm en un periodo de 3 a 5 horas. En el caso de extrusión se ocupa un motor que gire de 16 a 25 rpm con un torque mayor debido a que en esta etapa la mezcla será viscosa, requiriendo una mayor fuerza. 

Desafortunadamente no se puede proponer un motor en específico, ya que se requieren experimentos de viscosidad que se buscan realizar en la Escuela de Ingeniería Química e Industrias Extractivas (ESIQIE) de la mezcla de PLA y nuestro biopolímero para determinar mediante cálculos que par en el motor sería el más adecuado y de este modo definir uno en específico.

Como sabemos, los motores DC funcionan como generador de movimiento mecánico para distintos aparatos de uso cotidiano, tales como juguetes y electrodomésticos, pero también en maquinaria industrial. De manera general, según \cite{motoresDC}, el motor de corriente continua o motor de corriente directa es una máquina con la capacidad de convertir energía eléctrica en movimiento o trabajo mecánico a través de fuerzas electromagnéticas. Este dispositivo emplea un rotor con dos polos magnéticos que interactúan de manera constante con un estator con polo N y un fijo de polo S; el rotor gira sobre su eje y gracias a la repulsión y atracción de sus propios polos con los del estator, se produce un movimiento constante.

Existen dos tipos de motores de corriente continua según su forma de excitación: 
\begin{enumerate}
    \item Motor de excitación independiente: En este tipo de motor eléctrico, el bobinado de CC es excitado por una fuente de CC independiente. Con la ayuda de la fuente separada, el bobinado de la armadura del motor es excitado y produce corriente.
    \item Motor de autoexcitación: La autoexcitación significa que la corriente continua que excita las bobinas inductoras procede de la misma máquina generatriz. 
\end{enumerate}

\textbf{Características de un motor de corriente continua}

\begin{itemize}
    \item Tienen una amplia gama de velocidades.
    \item Alto rendimiento en el margen de velocidades que evita su sobrecarga.
    \item Fácil inversión de giro en grandes motores que ayuda a que actúen de modo reversible.
    \item Funcionan como convertidores electromecánicos rotativos de energía porque transforman la energía eléctrica en energía mecánica.
    \item Son de tamaño reducido y práctico.
\end{itemize}

\subsection{Unidad de Procesamiento Central (CPU)}

La CPU es la pieza fundamental de todo dispositivo, es considerado el cerebro de un sistema. En primer lugar, es el encargado de recibir e interpretar datos y ejecutar las secuencias de instrucciones a realizar por cada programa valiéndose de operaciones aritméticas y matemáticas. El CPU interpreta todos los datos que provienen del dispositivo, tanto de los programas como la información que envía el usuario a través de aplicaciones. Además, controla el buen funcionamiento de cada componente del sistema para que todas las acciones sean realizadas en tiempo y forma. \cite{CPU}

\subsection{Interfaz gráfica de Usuario (GUI)}

De manera general, la Interfaz Gráfica de Usuario, conocida en inglés como Graphical User Interface (GUI) es la forma en que un usuario puede interactuar con un dispositivo informático sin introducir comandos de texto en una consola \cite{GUIFer}. En otras palabras, es un entorno visual amigable que permite al usuario realizar cualquier acción sin necesidad de tener conocimientos de programación.

Para que una GUI tenga éxito y se pueda usar fácilmente debe cumplir una serie de requisitos. Lo primero es que sea sencilla de entender. Los elementos principales deben ser también muy identificables. Par ello es importante facilitar y predecir las acciones más comunes de un usuario. La información debe estar ordenada mediante menús, iconos, imágenes, etc. Así será intuitiva y las operaciones para hacer y deshacer se podrán realizar de forma rápida. La usabilidad debe ser fácil. 

Enfocada a nuestro proyecto, podría proporcionar un medio para controlar y supervisar el funcionamiento de la extrusora. Esto incluiría ajustes de temperatura, velocidad de extrusión, parámetros de mezcla, entre otros. Los operadores o técnicos podrían interactuar con la GUI para conFigurar el sistema. También podría mostrar datos en tiempo real sobre el estado de la máquina y el proceso de extrusión. Esto permitiría a los usuarios supervisar la temperatura, la velocidad de alimentación de materia prima y otros indicadores clave mientras la máquina está en funcionamiento. 

\newpage

\section{Diseño del sistema}

\subsection{Diseño conceptual}

\subsubsection{Definición de la metodología mecatrónica}
Se ocupará la metodología VDI-2206 desarrollada por la Asociación Alemana de Ingenieros \cite{VDI2206} aplicando el modelo "V" para darle un enfoque estructurado y sistemático al proyecto, estableciendo estrategias e identificando las etapas más significativas. 

Se muestra gráficamente el procedimiento de prueba y verificación, facilitando a la detección de puntos débiles que permita posteriormente una mejora si es necesario de calidad en las etapas.

Mencionado lo anterior, en la Figura \ref{fig:modeloV} se puede observar el modelo V como metodología mecatrónica.

\begin{figure}[H]
    \centering
    \includegraphics[width=\textwidth]{Figures/ModeloV.png}
    \caption{Modelo V del proyecto}
    \label{fig:modeloV}
\end{figure}

\subsubsection{Descripción de la propuesta solución}

Se pretende desarrollar un sistema de extrusión de biopolímeros,este sistema se fundamentará en los principios operativos de las extrusoras convencionales, tal como se ilustra en el esquema de la Figura \ref{Esquema1}. Sin embargo, se incorporarán características específicas diseñadas para abordar y resolver la problemática previamente expuesta.

\begin{figure}[H]
    \centering
    \includegraphics[scale=0.65]{Figures/Extrusora esquema.jpg}
    \caption{Esquema de extrusor convencional. Fuente:\cite{esquemaExtrusora}}
    \label{Esquema1}
\end{figure}

Una pieza clave en el sistema planteado es la inclusión de una etapa de mezclado. Esta fase tiene como objetivo principal lograr una homogeneización meticulosa entre el polímero natural y el sintético. Es esencial que esta mezcla alcance una uniformidad completa, evidenciada por una coloración consistente, antes de proceder a la etapa estándar de extrusión. Posteriormente, una vez que el material ha sido extruido, se procederá a una etapa de enfriamiento esto con el fin de que el filamento al salir pueda ser completamente manipulable por el usuario.

En resumen, contamos con las etapas mostradas en la Figura \ref{Etapas}.

\begin{figure}[H]
    \centering
    \includegraphics[scale=0.7]{Figures/Etapas.png}
    \caption{Etapas del sistema}
    \label{Etapas}
\end{figure}

Donde:

\begin{enumerate}
    \item \textbf{Mezclado:} Es la primera etapa del sistema y a su entrada se combina el biopolímero (en polvo) obtenido a partir de procesos químicos (10\%-15\%), pellets de PLA recortados para disminuir su tamaño (85\%-90\%) y mejorar la homogeneización y agua. Se inicia el mezclado con ayuda de un motor que gire a gran velocidad (150-200 rpm) por un tiempo de 3-5 horas, obteniendo una mezcla con partículas de un diámetro aproximado al orden de micrómetros. Cabe mencionar que las mezclas que se trabajaran serán meramente experimentales con el fin de encontrar la adecuada para un filamento que se asemeje en textura, flexibilidad y propiedades mecánicas a los comerciales. Sin embargo, se buscará enfocar la atención en mezclas que utilicen como biopolímero principal al Guayule.
    \item \textbf{Precalentado:} Se vierte la mezcla del biopolímero con el polímero sintético y cuenta con una tolva de acero inoxidable con calentadores instalados externamente alcanzando una temperatura en la mezcla de 50°C con el objetivo de obtener un filamento al final del proceso homogéneo y de un calibre constante. (Analizar como calentar en esta etapa de una manera eficiente)
    \item \textbf{Extrusión:} Se derrite la mezcla con calentadores instalados a lo largo del cañón y se somete a compresión para asegurarse de que esté uniformemente fundido y libre de burbujas de aire. Finalmente, la mezcla llega a la matriz con un diámetro de 1.5 a 1.8 mm.
    \item \textbf{Enfriamiento:} Al finalizar el proceso se enfriará con la implementación de un sistema basado en refrigeración por ventiladores a la salida de la matriz y que pasa a través de un canal con orificios, permitiendo al filamento enfriarse de manera rápida y con ello evitar deformaciones y adhesión entre sus propias capas.
\end{enumerate}

Finalmente, en la Figura \ref{fig:DiagramaSistemaCompleto} se muestra el diagrama completo del sistema que se pretende desarrollar, con las etapas anteriormente descritas.

\begin{figure}[H]
    \centering
    \includegraphics[scale=0.5]{Figures/Esquema completo.jpg}
    \caption{Diagrama completo del sistema}
    \label{fig:DiagramaSistemaCompleto}
\end{figure}

\vspace{-1cm}

\subsubsection{Lista de requerimientos}

Ahora, a continuación se presenta una lista detallada de los requerimientos necesarios para el desarrollo y operación del sistema de extrusión. Estos requerimientos han sido categorizados en funcionales y no funcionales, abarcando aspectos como la alimentación, los parámetros de funcionamiento, los parámetros de salida, las dimensiones aproximadas del equipo y el peso del sistema. La tabla \ref{tabla:requerimientos} proporciona una descripción exhaustiva de cada uno de estos requerimientos, especificando su clasificación y el rango aceptable para su correcto funcionamiento.

\begin{table}[H]
    \centering
    \small
    \begin{tabular}{|c|l|c|l|}
        \hline
        \textbf{No.} & \textbf{Descripción} & \textbf{Clasificación} & \textbf{Rango} \\ \hline
        R1 & \textbf{Alimentación} & Funcional & 110 V -- 127 V \\ \hline
        \multirow{5}{*}{R2} & \textbf{Parámetros de funcionamiento} & \multirow{5}{*}{Funcional} & \\ 
         & \quad \textbullet \ Cantidad total de mezcla & & 0.5 kg -- 1 kg \\ 
        & \quad \textbullet \ Velocidad del motor para mezcladora & & 200 rpm -- 400 rpm \\ 
        & \quad \textbullet \ Temperatura de precalentado & & 50 °C\\
        & \quad \textbullet \ Temperatura de extrusión & & 170 °C - 210 °C\\ 
        & \quad \textbullet \ Velocidad del motor para extrusora & & 5 rpm -- 25 rpm \\ \hline
        \multirow{3}{*}{R3} & \textbf{Parámetros de salida} & \multirow{3}{*}{Funcional} & \\ 
        & \quad \textbullet \ Grosor del filamento & & 1.50 mm -- 1.80 mm \\ 
        & \quad \textbullet \ Cantidad de extrusión & & 0.5 kg/h -- 1 kg/h \\ \hline
        \multirow{3}{*}{R4} & \multirow{3}{*}{\textbf{Dimensiones aproximadas}} & \multirow{3}{*}{No funcional} & Largo: 40 in (1016 mm) \\ 
        & & & Ancho: 20 in (508 mm) \\ 
        & & & Alto: 28 in (711.2 mm) \\ \hline
        R5 & \textbf{Peso del sistema (Sin mezcla)} & No funcional & 20 kg \\ \hline
    \end{tabular}
    \caption{Requerimientos del sistema}
    \label{tabla:requerimientos}
\end{table}

\subsubsection{Estructura de funciones}
El establecimiento de funciones a lo largo de la construcción de la máquina busca organizar y llevar a cabo una buena planificación para elaborar de manera idónea el desarrollo del proyecto. En este sentido, se presentan las funciones propuestas y el diagrama FBS a continuación.

\textbf{Funciones propuestas}
\begin{itemize}
    \item Función principal: Extruir
    \begin{itemize}
        \item F1 Mezclar elementos
        \begin{itemize}
            \item F1.1 Recabar materiales
            \item F1.2 Triturar mezcla
            \item F1.3 Observar homogeneización
        \end{itemize}
        \item F2 Precalentar
        \begin{itemize}
            \item F2.1 Verter mezcla
            \item F2.2 Enviar órdenes por la interfaz
            \item F2.3 Calentar en tolva
        \end{itemize}
        \item F3 Extruir
        \begin{itemize}
            \item F3.1 Revolucionar tornillo
            \item F3.2 Calentar para homogeneización
            \item F3.3 Extruir por matriz
        \end{itemize}
        \item F4 Interfaz H-M
        \begin{itemize}
            \item F4.1 Sensar parámetros
            \item F4.2 Mostrar datos
        \end{itemize}
        \item F5 Enfriar
        \begin{itemize}
            \item F5.1 Accionar ventilador
        \end{itemize}
    \end{itemize}
\end{itemize}

\begin{figure}[H]
    \centering
    \includegraphics[scale=0.65]{Figures/FBS.png}
    \caption{Diagrama de funciones del FBS}
    \label{fig:FBS}
\end{figure}

\subsubsection{IDEF0}

El diagrama IDEF0 muestra la relación de funciones del sistema, esto con el fin de observar de mejor manera las entradas, controles, mecanismos y salidas de cada proceso del proyecto.
\begin{figure}[H]
    \centering
    \includegraphics[scale=0.5]{Figures/A0.png}
    \caption{Diagrama del nodo A-0}
    \label{fig:idef0}
\end{figure}
\begin{figure}[H]
    \centering
    \includegraphics[scale=0.33]{Figures/A0 extendido.png}
    \caption{Nodo A-0 extendido}
    \label{fig:nodoA0_extendido}
\end{figure}

En la representación visual proporcionada por el IDEF0, la función principal se denota como nodo A0 (Figura \ref{fig:idef0}), el cual esboza los procesos críticos de mezclado y extrusión, involucrados en la producción de un rollo de filamento. Este nodo desglosa su operación en seis funciones auxiliares esenciales. (Figura \ref{fig:nodoA0_extendido})

A1 - Gestión de Energía: Este proceso juega un papel crucial en el proyecto, ya que se encarga de distribuir las fuentes de energía necesarias para el funcionamiento óptimo de los demás procesos. Su correcta administración asegura la continuidad y eficiencia de la operación.

A2 - Mezclado: En esta fase, los materiales son combinados de manera homogénea mediante el uso de aspas impulsadas por un motor. Este proceso está meticulosamente controlado en tiempo para asegurar la correcta formación del biopolímero.

A3 - Precalentado: La mezcla obtenida se somete a un proceso de calentamiento, utilizando resistencias térmicas, hasta alcanzar una temperatura de 50°C. Este paso es fundamental para preparar la mezcla para la extrusión.

\newpage
A4 - Extrusión: Aquí se toman en cuenta los parámetros definidos previamente por el usuario para moldear la mezcla en filamento. Este proceso se apoya en motores y sensores de temperatura para garantizar la precisión y calidad del producto.

A5 - Enfriamiento: Posterior a la extrusión, el filamento pasa a través de un tubo de acero inoxidable que se encuentra en un baño de agua fría que permite enfriar y solidificar el material. Este paso es vital para asegurar las propiedades físicas finales del filamento.

En conjunto, estos procesos secuenciales e interconectados culminan en la creación efectiva y eficiente de un filamento uniforme, listo para satisfacer las necesidades del usuario.

\subsubsection{Arquitectura física}

Con base en lo visto en el IDEF0 podemos determinar los módulos que constituirán a nuestro sistema así como sus respectivos submódulos de tal manera que la arquitectura física (Figura \ref{fig:arquitecturafisica}) estará constituida de la siguiente manera:

\begin{itemize}
    \item M1. Módulo energético 
    \begin{itemize}
        \item M1.1 Conversión de CA – CD.
        \item M1.2 Acondicionamiento. 
        \item M1.3 Distribución.
    \end{itemize}
    \item M2. Módulo de mezclado
    \begin{itemize}
        \item M2.1 Rotación del husillo.
    \end{itemize}
    \item M3. Módulo de precalentado
    \begin{itemize}
        \item M3.1 Calentar mezcla.
    \end{itemize}
    \item M4. Módulo de extrusión.
    \begin{itemize}
        \item M4.1 Rotación del husillo.
        \item M4.2 Calentar mezcla.
        \item M4.3 Extrusión del filamento.
    \end{itemize}
    \item M5. Módulo de enfriamiento.
    \begin{itemize}
        \item M5.1 Enfriar filamento. 
    \end{itemize}
    \item M6. Módulo de interfaz y control.
    \begin{itemize}
        \item M6.1 Velocidad de rotación de mezclado.
        \item M6.2 Velocidad de rotación de extrusión.
        \item M6.3 Tiempo de mezclado.
        \item M6.4 Temperatura de extrusión.
        \item M6.5 Mostrar información de parámetros sensados.
    \end{itemize}
\end{itemize}
\begin{figure}[H]
    \centering
    \includegraphics[width=\textwidth]{ImagenesTT1/Arquitectura física.jpg}
    \caption{Arquitectura física}
    \label{fig:arquitecturafisica}
\end{figure}

\subsubsection{Resultados esperados}

El objetivo principal es mezclar y obtener un filamento de biopolímero que sea homogéneo, flexible y lo suficientemente rígido para ser compatible con una amplia gama de impresoras 3D comerciales. Sin embargo, el encargado de valorar la precisión y homogeneización completa del filamento serán los operadores del sistema en el laboratorio a cargo de la Dr. Luz Arcelia García Serrano.

Por otro lado, durante el proceso de desarrollo, los técnicos que manejen el sistema, podrían crear una base de datos que contendrá información detallada sobre las mejores prácticas en cada etapa de la producción de biopolímeros, en donde contemplen las cantidades de mezcla, temperaturas y velocidades ocupadas en el proceso. 

Como parte de esto, en el laboratorio del CIIEMAD, se llevarán a cabo una serie de pruebas de calidad extras, que incluyen observación microscópica, pruebas de rigidez y flexibilidad, y verificación del diámetro constante en todo el filamento. Estas pruebas son cruciales para mejorar y refinar el producto, asegurando que cumple con los requisitos de calidad. Cabe mencionar que estas pruebas no tienen relación directa con el proyecto, pero se toma en cuenta como forma de verificación, para comprobar las propiedades del filamento

De manera analógica, se realizará una comparación con filamentos comerciales existentes en el mercado. Esto permitirá evaluar si el filamento de biopolímero desarrollado puede competir y entrar en el mercado comercial, garantizando su viabilidad económica. Esto como forma meramente exploratoria, en donde buscamos una comparación pero no una competencia. 

Finalmente, la prueba final consistirá en la impresión 3D utilizando el filamento de biopolímero desarrollado. El objetivo es verificar si el producto final cumple con los estándares iniciales establecidos al inicio del proyecto. 



\subsubsection{Selección del concepto solución}

Para comenzar con el diseño conceptual de la máquina, es fundamental generar bocetos iniciales que sirvan como punto de partida en el proceso. Estos bocetos son el resultado de una lluvia de ideas que busca explorar diferentes enfoques y soluciones para los desafíos planteados por el proyecto. 

En el caso específico de este proyecto, centrado en el diseño de una máquina para la producción de filamentos, los bocetos iniciales han sido presentados en las Figuras \ref{fig:Diseño1}, \ref{fig:Diseño2}, \ref{fig:Diseño3}. A partir de estos bocetos, se inicia un proceso de análisis y evaluación que tiene como objetivo seleccionar el diseño conceptual más adecuado. 

\begin{figure}[H]
    \centering
    \includegraphics[width=1\textwidth]{ImagenesTT1/Diseño1.png}
    \caption{Bosquejo 1 del concepto solución}
    \label{fig:Diseño1}
\end{figure}

En este primero boceto solución se propone una distribución en forma de L para la máquina de producción de filamentos. En este diseño, el proceso de producción se organiza en una disposición lineal, donde el material ingresa en la tolva de manera vertical y avanza a través de una serie de etapas, incluyendo la zona de mezclado, extrusión y enfriamiento. 

Esta distribución horizontal de las parte extrusora proporciona una estructura modular que permite un montaje sencillo de los componentes y una mayor estabilidad mecánica. Además, esta disposición facilita el acceso a los diferentes elementos de la máquina, lo que simplifica las tareas de mantenimiento y reparación. 

\newpage
Sin embargo, esta propuesta puede requerir más espacio debido a su superficie estructural, y los componentes están más expuestos al ambiente, lo que puede aumentar la necesidad de mantenimiento regular.

\begin{figure}[H]
    \centering
    \includegraphics[width=1\textwidth]{ImagenesTT1/Diseño2.png}
    \caption{Bosquejo 2 del concepto solución}
    \label{fig:Diseño2}
\end{figure}

El segundo boceto solución propone una distribución mixta para la máquina, combinando elementos de distribución horizontal y vertical. En este diseño, se presenta una estructura compacta que protege los módulos internos de la máquina, reduciendo así el riesgo de accidentes. 

A pesar de eso, esta disposición puede presentar desafíos en términos de acceso a los componentes internos para mantenimiento y ensamblaje. Además, la presencia de una estructura cerrada puede generar problemas de estabilidad debido a vibraciones internas, y las altas temperaturas dentro de la estructura pueden afectar el rendimiento de los componentes.

\begin{figure}[H]
    \centering
    \includegraphics[width=1\textwidth]{ImagenesTT1/Diseño3.png}
    \caption{Bosquejo 3 del concepto solución}
    \label{fig:Diseño3}
\end{figure}

El tercer boceto solución propone una distribución mixta abierta para la máquina de producción de filamentos. En este diseño, se elimina la estructura cerrada en favor de una disposición más abierta y accesible. Un agitador montado a un votor vertical permite una mezcla pertinente, además en la siguiente etapa un tornillo extrusor simple impulsa y funde el material, que luego es guiado por una estructura abierta hacia un carrete montado hacie el sistema de enfriamiento. 

Esta disposición ofrece una menor probabilidad de accidentes al no tener una estructura cerrada, y puede facilitar el acceso a los componentes para mantenimiento y ensamblaje. Sin embargo, la falta de una estructura cerrada puede resultar en una menor estabilidad mecánica y un control reducido sobre el ambiente interno, lo que podría afectar el rendimiento térmico y aumentar la necesidad de mantenimiento y limpieza.

\newpage

Cada propuesta de diseño presenta sus propias ventajas y desventajas, y es esencial realizar una evaluación exhaustiva para comprender plenamente las implicaciones de cada opción. En este contexto, se emplearán herramientas como tablas de ventajas y desventajas (V/D) y el método AHP (Analytic Hierarchy Process) para proporcionar una base sólida y objetiva para la toma de decisiones.

\begin{itemize}
    \item \textbf{Propuesta 1: Distribución Horizontal en forma de L}
    \begin{itemize}
        \item Ventajas:
        \begin{itemize}
            \item Menor material empleado en la estructura.
            \item Mejor flujo de aire debido a la exposición al ambiente.
            \item Facilidad de montaje de los módulos.
            \item Mayor estabilidad mecánica.
        \end{itemize}
        \item Desventajas:
        \begin{itemize}
            \item Sistema propenso a accidentes.
            \item Más espaciosa debido a la superficie estructural.
            \item Mantenimiento recurrente debido a la exposición al ambiente de los componentes.
        \end{itemize}
    \end{itemize}
    
    \item \textbf{Propuesta 2: Distribución Mixta}
    \begin{itemize}
        \item Ventajas:
        \begin{itemize}
            \item Compacto, cómodo, de fácil acomodo y traslado.
            \item Módulos protegidos por la estructura del sistema; poco riesgo de accidentes.
        \end{itemize}
        \item Desventajas:
        \begin{itemize}
            \item Posible inestabilidad debido a vibraciones dentro de la estructura.
            \item Difícil acceso a los componentes del interior del sistema; difícil ensamblaje.
            \item Altas temperaturas en el interior de la estructura pueden perjudicar el rendimiento de los componentes internos.
        \end{itemize}
    \end{itemize}

    \newpage
    \item \textbf{Propuesta 3: Distribución Mixta Abierta}
    \begin{itemize}
        \item Ventajas:
        \begin{itemize}
            \item Menor riesgo de accidentes al no ser una estructura cerrada.
            \item Posible mayor facilidad de acceso a los componentes para mantenimiento y ensamblaje.
            \item Potencial para un diseño más compacto que la primera propuesta debido a la eliminación de la estructura cerrada.
        \end{itemize}
        \item Desventajas:
        \begin{itemize}
            \item Posible inestabilidad debido a la falta de una estructura cerrada para proporcionar soporte.
            \item Menor control sobre el flujo de aire y la temperatura del entorno, lo que puede afectar el rendimiento térmico.
            \item Mayor exposición al ambiente, lo que puede aumentar la necesidad de mantenimiento y limpieza.
        \end{itemize}
    \end{itemize}
\end{itemize}


\begin{table}[h]
    \centering
    \begin{tabular}{lccccc}
        \toprule
        \rowcolor{gray!30}
        \textbf{Criterios} & \textbf{Ponderación} & \textbf{Propuesta 1} & \textbf{Propuesta 2} & \textbf{Propuesta 3} & \textbf{Total} \\ \midrule
        Estabilidad Mecánica & 0.25 & 0.9 & 0.6 & 0.7 &  \\
        Riesgo de Accidentes & 0.2 & 0.5 & 0.8 & 0.9 &  \\
        Espacio Requerido & 0.2 & 0.4 & 0.8 & 0.6 &  \\
        Mantenimiento & 0.35 & 0.7 & 0.5 & 0.6 &  \\ \midrule
        \rowcolor{gray!15}
        \multicolumn{2}{c}{\textbf{Total}} & \textbf{0.76} & \textbf{0.675} & \textbf{0.725} & \textbf{1.0} \\ \bottomrule
    \end{tabular}
    \caption{Tabla AHP para Selección del Concepto Solución}
    \label{tab:ahp}
\end{table}

Basándonos en el análisis de las ventajas y desventajas de cada propuesta, así como en los resultados del método AHP, la propuesta de distribución horizontal en forma de L con la etapa de mezclado vertical, se destaca como la opción más equilibrada y viable para el proyecto. Aunque presenta algunas desventajas, como la necesidad de más espacio y un mantenimiento recurrente, sus ventajas en términos de estabilidad mecánica, flujo de aire y facilidad de montaje la hacen la opción más favorable en comparación con las otras propuestas.

\subsection{Diseño detallado}

\subsection{Detalle módulo 1 ($M_1$)}

Para diseñar el módulo de mezclado adecuadamente con el fin de que tanto el polímero sintético como el natural puedan combinarse de manera uniforme es necesario considerar diversos factores.

Primero que nada, debemos de conocer las propiedades de los materiales que se utilizarán en la mezcla, la que nos interesa principalmente es la densidad.

\subsubsection{Propiedades de los Materiales}

\begin{table}[h!]
\centering
\begin{tabular}{|c|c|}
\hline
Material & Densidad \\
\hline
ABS & 1.0 - 1.05 g/cm³ \\
PLA & 1.25 g/cm³ \\
PETG & 1.27 g/cm³ \\
Caucho natural (Guayule) & 0.92 g/cm³ (Aprox.) \\
Agua & 1.0 g/cm³ \\
\hline
\end{tabular}
\caption{Densidad de los materiales utilizados}
\end{table}

El objetivo inicial de los requerimientos era utilizar de entre 1 a 3 kg de mezcla. Sin embargo, la Dra. Luz, líder del proyecto, modificó estos requisitos, reduciendo la cantidad de mezcla para los experimentos a un rango de entre 0.5 kg - 1 kg. Además, la mezcla se compondrá de biopolímero, polímero sintético (pellets) y agua en distintas proporciones. A continuación, se presentan tres opciones de concentración para dicha mezcla:
\begin{itemize}
    \item Mezcla 1: 10\% Biopolímero, 80\% Pellets, 10\% Agua
    \item Mezcla 2: 15\% Biopolímero, 75\% Pellets, 10\% Agua
    \item Mezcla 3: 10\% Biopolímero, 75\% Pellets, 15\% Agua
\end{itemize}

\newpage

Esto es en porcentaje, para calcular la cantidad de masa que se requiere de cada uno, consideramos la cantidad máxima de masa total de la mezcla que en este caso es de 1 kg. 

\begin{itemize}
    \item Mezcla 1: 0.1 kg Biopolímero, 0.8 kg Pellets, 100 mL Agua
    \item Mezcla 2: 0.15 kg Biopolímero, 0.75 kg Pellets, 100 mL Agua
    \item Mezcla 3: 0.1 kg Biopolímero, 0.75 kg Pellets, 150 mL Agua
\end{itemize}
Para calcular el volumen ocupado por cada componente de la mezcla, emplearemos la relación que existe entre la masa y el volumen a través de la densidad de cada material.
\begin{equation}
    \rho=\frac{m}{V}
\end{equation}
\begin{equation}
    V = \frac{m}{\rho}
\end{equation}
Cabe mencionar que el biopolímero y polímero a utilizar serán los de menor densidad, dado que la son los que ocuparían un volumen mayor.

Utilizando la Ec.(2) procedemos a realizar los cálculos de volumen para cada mezcla. Cabe mencionar que el volumen total será manejado en pulgadas cúbicas debido a la practicidad para establecer medidas en el diseño utilizando pulgadas.

\textbf{Mezcla 1}

\textbf{Biopolímero}
\begin{equation*}
V_B = \frac{m_B}{\rho_B} = \frac{0.1\, \text{kg}}{(0.92\, \text{g/cm}^3)} = 108.69565\, \text{cm}^3 \left( \frac{1\, \text{kg}}{1000\, \text{g}} \right) \left( \frac{1\times10^{-6}\, \text{m}^3}{1\, \text{cm}^3} \right) = 1.0869\times10^{-4}\, \text{m}^3
\end{equation*}

\textbf{Pellets}
\begin{equation*}
V_P = \frac{m_P}{\rho_P} = \frac{0.8\, \text{kg}}{(1\, \text{g/cm}^3)} = 800\, \text{cm}^3 \left( \frac{1\, \text{kg}}{1000\, \text{g}} \right) \left( \frac{1\times10^{-6}\, \text{m}^3}{1\, \text{cm}^3} \right) = 8\times10^{-4}\, \text{m}^3
\end{equation*}

\textbf{Agua}
\begin{equation*}
V_A = \frac{m_A}{\rho_A} = \frac{0.1\, \text{kg}}{(1\, \text{g/cm}^3)} = 100\, \text{cm}^3 \left( \frac{1\, \text{kg}}{1000\, \text{g}} \right) \left( \frac{1\times10^{-6}\, \text{m}^3}{1\, \text{cm}^3} \right) = 1\times10^{-4}\, \text{m}^3
\end{equation*}

\textbf{Total}
\begin{equation*}
V_T = V_B + V_P + V_A = 1.0869\times10^{-4}\, \text{m}^3 + 8\times10^{-4}\, \text{m}^3 + 1\times10^{-4}\, \text{m}^3
\end{equation*}
\begin{equation*}
V_T = 1.00869\times10^{-3}\, \text{m}^3 \left( \frac{61023.7\, \text{in}^3}{1\, \text{m}^3} \right)
\end{equation*}
\begin{equation*}
V_T = 61.55\, \text{in}^3
\end{equation*}
\textbf{Mezcla 2}

\textbf{Biopolímero}
\begin{equation*}
V_B = \frac{m_B}{\rho_B} = \frac{0.15\, \text{kg}}{(0.92\, \text{g/cm}^3)} = 163.0434\, \text{cm}^3 \left( \frac{1\, \text{kg}}{1000\, \text{g}} \right) \left( \frac{1\times10^{-6}\, \text{m}^3}{1\, \text{cm}^3} \right) = 1.6304\times10^{-4}\, \text{m}^3
\end{equation*}

\textbf{Pellets}
\begin{equation*}
V_P = \frac{m_P}{\rho_P} = \frac{0.75\, \text{kg}}{(1\, \text{g/cm}^3)} = 750\, \text{cm}^3 \left( \frac{1\, \text{kg}}{1000\, \text{g}} \right) \left( \frac{1\times10^{-6}\, \text{m}^3}{1\, \text{cm}^3} \right) = 7.5\times10^{-4}\, \text{m}^3
\end{equation*}

\textbf{Agua}
\begin{equation*}
V_A = \frac{m_A}{\rho_A} = \frac{0.1\, \text{kg}}{(1\, \text{g/cm}^3)} = 100\, \text{cm}^3 \left( \frac{1\, \text{kg}}{1000\, \text{g}} \right) \left( \frac{1\times10^{-6}\, \text{m}^3}{1\, \text{cm}^3} \right) = 1\times10^{-4}\, \text{m}^3
\end{equation*}

\textbf{Total}
\begin{equation*}
V_T = V_B + V_P + V_A = 1.6304\times10^{-4}\, \text{m}^3 + 7.5\times10^{-4}\, \text{m}^3 + 1\times10^{-4}\, \text{m}^3
\end{equation*}
\begin{equation*}
V_T = 1.013\times10^{-3}\, \text{m}^3 \left( \frac{61023.7\, \text{in}^3}{1\, \text{m}^3} \right)
\end{equation*}
\begin{equation*}
V_T = 61.819\, \text{in}^3
\end{equation*}

\newpage

\textbf{Mezcla 3}

\textbf{Biopolímero}
\begin{equation*}
V_B = \frac{m_B}{\rho_B} = \frac{0.1\, \text{kg}}{(0.92\, \text{g/cm}^3)} = 108.69565\, \text{cm}^3 \left( \frac{1\, \text{kg}}{1000\, \text{g}} \right) \left( \frac{1\times10^{-6}\, \text{m}^3}{1\, \text{cm}^3} \right) = 1.0869\times10^{-4}\, \text{m}^3
\end{equation*}

\textbf{Pellets}
\begin{equation*}
V_P = \frac{m_P}{\rho_P} = \frac{0.75\, \text{kg}}{(1\, \text{g/cm}^3)} = 750\, \text{cm}^3 \left( \frac{1\, \text{kg}}{1000\, \text{g}} \right) \left( \frac{1\times10^{-6}\, \text{m}^3}{1\, \text{cm}^3} \right) = 7.5\times10^{-4}\, \text{m}^3
\end{equation*}

\textbf{Agua}
\begin{equation*}
V_A = \frac{m_A}{\rho_A} = \frac{0.15\, \text{kg}}{(1\, \text{g/cm}^3)} = 150\, \text{cm}^3 \left( \frac{1\, \text{kg}}{1000\, \text{g}} \right) \left( \frac{1\times10^{-6}\, \text{m}^3}{1\, \text{cm}^3} \right) = 1.5\times10^{-4}\, \text{m}^3
\end{equation*}

\textbf{Total}
\begin{equation*}
V_T = V_B + V_P + V_A = 1.0869\times10^{-4}\, \text{m}^3 + 7.5\times10^{-4}\, \text{m}^3 + 1.5\times10^{-4}\, \text{m}^3
\end{equation*}
\begin{equation*}
V_T = 1.00869\times10^{-3}\, \text{m}^3 \left( \frac{61023.7\, \text{in}^3}{1\, \text{m}^3} \right)
\end{equation*}
\begin{equation*}
V_T = 61.55\, \text{in}^3
\end{equation*}

De tal manera que el volumen máximo será ocupado cuando se realice la mezcla 2, por lo que se tomará dicho volumen como referencia para el diseño de la tolva. Considerando este volumen, de igual manera agregaremos una tolerancia que en este caso se sugiere de un 15\% más del volumen máximo, esto con el fin de añadir más seguridad en el diseño ante malas mediciones de las proporciones que ingrese el usuario.

Entonces, el volumen total será:
\begin{equation*}
V_T = 61.819\, \text{in}^3 + 0.15(61.819\, \text{in}^3)
\end{equation*}
\begin{equation*}
V_T = 71.1\, \text{in}^3
\end{equation*}

\subsubsection{Diseño de la tolva}
El diseño de la tolva se basará en una forma de cono truncado invertido, elegida por su practicidad. Este diseño permite una entrada amplia y facilita el uso de la fuerza de gravedad en la salida. De esta manera, cuando la mezcla esté lista, podrá avanzar a la siguiente etapa sin necesidad de mecanismos adicionales. Para calcular el volumen de la tolva, utilizaremos la Ec.(3).
\begin{equation}
V = \frac{1}{3}\pi h(R^2 + Rr + r^2)
\end{equation}

\begin{figure}[H]
    \centering
    \includegraphics[width=0.3\textwidth]{ImagenesTT1/Cono_truncado.png}
    \caption{Cono truncado}
    \label{fig:Cono_truncado}
\end{figure}

Ahora, para el diseño de la pieza nos ayudaremos de un software CAD, donde es posible dibujar el croquis de la forma del diámetro más pequeño y extenderlo con un ángulo a una determinada distancia, por lo que tendríamos que definir solo a r y el ángulo en el que se va a extender la extrusión, una vista 2D de lo anterior se muestra en la Figura \ref{fig:Vista_2D}.
\begin{figure}[H]
    \centering
    \includegraphics[width=0.4\textwidth]{ImagenesTT1/Vista2D.PNG}
    \caption{Vista 2D para diseño de tolva}
    \label{fig:Vista_2D}
\end{figure}
Donde r se propone de 0.455 in, con un ángulo de salida de 35°. Teniendo como resultado las siguientes expresiones para los parámetros de la formula del volumen:

\begin{equation}
R = r + b
\end{equation}

Donde 

\begin{equation}
b = h\tan(35^\circ)
\end{equation}

Entonces
\begin{equation}
R = r + h\tan(35^\circ)
\end{equation}

Sustituimos r y R en la fórmula del volumen, así como el volumen máximo:

\begin{equation*}
    71.1 = \frac{1}{3}\pi h\left[(h\tan(35^\circ) + 0.455\right)^2 + \left(h\tan(35^\circ) + 0.455\right)(0.455) + (0.455)^2]
\end{equation*}

\begin{equation*}
   71.1 = \frac{1}{3}\pi h\left[h^2 \tan^2(35^\circ) + 0.455h\tan(35^\circ) + 0.455^2 + 0.455h\tan(35^\circ) + 0.455^2 + 0.455^2\right]
\end{equation*}

\begin{equation*}
    71.1 = \frac{1}{3}\pi h\left(0.49h^2 + 0.6372h + 0.621\right)
\end{equation*}

\begin{equation}
    0.5131h^3 + 0.66727h^2 + 0.65h - 71.1 = 0
\end{equation}

Resolviendo Ec.(7) para h, tenemos:

\begin{equation*}
    h = 4.7\, \text{in}
\end{equation*}

Llevando a cabo el diseño en CAD con todo lo anteriormente calculado, finalmente obtuvimos lo que se muestra en la Figura \ref{fig:Tolva_solid}.

\begin{figure}[H]
    \centering
    \includegraphics[width=0.85\textwidth]{ImagenesTT1/Tolva_solid.png}
    \caption{Volumen de la tolva según SolidWorks\textregistered}
    \label{fig:Tolva_solid}
\end{figure}

De la Figura \ref{fig:Tolva_solid} vemos que se obtuvo un volumen de 78.18 $\text{in}^3$, lo cual es incluso un poco mayor que lo calculado, por lo que perfectamente puede albergar en su interior 1 kg de mezcla con las condiciones que consideramos en los cálculos.

Ya definido lo que corresponderá a la tolva que es donde los ingredientes de la mezcla se mezclarán, ahora es importante definir el tipo de aspas que hará posible el mezclado para con ello determinar el motorreductor necesario según los cálculos que realizaremos. Para ello, es necesario conocer acerca del tipo de flujo que hay al realizar mezcla y cual de ellos nos conviene, esto es importante, ya que basándonos en el flujo que requerimos, seleccionaremos la forma de nuestras aspas. 

\subsubsection{Selección de forma de aspas}

\begin{itemize}
    \item \textbf{Agitadores de Flujo Axial:} Son ideales para la homogeneización y suspensión en volúmenes grandes y medios. Se destacan por su alta eficiencia en el bombeo y generan un flujo paralelo al eje de rotación. Se recomienda su uso en industrias como la química, minera, papelera, petrolera, y tratamiento de aguas. Sus impulsores varían desde hélices marinas hasta hélices con palas plegables, siendo efectivos tanto en productos de baja como de viscosidad moderada. Las velocidades de operación se encuentran principalmente entre 1150 a 1750 rpm con transmisión directa y 350 a 420 rpm con transmisión por engranajes, siendo los primeros más adecuados para reacciones rápidas y los segundos para la suspensión de sólidos.
    \item \textbf{Agitadores de Flujo Radial:} Los agitadores de flujo radial, como los de palas planas inclinadas, son muy versátiles y económicos. Se usan comúnmente en procesos de homogeneización, disolución, neutralización y preparación de reactivos. Aunque su eficiencia es más baja en comparación con los de flujo axial, su diseño simple los hace comunes en la industria.
    \item \textbf{Agitadores de Flujo Tangencial:} Estos agitadores, que incluyen los tipos ancla y helicoidal, son ideales para fluidos de alta viscosidad. Su diseño permite una mezcla eficiente cerca de las paredes del tanque, siendo óptimos para procesos que requieren un mezclado intensivo en estas áreas. Son particularmente efectivos en tareas de homogeneización, neutralización, coagulación e intercambio térmico.
\end{itemize}

Para nuestro caso, es conveniente usar una turbina, es decir, un agitador de flujo radial (Figura 19b), ya que estas nos permitirán una mayor homogenización de la mezcla por el movimiento que se observa en las demás alternativas.

Una turbina comúnmente usada para este tipo de flujo, es la turbina Rushton, también conocida como turbina de disco Rushton, es un impulsor de flujo radial utilizado en diversas aplicaciones de mezcla, especialmente eficaz en la dispersión de gas y aplicaciones de fermentación en la ingeniería de procesos. Fue inventada por John Henry Rushton y su diseño consiste en un disco horizontal plano con palas planas montadas verticalmente. Comúnmente, el disco cuenta con seis palas montadas verticalmente, que están ubicadas a 1/3 del disco y crean un flujo radial.

\begin{figure}[H]
    \centering
    \includegraphics[width=0.6\textwidth]{ImagenesTT1/Tipos de flujo.png}
    \caption{Tipos de flujo. Fuente: \cite{FlujoAgitadores}}
    \label{fig:Tipos_flujo}
\end{figure}

Las características clave de la turbina Rushton incluyen su capacidad para mezclar fluidos y gases de manera eficiente, manteniendo la estabilidad del elemento mezclador gracias a su flujo radial y su capacidad de autoequilibrio. El flujo se descarga radialmente hacia las paredes del recipiente, dirigiéndose la mitad del flujo hacia arriba y la otra mitad hacia abajo.

Una de las principales ventajas de la turbina Rushton es su alta capacidad de cizallamiento. Esto la hace ideal para aplicaciones en las que se requiere un alto nivel de turbulencia o cizallamiento, como en el contacto líquido-gas. El diseño de la turbina guía el gas a lo largo de una trayectoria de contacto óptimo con el líquido y previene que las burbujas de gas sigan un camino vertical directo a lo largo del eje del agitador, lo cual resultaría en un contacto mínimo. El disco también evita que las burbujas de gas pasen a través de la zona de bajo cizallamiento alrededor del cubo del impulsor. Esto se puede ver en las Figuras \ref{fig:Turbina_Rushton} y \ref{fig:Flujo_Rushton}

\begin{figure}[H]
  \centering
  \begin{minipage}{0.3\textwidth}
    \centering
    \includegraphics[width=\linewidth]{ImagenesTT1/Turbina Rushton.png}
    \caption{Turbina Rushton}
    \label{fig:Turbina_Rushton}
  \end{minipage}%
  \begin{minipage}{0.3\textwidth}
    \centering
    \includegraphics[width=\linewidth]{ImagenesTT1/Flujo Rushton.png}
    \caption{Flujo causado por la turbina Rushton}
    \label{fig:Flujo_Rushton}
  \end{minipage}
\end{figure}

\subsubsection{Selección del motorreductor}

Para la selección del motorreductor, primero es necesario calcular una potencia que vendrá dada en función del número de Reynolds, ya que si se trata de un flujo transitorio – laminar usaremos una determinada fórmula para el cálculo de la potencia y utilizaremos otra en caso de que el flujo resulte turbulento. 

La Ec.(8) describe el número de Reynolds para este caso en específico.

\begin{equation}
    nRe = \rho \ast N \ast \frac{D^2}{\mu}
\end{equation}

Donde:
\begin{itemize}
    \item $\rho$: Densidad de la sustancia.
    \item $N$: Revoluciones por segundo.
    \item $D$: Diámetro de la turbina.
    \item $\mu$: Viscosidad de la sustancia.
\end{itemize}

Cabe mencionar que, según estudios de la Dra. Luz, líder del proyecto, la viscosidad de la sustancia suele ser cercana a la de la miel de abeja, de la cual, realizando investigaciones sobre la misma, encontramos que la viscosidad de esta varía de diversas formas según factores como la temperatura, contenido de agua, composición de azúcar, proceso de cristalización, origen botánico, etc. Sin embargo, es posible encontrar una aproximación de la miel de abeja que se encuentra comúnmente de manera comercial y que se presenta a continuación en la gráfica recuperada del libro “Honey Analysis” \cite{ref29}.

\begin{figure}[H]
    \centering
    \includegraphics[width=0.75\textwidth]{ImagenesTT1/ViscosidadMiel.png}
    \caption{Gráfica Viscosidad vs Temperatura}
    \label{fig:Viscosidad_miel}
\end{figure}

Según está gráfica, muestra los resultados de experimentación que se aproximaron a una función, la cual logra ajustarse de manera casi exacta.
Considerando que la temperatura media anual en CDMX (lugar donde el proyecto residirá) es de 16 °C según datos del INEGI \cite{ref30}, considerando que llegue a temperaturas aún más bajas consideraremos la temperatura de la miel a aproximadamente 14 °C (287.15 K), y dado que la gráfica nos muestra un comportamiento que a mayor temperatura la viscosidad disminuye, tomaríamos este caso como nuestro caso crítico y por ende nuestro caso de estudio, donde la mezcla sería más difícil de mover.

\begin{equation}
\mu = (2\times10^{22})e^{-0.166T}
\end{equation}

\begin{equation*}
\mu = (2\times10^{22})e^{-0.166(287.15)}
\end{equation*}

\begin{equation*}
\mu = 39.77\, \text{Pa} \cdot \text{s} \approx 40\, \text{Pa} \cdot \text{s}
\end{equation*}

Ahora, considerando que el diámetro de la turbina será de 2.5 in (0.0635 m). Calculamos el número de Reynolds.

\begin{equation*}
nRe = \frac{1\, \text{kg}}{1.013\times10^{-3}\, \text{m}^3} \times (400\, \text{rpm}) \times \left(\frac{1\, \text{min}}{60\, \text{s}}\right) \times \left(\frac{(0.0635\, \text{m})^2}{40\, \text{Pa} \cdot \text{s}}\right)
\end{equation*}

\begin{equation*}
nRe = 0.6634 < 300 \quad \therefore \text{Es Laminar}
\end{equation*}


Para el caso donde el flujo es laminar, la potencia de cálculo estará dada por la Ec.(10).

\begin{equation}
P = KL \ast N^2 \ast D^3 \ast \mu
\end{equation}

Donde KL, está definido por el tipo de turbina, para el caso que nos corresponde es una turbina de seis palas planas. (Figura \ref{fig:Constantes})

\begin{figure}[H]
    \centering
    \includegraphics[width=0.85\textwidth]{ImagenesTT1/ConstanteKL.png}
    \caption{Constantes según el tipo de flujo y de turbina}
    \label{fig:Constantes}
\end{figure}

\newpage
Entonces el valor de KL es de 71.0 y es adimensional.

\begin{equation*}
P = 71.0 \ast (400\, \text{rpm})^2 \ast \left(\frac{1\, \text{min}}{60\, \text{s}}\right)^2 \ast (0.0635\, \text{m})^3 \ast (40\, \text{Pa} \cdot \text{s})
\end{equation*}

\begin{equation*}
P = 32.319\, \text{Watt}
\end{equation*}

\begin{equation*}
P = 32.319\, \text{W} \left(\frac{1\, \text{Hp}}{745.7\, \text{W}}\right) = 0.0433\, \text{Hp}
\end{equation*}

Tomando en cuenta la eficiencia del motor de un 85\% y pérdidas por fricción de aproximadamente 20\%, la potencia real necesaria estaría dada por:
\begin{equation*}
P_{\text{real}} = \frac{32.319 \times 1.20}{0.85} = 45.63\, \text{Watt}
\end{equation*}

Para optimizar la selección de un motor adecuado para un agitador, se deben considerar criterios clave para asegurar un funcionamiento eficiente y acorde a las necesidades del proceso. Estos incluyen:

•	Capacidad de Potencia y Velocidad: Es fundamental que el motor cumpla con los requerimientos específicos de potencia y velocidad del agitador, en este caso, 400 rpm y 45.63 W, para garantizar una mezcla efectiva.

•	Operación al Aire Libre: El motor debe ser apto para funcionar en condiciones de intemperie, lo cual implica una robustez y protección adecuada contra elementos externos.

•	Montaje con Brida Vertical: Se debe asegurar que el motor sea compatible con un montaje vertical mediante brida, lo cual es importante para la integración con el sistema de agitación.

•	Control Electrónico: El motor debe ser capaz de integrarse con sistemas de control electrónico para una regulación precisa de la velocidad y otros parámetros operativos.

•	Cumplimiento de Normas Internacionales: Es esencial que el motor cumpla con las normas de fabricación internacional, como las Normas IEC, garantizando así su calidad, seguridad y confiabilidad.

La elección cuidadosa que considerará estos aspectos asegurará que el motor no solo sea adecuado para la aplicación específica, sino también duradero y eficiente en su operación.

Consultando el catálogo del fabricante chino Xin Da Motor (Figura \ref{fig:Motorselec}), encontramos un motor (Figura \ref{fig:Motor}) que nos ofrece 60 W de potencia a una velocidad que puede ser variada de 10 rpm – 600 rpm, por lo que cubre con creces lo que se requiere.

\begin{figure}[H]
    \centering
    \includegraphics[width=0.9\textwidth]{ImagenesTT1/ParametrosMotor.png}
    \caption{Parámetros del motor seleccionado. Fuente:\cite{SeleccionMotor}}
    \label{fig:Motorselec}
\end{figure}

\begin{figure}[H]
    \centering
    \includegraphics[width=0.6\textwidth]{ImagenesTT1/MotorSeleccionado.png}
    \caption{Motor seleccionado. Fuente: \cite{SeleccionMotor}}
    \label{fig:Motor}
\end{figure}

\subsubsection{Cálculo del diámetro del eje}

El cálculo del diámetro de un eje según la norma ASME implica considerar tanto la rigidez como la resistencia necesaria para transmitir potencia y manejar cargas combinadas. Según el código de ASME, el diseño del eje incluye determinar el diámetro adecuado del eje para garantizar la fuerza y rigidez suficientes cuando se transmita potencia del motor o del motor bajo diversas condiciones de funcionamiento. Los ejes suelen ser redondos y pueden ser sólidos o huecos.

La fórmula que se muestra en la Ec.(11) es para el cálculo del diámetro de un eje macizo bajo cargas de torsión y flexión mínimas o nulas, lo que generalmente es aplicable a muchos escenarios en la ingeniería mecánica, particularmente en el diseño de ejes de transmisión de potencia.

\begin{equation}
D^3 = \frac{16}{\pi \tau d} \sqrt{(K_sT)^2 + \left(\frac{FD^2}{8}\right)^2}
\end{equation}

Para esta ecuación, se tiene:
\begin{itemize}
    \item $D$ es el diámetro del eje.
    \item $\tau$ es el esfuerzo cortante permitido para el material del eje.
    \item $d$ es el factor de diseño basado en el esfuerzo de fluencia ($S_y$) o el esfuerzo de ruptura ($S_f$).
    \item $K_s$ es el coeficiente del momento flector.
    \item $T$ es el momento torsor.
    \item $F$ es la fuerza axial.
    \item $D$ es también el diámetro del eje en el segundo término.
\end{itemize}

La presencia del término bajo la raíz cuadrada y elevado al cuadrado sugiere que esta fórmula utiliza el principio de superposición de esfuerzos de manera cuadrática para considerar el efecto combinado de torsión y flexión sobre el eje.

Primero, calculamos la fuerza axial, con la expresión de la Ec.(12).

\begin{equation}
F = \frac{T}{r}
\end{equation}

Donde \( r \) es el radio de las aspas y \( T \) es el par torsor, que se obtiene de la siguiente fórmula:
\begin{equation*}
T = \frac{P}{2\pi N} = \frac{45.63\, \text{W}}{2\pi (400\, \text{rpm})} \times \left(\frac{1\, \text{min}}{60\, \text{s}}\right) = 1.0893\, \text{Nm}
\end{equation*}

Entonces, la fuerza estará dada por:
\begin{equation*}
F = \frac{1.0893\, \text{Nm}}{0.0635\, \text{m} / 2} = 34.31\, \text{N}
\end{equation*}

Ahora bien, es importante definir el material del que estará hecho nuestro eje, esto para conocer tanto el esfuerzo de fluencia como el esfuerzo de ruptura.
Tras investigar las propiedades de diferentes tipos de acero, encontramos que una de las opciones que podemos utilizar es el \textbf{Annealed 304 Stainless Steel}, ya que por ser un acero inoxidable nos ofrece una buena resistencia a la corrosión.

\begin{figure}[H]
    \centering
    \includegraphics[width=0.6\textwidth]{ImagenesTT1/PropiedadesAISI304.png}
    \caption{Propiedades del AISI 304. Fuente:\cite{PropiedadesAISI304}}
    \label{fig:AISI304}
\end{figure}

\newpage

Entonces tenemos que \( S_y = 310\, \text{MPa} \) y \( S_r = 700\, \text{MPa} \).

Calculamos \( \tau d \):
\[ \tau d_{1} = 0.3 \ast S_y = 0.3 \ast 230\, \text{MPa} = 69\, \text{MPa} \]
\[ \tau d_{2} = 0.18 \ast S_r = 0.18 \ast 580\, \text{MPa} = 104.4\, \text{MPa} \]

El menor es \( \tau d_{1} \), entonces será el que utilizaremos, pero agregaremos un factor de seguridad.
\[ \tau_{d1} = 69\, \text{MPa} \ast 0.75 \]
\[ \tau_{d1} = 51.75\, \text{MPa} \]

Los valores del coeficiente del momento flector (Ks) para la carga ligera de un agitador de fluidos:

Ks: 1.5 -- 2

Entonces utilizaremos el máximo de ellos para considerar un caso crítico.

Ahora, calculamos el torque:
\[
T = \frac{7162000 \ast Hp}{\text{rpm}} = \frac{7162000 \ast 0.061Hp}{400\, \text{rpm}}
\]
\[ T = 1092.205\, \text{N} \text{--} \text{mm} \]

Sustituimos todos los valores en la ecuación para el diámetro:
\[ D^3 = \frac{16}{\pi \tau_d} \sqrt{(K_sT)^2 + \left(\frac{FD^2}{8}\right)^2} \]
\[ D^3 = \frac{16}{\pi(51.75)} \sqrt{(2 \ast 1092.205)^2 + \left(\frac{34.31D^2}{8}\right)^2} \]

Resolviendo para D, se obtiene:
\[ D = 6\, \text{mm} \]

\subsubsection{Ensamblaje del sistema de mezclado}

A continuación, se observa el dibujo CAD de los componentes que integran el sistema de mezclado.

\begin{figure}[H]
    \centering
    \includegraphics[width=0.4\linewidth]{ImagenesTT1/RenderMezclado.JPG}
    \caption{Ensamblaje del sistema de mezclado}
    \label{fig:SistemaMezclado}
\end{figure}

En la Figura \ref{fig:SistemaMezclado}, se puede observar que, además del eje en la parte superior del sistema, se añadió un acople. Este acople tiene la función de unir la parte rotativa del motor con el eje, ya que una conexión directa entre el eje y el motor causaría inestabilidad. Al utilizar el acople y fijarlo con tornillos M2, se asegura una mejor fijación y estabilidad del conjunto.

\begin{figure}[H]
    \centering
    \includegraphics[width=0.3\linewidth]{ImagenesTT1/RenderAcopleMezcla.JPG}
    \caption{Acople del sistema de mezclado}
    \label{fig:acople_mezclado}
\end{figure}

\subsubsection{Validación de piezas del módulo de mezclado}

Para asegurar la correcta implementación del sistema de mezclado, es esencial simular las condiciones operativas a las que estarán sometidas las piezas que lo componen. Mediante estas simulaciones, podemos verificar teóricamente que las piezas no fallarán bajo las condiciones previstas y confirmar que el diseño es adecuado.

\textbf{Validación del soporte de motor}

La primera pieza a estudiar será el soporte del motor de mezclado. Esta pieza es fundamental, ya que el motor estará orientado verticalmente, lo que presenta desafíos debido a la gravedad. Una correcta sujeción es esencial para evitar deslizamientos hacia abajo y para mitigar las vibraciones que se generen durante el funcionamiento del motor.

El diseño de este soporte fue recuperado de un libro de prácticas de SolidWorks\textregistered para Ingeniería Electromecánica. En la Figura \ref{fig:PlanoSoporteMotor} se muestra el plano original de donde se recuperó. Cabe mencionar que se realizaron las modificaciones pertinentes para adaptarlo al motor que se utilizará en el sistema.

Es importante destacar el tornillo ubicado en la parte superior del soporte. Este tornillo, junto con una tuerca, ayuda a reducir las vibraciones. En caso de que se presenten inconvenientes, se puede ajustar apretando más el tornillo. El límite de ajuste está definido por una ranura de 2 mm, lo que garantiza que el soporte mantenga su integridad y funcionalidad sin comprometer la estabilidad del motor.

Ahora, la pieza de soporte junto con el motor tendrán la configuración que se muestra en la Figura \ref{fig:SoporteMotor}.

\begin{figure}[H]
    \centering
    \includegraphics[width=0.5\linewidth]{ImagenesTT1/RenderSoporteMotor.JPG}
    \caption{Soporte con motor}
    \label{fig:SoporteMotor}
\end{figure}

\begin{figure}[H]
    \centering
    \includegraphics[width=\textwidth]{ImagenesTT1/PlanoSoporteMotor.png}
    \caption{Plano del soporte del motor \cite{solidworks_manual}}
    \label{fig:PlanoSoporteMotor}
\end{figure}

Mediante simulaciones de elementos finitos (FEA), se modelarán las condiciones operativas y se someterá el soporte a pruebas de esfuerzo y fatiga. Esto nos permitirá identificar posibles puntos de fallo y optimizar el diseño antes de la fabricación. 

El primer paso será estudiar la pieza de soporte por separado. En la Figura \ref{fig:SoporteMotor}, solo se colocó el motor para ilustrar la idea de la implementación. Para el análisis de esfuerzos, solo necesitaremos conocer el peso del motor y obtener una estimación de las vibraciones que genera, las cuales serán representadas mediante una torsión.

Para el estudio de esfuerzos, se debe definir el material de la pieza, en este caso utilizaremos el mismo con el que se ha venido trabajando las piezas del sistema de mezclado, es decir, el AISI 304, por las propiedades ya mencionadas anteriormente. Después en la pieza agregaremos las conexiones correspondientes a los tornillos que deben de estar para sujetar la pieza a la pared y el tornillo con tuerca para el cierre de la pieza. (Figuras \ref{fig:DescripcionTornillosSoporte} y \ref{fig:TornillosSoporte1})

\begin{figure}[H]
  \centering
  \begin{minipage}{0.4\textwidth}
    \centering
    \includegraphics[width=\linewidth]{ImagenesTT1/DescripcionTornillosSoporte.png}
    \caption{Descripción de los tornillos del soporte}
    \label{fig:DescripcionTornillosSoporte}
  \end{minipage}%
  \begin{minipage}{0.4\textwidth}
    \centering
    \includegraphics[width=\linewidth]{ImagenesTT1/TornillosSoporte1.png}
    \caption{Soporte con tornillos}
    \label{fig:TornillosSoporte1}
  \end{minipage}
\end{figure}

La correcta aplicación del torque de apriete en los tornillos es crucial para asegurar la integridad y fiabilidad de las uniones atornilladas. A continuación, se detalla el proceso de cálculo del par de apriete para tornillos  utilizando la norma UNE 17-108-81 que proporciona un método detallado el cálculo, basándose en varios parámetros técnicos.

\newpage

\textbf{Fórmulas y Parámetros Utilizados}

\textbf{Cálculo de la Fuerza de Tracción (\( F_T \))}

Para calcular la fuerza de tracción (\( F_T \)) se utiliza la siguiente expresión:

\begin{equation} \label{eq:FT}
F_T = \frac{0.8 E A_s}{\sqrt{1 + 3 \left( \frac{4}{1 + \frac{d_3}{d_2}} \left( \frac{P}{\pi d_2} + 1.155 \mu \right) \right)^2}}
\end{equation}

Donde:
\begin{itemize}
    \item \( E \) es el límite elástico en \( \text{N/mm}^2 \), obtenido del valor \( R_p 0.2 \) de la norma ISO 898-1 (ISO) para cada una de las clases de calidad.
    \item \( A_s \) es la sección resistente en \( \text{mm}^2 \), obtenida de la norma ISO 965-parte 2 en función de la métrica de la rosca.
    \item \( P \) es el paso de la rosca en milímetros, obtenido de la norma ISO 965-parte 2.
    \item \( \mu \) es el coeficiente de rozamiento entre la cabeza del tornillo y el apoyo (se considera similar al coeficiente de rozamiento de la rosca).
    \item \( d_2 \) es el diámetro medio de la rosca del tornillo expresado en mm, obtenido de la norma ISO 965-parte 2.
    \item \( d_3 \) es el diámetro del núcleo de la rosca del tornillo, expresado en mm, obtenido de la norma ISO 965-parte 2.
\end{itemize}

\textbf{Cálculo del Momento de Torsión (Torque) (\( M_T \))}

El momento de torsión (\( M_T \)) se puede calcular usando la siguiente expresión:

\begin{equation} \label{eq:MT}
M_T = \frac{F_T}{1000} \left[ 0.161 P + 0.583 \mu d_2 + 0.25 \mu (D_e + D_i) \right]
\end{equation}

\newpage

Donde:

\begin{itemize}
    \item \( D_e \) es el diámetro máximo de la superficie de rozamiento de la cabeza del tornillo.
    \item \( D_i \) es el diámetro mínimo de la superficie de rozamiento de la cabeza del tornillo.
\end{itemize}

\textbf{Aplicación a Tornillos M12 con Paso de 1 mm}

Para tornillos M12 como los que se utilizan en el soporte, consideramos las siguientes especificaciones estándar:
\begin{itemize}
    \item Diámetro nominal (\( d \)): 12 mm
    \item Paso de la rosca (\( P \)): 1 mm (ISO 965-1)
    \item Coeficiente de rozamiento (\( \mu \)): 0.1 (valor típico según la norma)
    \item Límite elástico para clase 8.8 (\( E \)): 640 \( \text{N/mm}^2 \)
    \item Área de la sección resistente (\( A_s \)): 84.3 \( \text{mm}^2 \)
    \item Diámetro medio de la rosca (\( d_2 \)): 10.84 mm
    \item Diámetro del núcleo de la rosca (\( d_3 \)): 10.2 mm
    \item Diámetro máximo de la superficie de rozamiento (\( D_e \)): 12.5 mm
    \item Diámetro mínimo de la superficie de rozamiento (\( D_i \)): 10.2 mm
\end{itemize}

\textbf{Cálculo de \( F_T \)}

Sustituyendo los valores en la fórmula:

\[
F_T = \frac{0.8 \times 640 \times 84.3}{\sqrt{1 + 3 \left( \frac{4}{1 + \frac{10.2}{10.84}} \left( \frac{1}{\pi \times 10.84} + 1.155 \times 0.1 \right) \right)^2}}
\]

El valor calculado de \( F_T \) es aproximadamente \textbf{38,339.7 N}.

\textbf{Cálculo de \( M_T \)}

Una vez obtenido \( F_T \), sustituimos en la fórmula del momento de torsión:

\[
M_T = \frac{F_T}{1000} \left[ 0.161 \times 1 + 0.583 \times 0.1 \times 10.84 + 0.25 \times 0.1 \times (12.5 + 10.2) \right]
\]

El valor calculado de \( M_T \) es aproximadamente \textbf{52.16 N·m}, entonces este valor será el utilizado para la simulación.

\subsubsection{Estudios y análisis}

Para llevar a cabo los estudios correspondientes, se ingresan los siguientes parámetros: la aceleración debida a la gravedad (9.81 m/s²), el peso del motor (1.9 kg), y la vibración representada como una torsión de aproximadamente 2 N-m. El resultado de las tensiones sobre el soporte del motor se presenta en la Figura \ref{fig:TensionesSoporte}.

\begin{figure}[H]
    \centering
    \includegraphics[width=0.7\linewidth]{ImagenesTT1/TensionesSoporte.png}
    \caption{Tensiones del soporte del motor}
    \label{fig:TensionesSoporte}
\end{figure}

\newpage
De la Figura \ref{fig:TensionesSoporte} podemos observar que en base a la escala que se muestra del lado derecho, las tensiones no superan el límite elástico del material que se utilizó para dicha pieza, lo que garantiza que nuestra pieza no fallará ante el peso del motor y la vibración que genera el mismo. 

Ahora, observaremos los resultados del análisis de desplazamiento:

\begin{figure}[H]
    \centering
    \includegraphics[width=0.65\linewidth]{ImagenesTT1/DesplazamientosSoporte.png}
    \caption{Desplazamientos del soporte del motor}
    \label{fig:DesplazamientosSoporte}
\end{figure}

\begin{figure}[H]
    \centering
    \includegraphics[width=0.65\linewidth]{ImagenesTT1/FatigaSoporte.png}
    \caption{Fatiga del soporte bajo carga constante}
    \label{fig:FatigaSoporte}
\end{figure}

El estudio de la Figura \ref{fig:DesplazamientosSoporte} representa que tantos mm se desplazará alguna parte de la pieza, en este caso vemos que donde se encuentra el tornillo con la tuerca para ajustar la sujeción del motor es la parte de mayor desplazamiento, y esto tiene sentido debido a que esta se ajustará para reducir las vibraciones que pueda ocasionar el motor al girar.

El último estudio realizado para el soporte fue un análisis de fatiga. Este análisis se llevó a cabo con el objetivo de determinar la duración operativa del motor montado sobre el soporte antes de que este sufra una deformación permanente (doblez) o falle (ruptura).

En la Figura \ref{fig:FatigaSoporte}, se muestra que el resultado del análisis es de 68,946 ciclos de vida. Considerando una velocidad máxima de 400 rpm, si multiplicamos este valor por los 60 minutos en una hora, obtenemos un total de 24,000 revoluciones, que pueden ser consideradas como ciclos. Por lo tanto, en un período de 2.5 horas se habrán alcanzado 60,000 ciclos, lo cual todavía está por debajo del número mínimo de ciclos de vida determinado por el análisis. Esto garantiza de manera efectiva la integridad del soporte bajo las condiciones máximas de operación del motor.

\textbf{Validación de acople, eje, turbina y tolva}

La validación para el acople, eje, turbina y tolva, deben hacerse en conjunto, esto dado que estos elementos interactúan de manera directa e indirecta como se logra observar en el ensamble de la Figura \ref{fig:SistemaMezclado}. 

De manera directa estarán sujetados por tornillos tanto el acople con el eje como el eje con la turbina, mientras que la tolva albergará la mezcla y lo que nos interesa en la validación de las piezas es observar que puedan soportar el par de torsión que provendrá del motor seleccionado mientras que al mismo tiempo habrá una presión directa sobre la turbina y el eje por parte del fluido. Cabe mencionar que todas las piezas han sido diseñadas en acero AISI 304.

Para poder analizar las condiciones anteriormente mencionadas, deberemos recurrir a una herramienta del software SolidWorks\textregistered, llamada Flow Simulation, dicha herramienta nos dará la posibilidad de simular el giro del motor, la descripción del flujo del fluido y así mismo, nos ayudará a determinar la presión que se ejercerá por los efectos del fluido.

Las propiedades utilizadas para el estudio de fluidos corresponden a las de la miel, que se empleará como referencia, dado que el proyecto está orientado a la experimentación con diversas mezclas y combinaciones. Cabe destacar que la viscosidad de la miel y cualquier líquido varía en función de la temperatura. 

Por lo tanto, al implementar el sistema de precalentamiento, descrito posteriormente, se puede reducir la viscosidad, disminuyendo así el esfuerzo generado por el fluido sobre las piezas que se están analizando al momento de ser mezclado. Una vez aclarado este punto, las propiedades utilizadas en el análisis se presentan en la Figura \ref{fig:PropiedadesFluido}.

\begin{figure}[H]
    \centering
    \includegraphics[width=0.8\linewidth]{ImagenesTT1/PropiedadesFluido.png}
    \caption{Propiedades del fluido}
    \label{fig:PropiedadesFluido}
\end{figure}

El resultado del análisis se muestra en la Figura \ref{fig:FlowMezclado}

\begin{figure}[H]
    \centering
    \includegraphics[width=0.55\linewidth]{ImagenesTT1/FlowMezclado.png}
    \caption{Presión y flujo del fluido}
    \label{fig:FlowMezclado}
\end{figure}

Se observa que la presión del fluido está entre unos 119,114.23 Pa - 85,765 Pa, cuando el motor funciona a unos 400 rpm y se cubre la tolva en su totalidad. Ahora, en la Figura \ref{fig:FlowMezcladoDetalle} se logra observar de mejor manera el flujo. Las flechas nos muestra que el fluido se moverá hacia arriba, después se moverá radialmente para cuando la turbina empuje más mezcla hacia arriba pase a descender y se repita el ciclo, de esta manera comprobamos que efectivamente se logrará un mezclado uniforme.

Ahora, pasaremos al análisis estático en el que se considerará el par de torsión, la gravedad y la presión del fluido que se importará del análisis de FlowSimulation. 

\begin{figure}[H]
    \centering
    \includegraphics[width=0.6\linewidth]{ImagenesTT1/FlowMezcladoDetalle.png}
    \caption{Detalle del flujo de mezclado}
    \label{fig:FlowMezcladoDetalle}
\end{figure}

Como se mencionó anteriormente algunas piezas estarán unidas mediante tornillos como se logra observar en la Figura \ref{fig:UniónAcopleEjeTurbina}

\begin{figure}[H]
  \centering
  \begin{minipage}{0.4\textwidth}
    \centering
    \includegraphics[width=\linewidth]{ImagenesTT1/DefinicionTornillosMezcla.png}
    \caption{Descripción de los tornillos del acople, eje y turbina}
    \label{fig:DefinicionTornillosMezcla}
  \end{minipage}%
  \begin{minipage}{0.4\textwidth}
    \centering
    \includegraphics[width=\linewidth]{ImagenesTT1/UniónAcopleEjeTurbina.png}
    \caption{Unión acople, eje y turbina}
    \label{fig:UniónAcopleEjeTurbina}
  \end{minipage}
\end{figure}

\newpage

Para la definición de los tornillos, utilizamos las ecuaciones requeridas en la validación del soporte del motor, es decir, la Ecuación \ref{eq:FT} y \ref{eq:MT} para calcular tanto la fuerza de tracción como el par de torsión para apretar tornillos, sólo que ahora se consideran los siguientes parámetros:

\begin{itemize}
    \item Paso de la rosca (\( P \)): 1 mm
    \item Límite elástico (\( E \)): 640 N/mm²
    \item Área de la sección resistente (\( A_s \)): 2.5 mm²
    \item Coeficiente de rozamiento (\( \mu \)): 0.2
    \item Diámetro medio de la rosca (\( d_2 \)): 1.6 mm
    \item Diámetro del núcleo de la rosca (\( d_3 \)): 1.3 mm
    \item Diámetro máximo de la superficie de rozamiento de la cabeza del tornillo (\( D_e \)): 2.2 mm
    \item Diámetro mínimo de la superficie de rozamiento de la cabeza del tornillo (\( D_i \)): 1.8 mm
\end{itemize}

Con estos datos, sustituyendo en las ecuaciones, obtenemos como resultado que la fuerza de tracción (\( F_T \)) es aproximadamente \textbf{666.67 N} y
el momento de torsión (\( M_T \)) es aproximadamente \textbf{0.366 N·m}.

Después de definir eso, se definen los tipos de sujeciones y así mismo las cargas externas. Dichas cargas se observan en la Figura \ref{fig:CargasMezclado}.

\begin{figure}[H]
    \centering
    \includegraphics[width=0.5\linewidth]{ImagenesTT1/CargasMezclado.png}
    \caption{Cargas externas}
    \label{fig:CargasMezclado}
\end{figure}

La carga de torsión que aparece es la que ejercerá el motor de mezclado. Y la presión y tensiones son aquellas que se importaron desde la simulación de flujo.

Finalmente, el resultado en el análisis de tensiones en las piezas se muestra en la Figura \ref{fig:TensionesMezclado}

\begin{figure}[H]
    \centering
    \includegraphics[width=0.5\linewidth]{ImagenesTT1/TensionesMezclado.png}
    \caption{Resultados de tensiones}
    \label{fig:TensionesMezclado}
\end{figure}

El punto de máxima tensión se puede observar a detalle en la Figura \ref{fig:MaximaTensionMezclado}. Este punto hace sentido pues donde habrá mayor concentración de esfuerzos es precisamente en el acople porque experimentará la torsión por parte del motor y además la torsión proveniente de la transmisión por su unión entre el eje y la turbina.

\begin{figure}[H]
    \centering
    \includegraphics[width=0.5\linewidth]{ImagenesTT1/MaximaTensionMezclado.png}
    \caption{Punto de máxima tensión}
    \label{fig:MaximaTensionMezclado}
\end{figure}

\subsection{Detalle módulo 2 ($M_2$)}

\subsubsection{Selección de calefactores}

El proyecto incluye una etapa crucial de precalentamiento que se realizará simultáneamente con el proceso de mezclado. Esta coordinación es esencial para garantizar que, una vez obtenida una mezcla homogénea de los componentes, la misma pueda ser manejada de manera eficaz en el sistema de extrusión. La finalidad es que durante este precalentamiento, la mezcla alcance una temperatura óptima de hasta 50°C, facilitando así su procesamiento.

Para implementar este proceso, se utilizará la misma tolva que opera en el sistema de mezclado. Esta tolva desempeña un papel doble, ya que además de mezclar, alojará los elementos calefactores PTC que se encargarán del calentamiento. Los elementos calefactores PTC, seleccionados por su eficiencia energética y su capacidad de autoregulación, serán activados en paralelo con el mezclado de la materia prima necesaria para la elaboración del filamento. Este método de calentamiento simultáneo no solo optimiza los tiempos y recursos, sino que también asegura un control más efectivo y seguro de la temperatura, evitando cualquier riesgo de sobrecalentamiento gracias a las propiedades inherentes de los elementos PTC.

Esta integración eficiente de procesos refleja un enfoque sistemático para mantener la mezcla a la temperatura adecuada durante su preparación, mejorando así la calidad y la seguridad del producto final.

Los calentadores PTC son un tipo de elemento calefactor que exhibe una propiedad de autorregulación única. A medida que aumenta la temperatura del calentador, también aumenta su resistencia eléctrica, provocando la correspondiente disminución en el consumo de energía. Este mecanismo de autorregulación evita el sobrecalentamiento y garantiza un uso eficiente de la energía.

Los calentadores PTC ofrecen una multitud de ventajas sobre los elementos calefactores tradicionales, lo que los convierte en una opción ideal para diversas aplicaciones:

\begin{itemize}
    \item \textbf{Control de temperatura autorregulable}: los calentadores PTC ajustan automáticamente su potencia de salida en función de la temperatura, lo que elimina el riesgo de sobrecalentamiento y garantiza un funcionamiento constante y seguro.
    \item \textbf{Eficiencia energética}: la naturaleza autorreguladora de los calentadores PTC genera importantes ahorros de energía en comparación con los calentadores tradicionales.
    \item \textbf{Características de seguridad}: Los calentadores PTC son intrínsecamente seguros debido a sus propiedades de autorregulación, lo que minimiza el riesgo de quemaduras o incendios.
    \item \textbf{Versatilidad}: los calentadores PTC se pueden personalizar para cumplir con los requisitos de aplicaciones específicas en términos de tamaño, forma, potencia de salida y rango de temperatura.
\end{itemize}

Los calentadores PTC se utilizan ampliamente en una amplia gama de aplicaciones, que incluyen:
\begin{itemize}
    \item \textbf{Aeroespacial}: los calentadores PTC se emplean en sistemas de deshielo de aeronaves, calefacción de cabina y control de temperatura del compartimiento del motor.
    \item \textbf{Dispositivos médicos}: los calentadores PTC se utilizan en dispositivos médicos como calentadores de sangre, baños con temperatura controlada e instrumentos quirúrgicos.
    \item \textbf{Automoción}: los calentadores PTC se utilizan en aplicaciones automotrices, como calentadores de asientos, desempañadores y gestión de la temperatura del compartimiento del motor.
    \item \textbf{Industrial}: los calentadores PTC se utilizan en procesos industriales como soldadura de plástico, embalaje y procesamiento de alimentos.
\end{itemize}

\begin{figure}[H]
    \centering
    \includegraphics[width=0.3\textwidth]{ImagenesTT1/CalefactorPTC.png}
    \caption{Calefactor PTC}
    \label{fig:CalefactorPTC}
\end{figure}

Para la aplicación que nos compete, utilizaremos elementos calefactores AC DC 12 V de 70 °C 20x15x5mm. 

\subsubsection{Rediseño de la tolva}
Se realizarán unas ranuras a la tolva, específicamente 4 de ellas para que alberguen los elementos calefactores y de esta manera garantizar la correcta distribución de la temperatura sobre la mezcla.

\begin{figure}[H]
    \centering
    \includegraphics[width=0.5\textwidth]{ImagenesTT1/RediseñoTolva.png}
    \caption{Ranuras de la tolva}
    \label{fig:RanurasTolva}
\end{figure}

\subsubsection{Diseño de compartimiento para calefactores}
Para sostener los calefactores, se implementará un sistema de tipo puerta que simulará el compartimiento de pilas en los aparatos eléctricos. Esto permitirá un acceso fácil a los calefactores cuando sea necesario reemplazarlos.

Ocuparemos una bisagra tipo americana de 10 mm (comercial) para implementar las puertas para los calefactores. La puerta se diseñará para cubrir la mayor parte de estos elementos, dejando una ranura para los cables del calefactor. Además, la puerta tendrá los agujeros necesarios tanto para su sujeción con la bisagra como para su fijación a la tolva por medio de tornillos.

\begin{figure}[H]
    \centering
    \includegraphics[width=0.45\linewidth]{ImagenesTT1/Bisagra.png}
    \caption{Bisagra tipo americana}
    \label{fig:BisagraPrecalentado}
\end{figure}

La puerta de los elementos calefactores se diseño en base a las medidas de la ranura que se realizó tomando en cuenta los datos de las dimensiones de los mismos elementos, así el croquis que se utilizó para dicho diseño se logra observar en la Figura \ref{fig:croquispuerta}.

\begin{figure}[H]
    \centering
    \includegraphics[width=0.5\linewidth]{ImagenesTT1/CroquisPuerta.png}
    \caption{Croquis de diseño de puerta}
    \label{fig:croquispuerta}
\end{figure}

Posteriormente, se realizaron los barrenos necesarios para los tornillos que unirán la puerta con la bisagra. 

\begin{figure}[H]
    \centering
    \includegraphics[width=0.35\linewidth]{ImagenesTT1/PuertaCalefactores.png}
    \caption{Puerta de los calefactores}
    \label{fig:puertacalefactores}
\end{figure}

\subsubsection{Ensamble final de tolva}
Finalmente, el ensamble final considerando la integración de las bisagras y la puerta sobre la tolva se observa en la Figura \ref{fig:EnsamblePrecalentado}.

\begin{figure}[H]
    \centering
    \includegraphics[width=0.5\linewidth]{ImagenesTT1/EnsamblePrecalentado.png}
    \caption{Ensamble Precalentado}
    \label{fig:EnsamblePrecalentado}
\end{figure}


Cabe mencionar que los agujeros utilizados tanto en la bisagra como en la puerta son de 2 mm de diámetro, por lo que se utilizarán tornillos M2. Para los tornillos que fijarán la bisagra a la tolva, se seleccionarán de 5 mm de longitud, esto debido a la geometría de la tolva que requiere asegurar un agarre firme. Los demás tornillos tendrán una longitud de 1 mm.

Además, se considerará que la puerta lleve un aislante térmico en el interior, donde estará en contacto con los elementos calefactores. Para ello, se ha seleccionado la fibra cerámica.

La fibra cerámica ha sido elegida debido a su capacidad excepcional para soportar temperaturas extremadamente altas, superiores a 1200°C, sin degradar sus propiedades aislantes. Compuesta principalmente por alúmina (Al₂O₃) y sílice (SiO₂), esta fibra ofrece una resistencia térmica sobresaliente, combinada con flexibilidad y estabilidad química. Estas características la convierten en un material ideal para su uso en entornos altamente corrosivos, proporcionando un aislamiento térmico y eléctrico efectivo. Por estas razones, la fibra cerámica es ampliamente utilizada en diversas aplicaciones industriales.

Por ejemplo, en el revestimiento de hornos y calderas industriales, la fibra cerámica proporciona una barrera eficiente contra el calor, lo que contribuye a mejorar la eficiencia energética y reducir los costos operativos. En la industria aeroespacial, su capacidad de protección térmica es crucial para manejar las extremas condiciones de temperatura durante el vuelo y la reentrada atmosférica. Además, en la fabricación de dispositivos electrónicos, su capacidad para aislar térmica y eléctricamente protege los componentes sensibles, asegurando su funcionamiento óptimo.

La fabricación de fibra cerámica mediante procesos como el sol-gel, la fusión y extracción, y el electrohilado permite una adaptación precisa del tamaño y la composición de las fibras a necesidades específicas. Esto facilita la producción de diversas formas, como mantas, paneles, papeles y hilos, que se ajustan específicamente a las aplicaciones industriales deseadas.

Específicamente, las mantas de fibra cerámica, que ofrecen baja conductividad térmica y buena absorción acústica, son ideales para el revestimiento de paredes de hornos industriales, sellados de alta temperatura y como aislantes térmicos en tuberías de alta temperatura. Estas propiedades hacen que la fibra cerámica sea invaluable en aplicaciones que requieren materiales con alta resistencia térmica y estabilidad química, manteniendo su relevancia en la vanguardia de la tecnología e industria.

\newpage

Por lo tanto, la elección de la fibra cerámica como aislante en nuestras aplicaciones se debe a su rendimiento superior en condiciones extremas, su versatilidad en la fabricación y su capacidad para adaptarse a diversos entornos industriales, lo que la convierte en una solución óptima para los desafíos de nuestro proyecto.


\begin{figure}[H]
    \centering
    \includegraphics[width=0.5\linewidth]{ImagenesTT1/fibraceramica.png}
    \caption{Fibra de cerámica}
    \label{fig:fibraceramica}
\end{figure}

\subsubsection{Unión de la etapa de mezclado, precalentado y extrusión}

Para garantizar que la mezcla en el proceso de homogeneización no se transfiera a la etapa de extrusión antes de completarse, es crucial implementar un actuador que funcione como una barrera efectiva entre ambas etapas. En este contexto, se ha elegido una electroválvula por sus numerosas ventajas y características específicas para esta aplicación.

\textbf{Justificación de la Elección de la Electroválvula}
\begin{itemize}
    \item \textbf{Control Preciso y Rápido}: Las electroválvulas proporcionan un control preciso y rápido del flujo, lo cual es esencial para detener la mezcla en el momento exacto hasta que se complete el proceso de homogeneización. La capacidad de respuesta en milisegundos de estas válvulas asegura que la transición entre las etapas de homogeneización y extrusión sea instantánea, minimizando el riesgo de que una mezcla incompleta entre en el proceso de extrusión.
    \item \textbf{Operación Automatizada}: Las electroválvulas permiten un control remoto mediante señales eléctricas, lo que facilita su integración en sistemas de control automatizado. Esta característica es especialmente útil en procesos industriales donde se requiere una coordinación precisa entre diferentes etapas del proceso.
    \item \textbf{Fiabilidad y Durabilidad}: Las electroválvulas son conocidas por su alta durabilidad y fiabilidad en condiciones operativas adversas, lo cual es crucial en entornos industriales. Estas válvulas pueden manejar presiones y temperaturas variables sin comprometer su rendimiento.
    \item \textbf{Compatibilidad con Diferentes Fluidos}: Disponibles en una variedad de materiales y conFiguraciones, las electroválvulas pueden ser utilizadas con diferentes tipos de fluidos y mezclas, asegurando la compatibilidad con los materiales específicos empleados en los procesos de homogeneización y extrusión.
    \item \textbf{Practicidad}: La practicidad que nos brinda la implementación de una electroválvula es mejor comparado con otras alternativas, dado que con esta simplemente es necesario realizar el roscado adecuado tanto a la salida de la tolva como a la entrada del cilindro. 
\end{itemize}

La elección de una electroválvula como barrera entre la etapa de homogeneización y la etapa de extrusión en nuestro proyecto se basa en lo anteriormente mencionado. Estas características aseguran que el proceso de mezclado se complete de manera óptima antes de proceder a la extrusión, lo cual garantiza tanto la calidad del producto final como la eficiencia del proceso.

Para nuestra aplicación específica, utilizaremos una electroválvula de la marca DrillPro de 1'' a 110V CA normalmente cerrada de dos posiciones, dos vías con sello NBR y rosca NPT, que soporta una temperatura de entre 0 - 90° C.

\begin{figure}[H]
    \centering
    \includegraphics[width=0.3\linewidth]{ImagenesTT1/Electrovalvula.png}
    \caption{Electroválvula a utilizar}
    \label{fig:electrovalvula}
\end{figure}

\subsection{Detalle módulo 3 ($M_3$)}

\subsubsection{Cálculos para la etapa de extrusión}

En una extrusora, la etapa de extrusión es crucial para determinar la calidad del producto final. Por ello, es esencial calcular con precisión las medidas y dimensiones del husillo extrusor, así como seleccionar un material de fabricación capaz de resistir las cargas aplicadas, la fricción, la corrosión y otros factores.

Para este propósito, nos guiamos por las metodologías y fórmulas ampliamente reconocidas de Savgorodny \cite{Savgorodny}. Este experto establece una relación longitud-diámetro de $(6-40):1$ para la transformación de plásticos utilizando husillos que varían en longitud de 9 a 580 mm. Dada la naturaleza y los requisitos de nuestro sistema, hemos elegido una relación de $20:1$. Siguiendo esta metodología, el paso del husillo se establecerá con un valor equivalente al diámetro.

Como medidas iniciales contamos entonces con una longitud del husillo de 20 in y un diámetro de 1 in.

\begin{equation*}
    \frac{L}{D}=\frac{20}{1}=20:1
\end{equation*}

Para el paso, el autor recomienda entre 0.8 y 1.2 del diámetro, pero siguiendo la convección lo seleccionamos igual que el diámetro

\begin{equation*}
    t=D=1\hspace{0.1cm}in
\end{equation*}

El espesor tiene un mayor rango de selección, aunque se sugiere un valor cercano al del diámetro

\begin{equation*}
    e=(0.06\div0.1)D=0.1D=0.1(1\hspace{0.1cm}in)=0.1\hspace{0.1cm}in
\end{equation*}

La profundidad del canal tanto para el área de alimentación como de dosificación siguen las siguientes fórmulas

\begin{equation*}
   h_{alim}=(0.12\div0.16)D=0.16D=0.16(1\hspace{0.1cm}in)=0.16\hspace{0.1cm}in
\end{equation*}
\begin{equation*}
   h_{dosif}=0.5\left[D-\sqrt{D^2-\frac{4h_{alim}}{i}(D-h_{alim})\right]
\end{equation*}
\begin{equation*}
   h_{dosif}=0.5\left[1-\sqrt{1^2-\frac{4(0.16)}{3}(1-0.16)\right]=0.047\hspace{0.1cm}in
\end{equation*}

La longitud de la zona de extrusión está entre $(0.4\div0.6)L$

\begin{equation*}
    L_{extr}=0.5L=0.5(20\hspace{0.1cm}in)=10\hspace{0.1cm}in
\end{equation*}

En cuánto a la holgura tenemos un rango de $(0.02\div0.03)D$. Mismo valor que nos ayudará para calcular el diámetro del cilindro que contiene al husillo. Tomando un valor medio tenemos:

\begin{equation*}
    \delta=0.025D=0.025(1\hspace{0.1cm}in)=0.025\hspace{0.1cm}in
\end{equation*}

\newpage

El ángulo del filete no varía

\begin{equation*}
    \theta=\tan^{-1}\left(\frac{1}{\pi}\right)=17.7°
\end{equation*}

Con los resultados obtenidos se diseña la pieza en SolidWorks\textregistered y se obtiene

\begin{figure}[H]
    \centering
    \includegraphics[width=\textwidth]{ImagenesTT1/Husillo_esquema.png}
    \caption{Esquema de husillo extrusor}
    \label{fig:Husillo_esquema}
\end{figure}

La selección del material para el husillo es clave y debe estar justificada, es por ello que se aplicará la metodología de Ashby \cite{Ashby} para la selección este. De tal manera, se tiene que considerar lo siguiente:
\begin{enumerate}
    \item Identificar el perfil de atributos deseados para la aplicación
    \item Comparar este perfil con los de los materiales de ingeniería para encontrar el mejor candidato
\end{enumerate}

En la imagen \ref{fig:Ashby_universo} se observa que partimos del 'universo de materiales' el cual contiene familias como cerámicos, metales, polímeros, entre otros. Dentro de las familias tenemos clases y a su vez subclases por lo que tendríamos un ejemplo como la familia de los metales, con la clase de una aleación de aluminio, la subclase de la serie 6000 y por último el miembro como el 6001. De este modo, llegamos a un perfil de atributos en el que incluimos las propiedades térmicas. mecánicas, eléctricas, ópticas, densidad, etcétera. La selección así, implica buscar la mejor correspondencia entre los perfiles de propiedad de los materiales en el universo y el perfil de propiedad requerido por el diseño.

\begin{figure}[H]
    \centering
    \includegraphics[width=0.8\textwidth]{ImagenesTT1/Ashby_universo.png}
    \caption{Clasificación para el proceso de selección de materiales según Ashby. Fuente: \cite{Ashby}}
    \label{fig:Ashby_universo}
\end{figure}

\vspace{-1cm}

A continuación, pasamos a la etapa de seguir una estrategia como nos muestra Ashby a partir del diagrama mostrado en la Figura \ref{fig:estrategias_ashby}.

\begin{figure}[H]
    \centering
    \includegraphics[scale=0.45]{ImagenesTT1/estrategias_ashby.png}
    \caption{Estrategias de selección de Ashby. Fuente: \cite{Ashby}}
    \label{fig:estrategias_ashby}
\end{figure}

Entonces para esta extrusora se definen los atributos más importantes que son:
\begin{itemize}
    \item Dureza
    \item Resistencia a altas temperaturas
    \item Resistencia a la cedencia
\end{itemize}

Siguiendo la estrategia, eliminamos materiales que no sean metálicos y no cuenten con una buena conductividad térmica. Dentro de las posibles opciones encontramos los aceros AISI de la seria 4000 y 300. Por facilidad de obtención se opta por la serie 300 y dentro de esta serie destacan por su excelente resistencia a la corrosión, gracias a sus altos niveles de cromo y níquel y su resistencia a altas temperaturas.

Buscando los más comunes de la serie 300 tenemos al 301, 304, 316 y 321, los cuales tienen ciertas similitudes entre ellos y es por eso que se realiza una tabla comparativa para observar sus diferencias.


\begin{table}[H]
    \centering
    \begin{tabularx}{\textwidth}{|X|X|X|X|X|}
        \hline
        Propiedad & AISI 301 & AISI 304 & AISI 316 & AISI 321\\
        \hline
        Resistencia a la corrosión & Menor que la del 304 y 316 & Buena & Excelente & Ligeramente mejor que la del 304\\
        \hline
        Aplicaciones comunes & Componentes automotrices, resortes, componentes estructurales & Utensilios de cocina, electrodomésticos, equipo de procesamiento de alimentos, equipo médico & Equipos marinos, químicos, farmacéuticos, equipos de procesamiento de alimentos en ambientes salinos & Componentes aeronáuticos, sistemas de escape, equipos de alta temperatura\\
        \hline
        Costo & Menor costo en comparación con 304 y 316 & Moderado & Más costoso debido al contenido de molibdeno & Similar al 304, ligeramente más alto debido al titanio\\
        \hline
    \end{tabularx}
\end{table}

\begin{table}[H]
    \centering
    \begin{tabularx}{\textwidth}{|X|X|X|X|X|}
        \hline        
        Temperatura de servicio & Hasta 870°C & Hasta 870°C & Hasta 870°C, pero con mejor rendimiento a temperaturas elevadas en presencia de cloruros & Hasta 870°C, con mejor resistencia a la oxidación a altas temperaturas\\
        \hline
        Soldabilidad & Buena, aunque se requiere precaución para evitar la formación de grietas & Muy buena & Muy buena & Muy buena, se usa cuando se necesita una buena resistencia a la soldadura en alta temperatura\\
        \hline
        Formabilidad & Excelente, se endurece rápidamente por deformación & Muy buena & Muy buena & Buena, similar al 304\\
        \hline
    \end{tabularx}
    \caption{Comparación de aceros inoxidables de la serie 300}
    \label{tab:aisi_300}
\end{table}

A partir de la tabla \ref{tab:aisi_300} vemos que el AISI 304 es un punto medio y con un costo moderado, por lo que es el seleccionado finalmente para la construcción del husillo.


\subsubsection{Diseño del cilindro}
En cuanto a los cálculos del cilindro en el cual se llevará a cabo la extrusión, debemos contemplar las dimensiones de nuestro husillo más la holgura recomendad para evitar colisiones entre ambos. De este modo, tenemos.

\begin{equation*}
    D_{cilindro}=D_{husillo}+\delta=1\hspace{0.1cm}in+0.025\hspace{0.1cm}in=1.025\hspace{0.1cm}in
\end{equation*}

Al estar sumamente cercanos nos aseguramos de que haya un buen cizallamiento y se aumente la
temperatura para obtener una mejor consistencia en el filamento final.

\begin{figure}[H]
    \centering
    \includegraphics[width=\textwidth]{ImagenesTT1/Cilindro_esquema.png}
    \caption{Esquema de cilindro para extrusión}
    \label{fig:Cilindro_esquema}
\end{figure}

Observamos que al final tenemos una sección de roscado para ensamblar la matriz con el diámetro del filamento deseado, este se verá más tarde. Asimismo, el diámetro de alimentación se escogió a partir de una medida coherente del diámetro del cilindro.

Igualmente se ocupa la metodología de Ashby para la selección del cilindro y en este caso tenemos el mismo resultado, el AISI 304 porque básicamente tenemos el mismo perfil de atributos que para el husillo extrusor.


\subsubsection{Diseño de la matriz de extrusión}

Esta sección final en la etapa de extrusión se le da la forma y medidas deseadas al plástico saliente y siguiendo con los objetivos planteados, este es de 1.5 mm a 1.8 mm, o bien en su correspondiente al sistema inglés es de 0.06 pulg. a 0.07 pulg.

Con el motivo de darle un fácil mantenimiento a la extrusora, se añadió un roscado entre el extremo del cilindro y el inicio de la matriz con el fin de poner y quitar fácilmente o en su defecto, cambiar la matriz a un diámetro distinto si así se desea.

\begin{figure}[H]
    \centering
    \includegraphics[width=0.35\textwidth]{ImagenesTT1/Matriz_esquema.png}
    \caption{Vista lateral de la boquilla}
    \label{fig:Matriz_esquema}
\end{figure}

\textbf{Plato romperador}

Dentro de la matriz se recomienda colocar lo que se conoce como plato rompedor que como bien nos mencionan Goff y Whelan en su libro \cite{goff2009dynisco}, este es un disco o placa que tiene una serie de orificios uniformes que están alineados en la dirección del flujo. El plato rompedor encaja en el extremo del cilindro y se usa para soportar uno o más finos pantallas metálicas. Estas pantallas eliminan la contaminación de la masa fundida y aumenta la presión dentro del sistema, mejorando así la mezcla.

El diseño básicamente consta de un cilindro agujereado como se muestra en la figura \ref{fig:plato_rompedor}

\begin{figure}[H]
    \centering
    \includegraphics[scale=0.25]{ImagenesTT1/plato_rompedor.png}
    \caption{Plato rompedor}
    \label{fig:plato_rompedor}
\end{figure}

\subsubsection{Cálculos para el motor de extrusión}
Al seleccionar el motor de la extrusora, es imperativo abordar varios factores fundamentales que influirán en el rendimiento y eficiencia del sistema completo. Inicialmente, se inclinaría hacia un motor de corriente continua (DC) debido a su equilibrio inherente entre costo, rendimiento y control. Este tipo de motor parece ser particularmente apropiado para el proyecto y en específico un motor a pasos que son dispositivos óptimos para aplicaciones que demandan un posicionamiento preciso a velocidades bajas y medias. Por otro lado, los servomotores presentan un control de posición y velocidad más preciso, junto con una respuesta rápida, atributos que podrían ser cruciales si se busca un alto grado de precisión y rendimiento.

La inercia total del sistema (compuesta por el motor, el husillo y la carga) es un aspecto crítico que no se puede pasar por alto en la selección del motor. Un motor con un alto torque de inercia (y, por ende, menor inercia) sería más adecuado. Es por ello que mediante las fórmulas de x a x se obtienen los valores de torque a inercia que debe tener el motor, pero previo a esto se incluyen los datos a sustituir.\\\\
Peso de la masa: $1\hspace{0.1cm}kg$\\
Peso del husillo: $4.23\hspace{0.1cm}lb=1.92\hspace{0.1cm}kg$
Diámetro del husillo: $1\hspace{0.1cm}pulg=0.0254\hspace{0.1cm}m$\\
Longitud del husillo: $23\hspace{0.1cm}pulg=0.5842\hspace{0.1cm}m$\\
Paso del husillo: $1\hspace{0.1cm}in/rev=39.37\hspace{0.1cm}rev/m$\\
Densidad de AISI 304: $8000\hspace{0.1cm}kg/m^3$\\
Reductor de engranes: $15:1$\\
Fricción: $0.58$

Primeramente se calcula la inercia.

\begin{equation*}
    J_{carga}=\frac{m}{0.9}\left(\frac{1}{2\pi P}\right)^2=\frac{1+1.92}{0.9}\left(\frac{1}{2\pi (39.37)}\right)^2=5.3021\times10^{-5}\hspace{0.1cm}[kg-m^2]
\end{equation*}
\begin{equation*}
    J_{husillo}=\frac{\pi L\rho r^4}{2}=\frac{\pi (23)(8000)(0.0127)^4}{2}=1.9098\times10^{-4}\hspace{0.1cm}[kg-m^2]
\end{equation*}
\begin{equation*}
    J_{motor}=J_{carga}+J_{husillo}=2.44\times10^{-4}\hspace{0.1cm}[kg-m^2]
\end{equation*}

\newpage

Posteriormente, se calcula el torque mínimo que debe llevar el motor.

\begin{equation*}
    F=F_{ext}+F_{friccion}+F_{peso}=0+\mu mg\cos{\theta}+mg
\end{equation*}

\begin{equation*}
    F=0.58(2.92)(\cos{0})(9.81)+2.92(9.81)=45.2594\hspace{0.1cm}[N]
\end{equation*}

\begin{equation*}
    T_{acel}=J\left(\frac{\Delta vel}{\Delta t}\right)\left(\frac{2\pi}{60}\right)=2.44\times10^{-4}\left(\frac{30}{3}\right)\left(\frac{2\pi}{60}\right)=2.5552\times10^{-4}\hspace{0.1cm}[N-m]
\end{equation*}

\begin{equation*}
    T_{resis}=\left(\frac{F}{2\pi P}\right)\div 3=0.061\hspace{0.1cm}[N-m]
\end{equation*}
\begin{equation*}
    T_{mov}=T_{acel}+T_{resis}=2.5552\times 10^{-4}+0.061=0.06124\hspace{0.1cm}[N-m]
\end{equation*}

Al ya tener ambos valores calculados, se busca un motor que tenga capacidades mayores a las calculadas y se encuentra que el motor a pasos NEMA 23 el cual tiene un torque de 1.8 N-m, suficiente para nuestras necesidades.

\begin{figure}[h]
    \centering
    \includegraphics[width=0.4\textwidth]{ImagenesTT1/nema23.png}
    \caption{Motor a pasos nema 23}
    \label{fig:nema23}
\end{figure}

Aunado a ello, de las ecuaciones se observa un factor de reducción de 3, esto debido a la transmisión de potencia a partir de un juego de engranes (Figura \ref{fig:Potencia_engranes}) para ganar torque sacrificando la velocidad y a su vez poder mover la mezcla en el proceso de extrusión.

\begin{figure}[H]
    \centering
    \includegraphics[scale=0.3]{ImagenesTT1/Potencia_engranes.png}
    \caption{Transmisión de potencia por engranajes}
    \label{fig:Potencia_engranes}
\end{figure}

\subsubsection{Elementos calefactores}
Según un artículo publicado por la Universidad Antonio Nariño \cite{moreno2011}, el rango óptimo de temperatura es de 160ºC a 250ºC para plásticos comunes implementados en impresoras 3D. Asimismo, nos menciona que la implementación del sistema de control de temperatura reduce el desperdicio de materia prima y prolonga el tiempo efectivo de trabajo de la máquina.

Es por ello que se busca un elemento calefactor capaz de llegar a dichas temperaturas y que cumpla con los requerimientos del sistema. En la actualidad existen una gran variedad de elementos calefactores como lo son por cartuchos, resistencias tipo banda y por alambres resistentes a altas temperaturas, por lo que se analizaran las propiedades de cada uno para la toma de decisión.

\begin{table}[H]
    \centering
    \begin{tabular}{|>{\centering\arraybackslash}m{4cm}|>{\centering\arraybackslash}m{4cm}|>{\centering\arraybackslash}m{4cm}|>{\centering\arraybackslash}m{4cm}|}
        \hline
         & Resistencias tipo cartucho & Resistencias tipo banda & Alambre de Nicromel\\
         \hline
        Rango de temperatura & 400-870ºC & 200-250ºC & 1400ºC\\
        Aplicación & Barrenos & Ajustable a un área & Flexible\\
        Desventaja & Requiere perforación & Limitado a la forma & Requiere aislante térmico\\
        Precio & \$480 (10cm) & \$495 (1in de diámetro) & \$400 (1m)\\
        \hline
    \end{tabular}
    \caption{Criterios de selección para el elemento calefactor}
    \label{tab:calefactores}
\end{table}

A partir de la tabla observamos que todos cumplen con el requisito de la temperatura, al igual los tres tienen precios aproximados el uno del otro. Por lo tanto, el parámetro por el cual fue el determinante fue la flexibilidad que ofrece el alambre de nicromel (fig. \ref{fig:alambre_de_nicromel}), ya que solo basta con enrollarlo al cilindro con separaciones uniformes para asegurar que se distribuya correctamente la temperatura. Ya que no cuenta con aislante, se optó por ocupar fibra de cerámica, misma que se tiene en la etapa de precalentado (fig. \ref{fig:fibraceramica}).

\begin{figure}[H]
    \centering
    \includegraphics[scale=0.4]{ImagenesTT1/alambre_de_nicromel.png}
    \caption{Alambre de nicromel}
    \label{fig:alambre_de_nicromel}
\end{figure}

\subsubsection{Estudios y análisis}

Primeramente, se estudiará el caso de deformación debido a su propio peso por gravedad tomando un extremo fijo (el extremo del motor) y con un estudio de gravedad veríamos lo que se muestra en la Figura \ref{fig:estudio_husillo_1}.

\begin{figure}[H]
    \centering
    \includegraphics[width=\textwidth]{ImagenesTT1/estudio_husillo_1.jpg}
    \caption{Estudio de gravedad del husillo extrusor}
    \label{fig:estudio_husillo_1}
\end{figure}

El estudio muestra un punto de esfuerzos máximo muy inferior al límite elástico, por lo que ni siquiera es necesario calcular el factor de seguridad en este caso. Asimismo, es importante observar las deformaciones para asegurar que no choque con la pared interna del cilindro.

\begin{figure}[H]
    \centering
    \includegraphics[width=\textwidth]{ImagenesTT1/estudio_husillo_2.jpg}
    \caption{Deformaciones del husillo}
    \label{fig:estudio_husillo_2}
\end{figure}

Tenemos como valor máximo 0.135 mm y si recordamos se considero un diámetro interno del cilindro de 1.025 pulgadas, esto es una separación de 0.0125 pulgadas entre ambos, convertimos a milímetros y eso es 0.3175 mm, suficiente para que no toque a pesar de la deformación.

\newpage

El peso del cilindro según marcado por SolidWorks\textregistered considerando el material AISI 304 es de 4.47 libras. Con el fin de obtener un mayor factor de seguridad aproximamos el peso a 5 libras que se distribuirán en 2 apoyos que tiene el cilindro y de este modo se tienen los resultados a partir de un análisis de esfuerzos que se muestran en la Figura \ref{fig:estudio_soporte_1}.

\begin{figure}[H]
    \centering
    \includegraphics[width=0.5\textwidth]{ImagenesTT1/estudio_soporte_1.jpg}
    \caption{Análisis de deformación por el peso del cilindro}
    \label{fig:estudio_soporte_1}
\end{figure}

\vspace{-1cm}

Se observa que el máximo desplazamiento sería de un valor muy inferior siquiera al de 1 mm por lo que con esto se asegura un buen soporte al cilindro.  Además, en la Figura \ref{fig:estudio_soporte_2} podemos ver que tampoco se supera el límite elástico.

\begin{figure}[H]
    \centering
    \includegraphics[width=0.5\textwidth]{ImagenesTT1/estudio_soporte_2.jpg}
    \caption{Análisis de esfuerzos en los soportes del cilindro}
    \label{fig:estudio_soporte_2}
\end{figure}

Como se mencionó anteriormente, el calentamiento del cilindro será a partir del alambre de nicrom el cual se enrollará a lo largo de este y el metal por su propiedad de transmisión de calor, calentará la mezcla interna. Para cuantificar la temperatura que proporcionará este método, se realiza un estudio de flujo de calor para ver las temperaturas distribuidas. El dato del flujo de calor se calcula de la siguiente manera.

\begin{equation}
    q=\frac{k\cdot (T_{ext}-T_{int})}{e}
\end{equation}

Donde\\
$k$ es la conductividad térmica $[W/m\cdot K]$\\
$T_{ext}$ es la temperatura en la parte exterior $[ºC]$\\
$T_{int}$ es la temperatura en la parte interior $[ºC]$\\
$e$ es el espesor $[m]$

\begin{equation*}
    q=\frac{16.2(250-25)}{0.00508}=717716.54\hspace{.1cm}W/m^2
\end{equation*}


\begin{figure}[H]
    \centering
    \includegraphics[width=0.8\textwidth]{ImagenesTT1/estudio_cilidnro_1.jpg}
    \caption{Transferencia de calor del exterior al interior}
    \label{fig:estudio_cilidnro_1}
\end{figure}

Observamos que el metal AISI 304 cuenta con una buena conducción de calor teniendo casi la misma temperatura en el exterior que en el interior. Esto lo que nos dice es que no debemos llevar la resistencia del alambre de nicrom a una temperatura de 250, si no una menor ya que la extrusión por si sola genera también calor al interior del cilindro. Asimismo, se entiende el porque es necesario recubrir con un aislante la superficie, principalmente por la seguridad e integridad de las personas.


\subsection{Detalle módulo 4 ($M_4$)}

\subsubsection{Modulo de Enfriamiento}

El enfriamiento instantáneo a la salida de la matriz es necesario en primer lugar como medida de seguridad al operador y en segundo lugar busca mejorar las propiedades mecánicas del filamento al enfriarse rápidamente y solidificarse para mantener su forma. Esto es crucial para evitar deformaciones o cambios en las dimensiones del producto.

El autor M. Beltrán \cite{beltran} nos menciona que el enfriamiento del material fundido produce su contracción, reduce el tamaño y aumenta su densidad y en cualquier caso, el método, velocidad y homogeneidad del enfriamiento condicionan la microestructura del producto.

Por tales motivos, se considera un sistema de tres ventiladores idénticos ensamblados a perfiles de aluminio comerciales como se observa en la imagen (\ref{fig:ventilacion}).

\begin{figure}[h]
    \centering
    \includegraphics[width=0.8\textwidth]{ImagenesTT1/enfriamiento.png}
    \caption{Propuesta de enfriamiento}
    \label{fig:ventilacion}
\end{figure}

Las consideraciones que se deben tener en los ventiladores (120mm porque son más comerciales) es un gran flujo de aire ya que los polímeros comerciales se empiezan a reblandecer a los 70ºC aproximadamente, por lo que hay que encontrar un valor para los ventiladores a partir de este dato. Además, es importante destacar que el enfriamiento ocupado debe ser más rápido ya que no se cuenta con un sistema de enrollamiento para enfriar por más tiempo y por distintas etapas del proceso.

Datos:\\
Temperatura inicial - $Ti=250ºC$\\
Temperatura ambiental - $Ti=25ºC$\\
Diámetro del ventilador - $D_v=120\hspace{5pt}mm$\\
Diámetro del filamento - $D_f=1.6\hspace{5pt}mm$\\
Longitud del filamento - $l=20\hspace{5pt}cm$

La fórmula para el cálculo de la energía disipada es:

\begin{equation}
    Q=m\cdot c_p\cdot (T_{inicial}-T_{final})\label{eq:q}
\end{equation}

Donde m es la masa del filamento y $c_p$ es la capacidad calorífica específica del filamento.

Para el cálculo de la masa, ocupamos SolidWorks\textregistered y aplicamos un material de ABS para obtener las propiedades físicas y nos arroja valores de masa igual a $0.04\hspace{5pt}g$, un volumen de $40.21\hspace{5pt}m^3$ y una capacidad calorífica específica de $1386\hspace{5pt}J/(kg\cdot K)$.

Sustituimos en la ecuación \eqref{eq:q} y consideramos

\begin{equation*}
    Q=0.00004(1386)(250-70)=9.9792\hspace{5pt}J
\end{equation*}

De tal modo, la transferencia de calor en un tiempo aproximado de 15s por el cual el filamento pasará por los ventiladores es de

\begin{equation*}
    \dot{Q}=\frac{Q}{t}=\frac{9.9792}{15}=0.66528\hspace{5pt}W
\end{equation*}

Para el coeficiente de transferencia de calor total tenemos lo siguiente

\begin{equation}
    h_{total}=\frac{Q}{A(T_{inicial}-T_{aire})}\label{eq:q}
\end{equation}

\begin{equation*}
    A=3\pi\left(\frac{0.12}{2}\right)^2=0.03393\hspace{5pt}m^2
\end{equation*}

Sustituimos en la ecuación \eqref{eq:q}

\begin{equation*}
    h_{total}=\frac{0.66528}{0.3393(250-25)}=0.008714\hspace{5pt}W/m^2K
\end{equation*}

De este modo, el flujo de aire necesario es de

\begin{equation}
    Q_{total}=3\cdot\frac{\dot{Q}}{h_{total}}
\end{equation}
\begin{equation*}
    Q_{total}=3\cdot\frac{0.66528}{0.008714}=229.03833\hspace{5pt}m^3/h=135\hspace{5pt}CFM
\end{equation*}

Si lo dividimos entre los tres ventiladores que tenemos nos queda que cada uno debe tener un flujo de aire mayor o igual a 45 CFM.

De la Figura \ref{fig:ventilacion} observamos que la ventilación no es directamente sobre el filamento, si no que este pasa por un canal agujereado que da el camino a seguir y a su vez permite la refrigeración del mismo.


\subsection{Detalle módulo 5 ($M_5$)}

\subsubsection{Módulo de Monitoreo y Control}

El módulo de monitoreo y control se encargará de visualizar y manipular los parámetros actuales de trabajo de la máquina, abarcando varios subsistemas esenciales. A continuación, se detallan las funciones específicas que debe realizar el microcontrolador en cada uno de estos módulos:

\subsubsection{Módulo de Mezclado}

\begin{itemize}
    \item \textbf{Control y visualización de la velocidad del motor para mezclado}: 
    \begin{itemize}
        \item El microcontrolador ajustará la velocidad del motor según las necesidades del proceso.
        \item Se implementará una interfaz de usuario para mostrar la velocidad actual en revoluciones por minuto (RPM).
    \end{itemize}
    \item \textbf{Control y visualización del tiempo de mezclado}:
    \begin{itemize}
        \item El microcontrolador establecerá y mantendrá el tiempo de mezclado necesario para cada lote.
        \item Se incluirá un temporizador visible que muestre el tiempo restante de mezclado.
    \end{itemize}
\end{itemize}

\subsubsection{Módulo de Extrusión}
\begin{itemize}
    \item \textbf{Control y visualización de la temperatura de extrusión}:
    \begin{itemize}
        \item El microcontrolador regulará la temperatura del sistema de extrusión para asegurar que se mantenga dentro de los parámetros deseados.
        \item Se implementará un sensor de temperatura que alimente datos en tiempo real al microcontrolador, los cuales serán mostrados en una pantalla.
    \end{itemize}
\end{itemize}

\subsubsection{Módulo de Moldeo y Enfriamiento}
\begin{itemize}
    \item \textbf{Control del giro de los ventiladores}:
    \begin{itemize}
        \item El microcontrolador gestionará el encendido, apagado y velocidad de los ventiladores para enfriar el material moldeado.
    \end{itemize}
\end{itemize}

\subsubsection{Interfaz de Usuario y Comunicación}

El objetivo de la interfaz del sistema es mostrar los estados operativos de la máquina en los diferentes subsistemas, tales como la velocidad de los motores, la potencia de los ventiladores de enfriamiento, el estado de los actuadores, el peso del carrete actual, entre otros. 

Para cada uno de estos módulos, se diseñará una interfaz gráfica de usuario que permitirá:

\begin{itemize}
    \item \textbf{Monitoreo en tiempo real} de los parámetros críticos como velocidad, tiempo y temperatura.
    \item \textbf{Control manual} para ajustar los parámetros según las necesidades del proceso.
\end{itemize}

El dispositivo seleccionado para la implementación en el sistema es una pantalla táctil TFT LCD de 3.2' 240x320 IC ILI9341, elegida por el laboratorio del CIIEMAD. Esta pantalla servirá tanto para el control, preparación como para el monitoreo de datos e información.

En cuanto a esta pantalla, aunque es táctil, nosotros no ocuparemos las funciones táctiles debido a la recomendación y uso posterior por parte de los empleados del CIIEMAD, que preferian se manejaran con un botón físico. 

En la Figura \ref{fig:ILI9341Ejemplo} se muestra el elemento seleccionado, en este caso, es representativo debido a que el tamaño de este tipo de pantallas varia. 

\begin{figure}[H]
    \centering
    \includegraphics[width=0.4\textwidth]{ImagenesTT2/ILI9341.jpg}
    \caption{Pantalla TFT LCD ILI9341}
    \label{fig:ILI9341Ejemplo}
\end{figure}

Tras un análisis exhaustivo de los costos de adquisición, se decidió utilizar la pantalla ILI9341 debido a su relación calidad-precio y su facilidad de integración. La elección se fundamenta en la capacidad de la pantalla para ofrecer una interfaz intuitiva y eficiente para el usuario, sin la necesidad de muchos componentes adicionales. 

A continuación, se detallan los principios de diseño seguidos para asegurar que el usuario tenga el control adecuado de este módulo:

\begin{itemize}
    \item \textbf{Interacción Flexible}: La interfaz está diseñada para adaptarse a las necesidades del usuario, permitiendo ajustes rápidos y precisos de los parámetros operativos.
    \item \textbf{Simplicidad y Claridad}: Se ocultarán los tecnicismos internos no necesarios para el usuario ocasional, proporcionando una experiencia de uso sencilla y directa.
    \item \textbf{Interacción Interrumpible y Reversible}: Las acciones del usuario podrán ser interrumpidas y revertidas fácilmente, asegurando que cualquier cambio realizado pueda ser deshecho sin complicaciones.
    \item \textbf{Eficiencia en la Interacción}: Los modos de interacción estarán definidos de manera que no obliguen al usuario a realizar acciones innecesarias o no deseadas, optimizando el flujo de trabajo.
\end{itemize}

Ahora que sabemos esto, en la Figura \ref{fig:Flujos} se muestra un diagrama de flujo en el que se detallan la operación de la interfaz así como las transiciones entre las pantallas y funciones que se explicaran más adelante. 


\begin{figure}[H]
    \centering
    \begin{subfigure}[b]{0.8\textwidth}
        \centering
        \includegraphics[width=\textwidth]{ImagenesTT1/Diagrama1.pdf}
        \caption{Capa 1 del diagrama de flujo}
        \label{fig:Flujo1}
    \end{subfigure}
    \vfill
    \begin{subfigure}[b]{0.8\textwidth}
        \centering
        \includegraphics[width=0.85\textwidth]{ImagenesTT1/Diagrama 2.pdf}
        \caption{Capa 2 del diagrama de flujo}
        \label{fig:Flujo2}
    \end{subfigure}
    \caption{Diagramas de flujo para la interacción del sistema}
    \label{fig:Flujos}
\end{figure}

\newpage
\subsubsection{Diseño preliminar y funcionalidad de la interfaz}

La interfaz para el sistema ha sido desarrollado para proporcionar una interacción intuitiva y eficiente con el usuario, esta fue desarrollada mediante el IDE de Arduino. De manera general, se despliega en una pantalla ILI9341 y consta de 2 pantallas principales: una pantalla de bienvenida, un menú principal de opciones. A continuación, se detalla el funcionamiento de cada una de estas secciones:

\begin{itemize}
    \item \textit{\textbf{Pantalla de bienvenida:}} Al iniciar el sistema, el usuario es recibido con una pantalla de bienvenida (Figura \ref{fig:Pantalla1}) que muestra el nombre del sistema, MyE System, junto con el mensaje "Hecho en UPIITA y CIIEMAD". Esta pantalla inicial aparece durante unos segundos para indicar el correcto arranque del sistema y proporcionar una introducción básica. Después de un breve tiempo, la interfaz redirige automáticamente al menú principal para que el usuario pueda comenzar a interactuar con las funciones del sistema.
\end{itemize}

\begin{figure}[h!]
    \centering
    \includegraphics[width=0.4\textwidth]{ImagenesTT2/Pantalla1.png}
    \caption{Ventana de Bienvenida al sistema.}
    \label{fig:Pantalla1}
\end{figure}

\begin{itemize}
    \item \textit{\textbf{Menú principal:}} El menú principal presenta una serie de opciones que el usuario puede seleccionar para acceder a diferentes funcionalidades y tareas del sistema (Figura \ref{fig:Pantalla2}). Las opciones disponibles son: Info, Preparar, Control, Memoria y Acerca de.
\end{itemize}

\begin{figure}[h!]
    \centering
    \includegraphics[width=0.4\textwidth]{ImagenesTT2/Pantalla2.png}
    \caption{Ventana principal de operación.}
    \label{fig:Pantalla2}
\end{figure}

\begin{itemize}
    \item \textit{\textbf{Pantalla de Información:}} Esta ventana proporciona un resumen en tiempo real del estado de los principales componentes del sistema, permitiendo al usuario supervisar su funcionamiento sin necesidad de instrumentos adicionales. (Figura \ref{fig:Pantalla3}). 

    Los datos presentados incluyen:
    \begin{itemize}
        \item \textbf{Motores (M1 y M2)}: Muestran el estado (ON/OFF), la velocidad en RPM (revoluciones por minuto) y la dirección de rotación (H para horario, AH para antihorario).
        \item \textbf{Ventilación}: Indica el porcentaje de potencia utilizada por el sistema de ventilación.
        \item \textbf{Calentadores}: Informa si los calentadores están activos (ON) o inactivos (OFF).
    \end{itemize}
    
\end{itemize}

\begin{figure}[h!]
    \centering
    \includegraphics[width=0.4\textwidth]{ImagenesTT2/Pantalla3.png}
    \caption{Ventana de información.}
    \label{fig:Pantalla3}
\end{figure}

\begin{itemize}
    \item \textit{\textbf{Pantalla de preparar:}} En esta pantalla, el usuario tiene varias opciones de configuración para los motores y ventilación (Figura \ref{fig:Pantalla4}). Esta pantalla contiene los siguientes elementos:

    \begin{itemize}
        \item \textbf{Regresar}: Permite al usuario volver a la pantalla anterior.
        \item \textbf{Motor extrusor}: Accede a las configuraciones del motor de extrusión.
        \item \textbf{Motor mezclado}: Accede a las configuraciones específicas del motor de mezclado.
        \item \textbf{Ventilación}: Permite ajustar las configuraciones del sistema de ventilación.
\end{itemize}

    Al seleccionar una opción, el usuario es dirigido a la pantalla correspondiente para configurar ese componente específico.
    
\end{itemize}

\begin{figure}[h!]
    \centering
    \includegraphics[width=0.4\textwidth]{ImagenesTT2/Pantalla4.png}
    \caption{Ventana de preparación.}
    \label{fig:Pantalla4}
\end{figure}

\newpage

\begin{itemize}
    \item \textit{\textbf{Pantalla de Configuración del Motor:}} En esta pantalla (Figura \ref{fig:Pantalla5}), el usuario puede cambiar tanto el sentido de rotación como la velocidad del motor según sea necesario para su operación. Contiene los siguientes elementos:

    \begin{itemize}
        \item \textbf{Regresar}: Vuelve a la pantalla de "Preparar".
        \item \textbf{Sentido}: Muestra el sentido de rotación actual del motor (Horario o Antihorario). Esto se indica en color verde para una mejor visibilidad.
        \item \textbf{Velocidad}: Muestra la velocidad actual del motor en Revoluciones Por Minuto (RPM), también en color verde.
    \end{itemize}
    
\end{itemize}

\begin{figure}[h!]
    \centering
    \includegraphics[width=0.4\textwidth]{ImagenesTT2/Pantalla5.png}
    \caption{Ventana para la configuración de los motores.}
    \label{fig:Pantalla5}
\end{figure}

\newpage

\begin{itemize}
    \item \textit{\textbf{Pantalla de Control:}} En esta pantalla (Figura \ref{fig:Pantalla6}), el usuario tiene opciones para iniciar manualmente varios componentes del sistema. Esta pantalla presenta los siguientes elementos:

    \begin{itemize}
        \item \textbf{Regresar}: Permite al usuario volver a la pantalla anterior.
        \item \textbf{Iniciar extrusor}: Activa el motor extrusor para comenzar el proceso de extrusión.
        \item \textbf{Iniciar mezclado}: Activa el motor de mezclado, comenzando así el proceso de mezcla.
        \item \textbf{Iniciar vent}: Activa el sistema de ventilación para regular la temperatura o expulsar aire.
        \item \textbf{Iniciar PTC}: Activa el sistema de calentamiento PTC, utilizado para el módulo de pre-calentamiento. 
    \end{itemize}

Cada una de estas opciones permite al usuario controlar de manera independiente cada componente del sistema, iniciándolos según sea necesario para el proceso.
    
\end{itemize}

\begin{figure}[h!]
    \centering
    \includegraphics[width=0.3\textwidth]{ImagenesTT2/Pantalla6.png}
    \caption{Ventana de control.}
    \label{fig:Pantalla6}
\end{figure}

\begin{itemize}
    \item \textit{\textbf{Pantalla de Memoria:}} Esta pantalla (Figura \ref{fig:Pantalla7}), se implementó con la finalidad de que el usuario no restablezca los valores de la máquina cada vez que se reinicie o apague el sistema. Por lo tanto, esta pantalla permite guardar los parámetros de operación en la memoria EEPROM del microcontrolador, con la finalidad de ahorrar tiempo si es que típicamente se trabaja con la misma mezcla para su extrusión. 
\end{itemize}

\begin{figure}[h!]
    \centering
    \includegraphics[width=0.3\textwidth]{ImagenesTT2/Pantalla7.png}
    \caption{Ventana de memoria.}
    \label{fig:Pantalla7}
\end{figure}

\begin{itemize}
    \item \textit{\textbf{Pantalla de Acerca de:}} Finalmente, esta pantalla (Figura \ref{fig:Pantalla8}), permite visualizar los datos de contacto de los colaboradores del proyecto, así como la versión del código que se esta cargando al sistema. 
\end{itemize}

\begin{figure}[h!]
    \centering
    \includegraphics[width=0.4\textwidth]{ImagenesTT2/Pantalla8.png}
    \caption{Ventana de memoria.}
    \label{fig:Pantalla8}
\end{figure}

\subsubsection{Microncontrolador}

\subsubsection{Análisis de Selección del Microcontrolador}

El microcontrolador es el componente clave en nuestro sistema, encargado de procesar la información de las entradas del usuario y diversos sensores, además de ejecutar los comandos necesarios para el control y monitoreo de los distintos subsistemas. Para su selección, consideramos los siguientes criterios:

\newpage

\begin{itemize}
    \item Alta velocidad de procesamiento para las diversas tareas.
    \item Capacidad para realizar múltiples tareas simultáneamente.
    \item Disponibilidad de puertos de comunicación serial.
    \item Al menos 20 pines GPIO.
    \item Costo.
\end{itemize}

La Tabla \ref{tabla:clasificacion_parametros} muestra la ponderación asignada a cada uno de estos criterios.

\begin{table}[H]
\centering
\begin{tabular}{|c|c|c|c|c|c|c|c|}
\hline
\textbf{Criterio} & \textbf{A} & \textbf{B} & \textbf{C} & \textbf{D} & \textbf{E} & \textbf{Total} & \textbf{Porcentaje (\%)} \\ \hline
Velocidad de procesamiento & x & 1 & 0 & 1 & 1 & 3 & 30\% \\ \hline
Multiprocesamiento & 0 & x & 1 & 1 & 1 & 3 & 30\% \\ \hline
Puertos de comunicación & 1 & 0 & x & 1 & 1 & 3 & 30\% \\ \hline
Pines & 0 & 0 & 0 & x & 0 & 0 & 0\% \\ \hline
Costo & 0 & 0 & 0 & 1 & x & 1 & 10\% \\ \hline
\end{tabular}
\caption{Clasificación de parámetros para selección de microcontrolador.}
\label{tabla:clasificacion_parametros}
\end{table}

Evaluamos diferentes microcontroladores utilizando los siguientes criterios:

\begin{itemize}
    \item \textbf{A:} Velocidad de procesamiento
    \item \textbf{B:} Multiprocesamiento
    \item \textbf{C:} Puertos de comunicación
    \item \textbf{D:} Pines GPIO
    \item \textbf{E:} Costo
\end{itemize}

La Tabla \ref{tabla:matriz_decision} presenta la matriz de decisión que compara varias opciones de microcontroladores disponibles en el mercado.

\newpage

\begin{table}[H]
\centering
\begin{tabular}{|c|c|c|c|c|c|c|}
\hline
\textbf{Criterio} & \textbf{A} & \textbf{B} & \textbf{C} & \textbf{D} & \textbf{E} & \textbf{Ponderación} \\ \hline
Peso & 30\% & 30\% & 30\% & 0\% & 10\% & \\ \hline
\rowcolor[gray]{0.9} ESP32 & 9 & 9 & 9 & 9 & 10 & 9.1 \\ \hline
Raspberry Pi Pico & 8 & 9 & 9 & 9 & 10 & 8.8 \\ \hline
Arduino MEGA & 6 & 6 & 10 & 10 & 7 & 7.7 \\ \hline
STM32 BluePill & 8 & 9 & 9 & 8 & 10 & 8.6 \\ \hline
\end{tabular}
\caption{Matriz de decisión para el microcontrolador.}
\label{tabla:matriz_decision}
\end{table}

De acuerdo con la matriz de decisión, seleccionamos el \textbf{ESP32} como el microcontrolador para nuestro sistema. Esta elección se basa en varios aspectos importantes:

\begin{itemize}
    \item \textbf{Requerimientos de procesamiento:} El ESP32 ofrece una alta velocidad de procesamiento y capacidad de multiprocesamiento, lo cual es ideal para manejar tareas en paralelo y mejorar la eficiencia del sistema.
    \item \textbf{Cantidad de pines GPIO:} El ESP32 proporciona una cantidad adecuada de pines GPIO que permite gestionar múltiples sensores y actuadores necesarios en el sistema.
    \item \textbf{Relación costo-beneficio:} A pesar de su amplio conjunto de características, el ESP32 tiene un costo accesible en comparación con otras opciones, lo que lo convierte en una solución rentable para proyectos avanzados.
\end{itemize}

La Figura \ref{fig:esp32} muestra el microcontrolador seleccionado:

\begin{figure}[H]
\centering
\includegraphics[width=0.4\textwidth]{ImagenesTT2/esp32.jpg}
\caption{Tarjeta de desarrollo ESP32.}
\label{fig:esp32}
\end{figure}

\subsubsection{Control de velocidad y desplazamiento de los motores}

Como hemos dicho, la velocidad y control de los motores, se verá controlada desde la interfaz que estará interconectada con nuestro microntrolador. Debido a ambos tipos de motores utilizados para cada etapa, tanto de mezclado como de extrusión, puede hacerse amplio uso de un sistema de lazo abierto, véase la Figura \ref{fig:LazoAbierto}.

\begin{figure}[H]
    \centering
    \includegraphics[width=0.9\textwidth]{ImagenesTT1/Lazo.png}
    \caption{Lazo abierto de control para los motores}
    \label{fig:LazoAbierto}
\end{figure}

Como sabemos, en un sistema de lazo abierto, el microcontrolador genera señales de control que se envían directamente a los actuadores (en este caso, los motores) sin recibir retroalimentación sobre el estado del sistema. Esto significa que el sistema no ajusta su salida en función de las variaciones o errores que puedan ocurrir durante el funcionamiento. En lugar de ello, confía en que las señales de control generadas sean suficientes para lograr el comportamiento deseado.

\subsubsection{Control de velocidad de los ventiladores}

\subsubsection{Envío de Pulsos PWM desde el Microcontrolador}

El microcontrolador, como el Arduino UNO, genera una señal PWM que consiste en una serie de pulsos con una frecuencia constante pero con un ancho variable. El ancho del pulso (duty cycle) determina la cantidad de energía entregada al motor. Por ejemplo, un duty cycle del 50\% significa que la señal está encendida la mitad del tiempo y apagada la otra mitad.

    \newpage
    \begin{enumerate}
    \item \textbf{Control del Motor de Mezclado con el Driver BTS 7960:}
    \begin{itemize}
        \item \textbf{Driver BTS 7960:} Este driver es adecuado para motores de alta potencia, como el motorreductor utilizado en el módulo de mezclado.
        \item \textbf{Señal PWM:} El microcontrolador envía la señal PWM al driver BTS 7960. El driver utiliza esta señal para controlar la velocidad del motorreductor. Al variar el duty cycle de la señal PWM, se puede ajustar la velocidad del motor. Un duty cycle mayor proporciona más potencia al motor, aumentando su velocidad, mientras que un duty cycle menor reduce la velocidad.
    \end{itemize}

    \item \textbf{Control del Motor de Extrusión con el Driver A4988:}
    \begin{itemize}
        \item \textbf{Driver A4988:} Este driver es utilizado para controlar el motor a pasos NEMA 23 en el módulo de extrusión.
        \item \textbf{Pulsos de Control:} El microcontrolador envía una serie de pulsos de control al driver A4988. Cada pulso hace que el motor avance un paso. La frecuencia de estos pulsos determina la velocidad del motor a pasos. Una frecuencia mayor de pulsos resulta en una velocidad de rotación más rápida.
    \end{itemize}
\end{enumerate}

Para ejemplificar esta parte, usaremos la simulación y comprobación mediante Proteus. Como hemos dicho anteriormente, tenemos nuestra interfaz creada, que esta conectada con el microcontrolador, y como condición tenemos que nuestro sistema podrá trabajar en rangos de:

\begin{itemize}
    \item Velocidad de mezclado: 200 - 400 rpm
    \item Tiempo de mezclado: 1 - 90 min
    \item Temperatura de extrusión: 170 °C - 210 °C
\end{itemize}

Una vez que tenemos cargado nuestro programa y corremos la interfaz, vemos lo siguiente. En la Figura \ref{fig:Config1} se muestran los valores que están siendo enviados desde la interfaz al microntrolador. En este caso tenemos una velocidad de 50 rpm, un tiempo de mezclado de 200 min y una temperatura de 180 °C, se puede ver una relación de estos valores versus un ancho de pulso mandando a los pines 9, 10 y 11 de la placa, reflejados en el osciloscopio que podemos ver en la Figura \ref{fig:Config2}.

\begin{figure}[H]
    \centering
    \includegraphics[width=0.7\textwidth]{ImagenesTT1/Config1.jpg}
    \caption{Valores conFigurados por la interfaz}
    \label{fig:Config1}
\end{figure}

\begin{figure}[H]
    \centering
    \includegraphics[width=0.8\textwidth]{ImagenesTT1/Config2.jpg}
    \caption{Simulación de los diferentes PWM entregados por el micro}
    \label{fig:Config2}
\end{figure}

Una vez entendido esto, también debemos explicar una pequeña funcionalidad que le dimos a nuestro sistema como protección de los valores ingresados. Como se pudo observar en la explicación de nuestra interfaz, ingresamos nuestros valores mediante una pantalla con un teclado númerico, este teclado permite ingresar cualquier valor sin restricción alguna, lo que podría ser contraproducente en nuestro sistema, por lo que decidimos establecer ciertos limites en nuestros valores mínimos y máximos en nuestros PWM. Esto debido a que aunque una persona ingrese un valor por debajo o por arriba de estos limites, la interfaz muestra el mínimo o máximo valor permitido y que esa sea la señal que mande a nuestro microntrolador. 

Por ejemplo, para nuestra velocidad de mezclado consideramos que 255 = 600 \, \text{rpm}

\[
\text{pwm min val} = \frac{200 \, \text{rpm} \times 255}{600 \, \text{rpm}} = 85
\]

\[
\text{pwm max val} = \frac{400 \, \text{rpm} \times 255}{600 \, \text{rpm}} = 170
\]

Esta funcionalidad se puede ejemplificar viendo los valores ingresados y mostrados para este parámetro. De manera representativa en la Figura \ref{fig:Config3}, vemos que se ingresaron valores de 0 y 450 rpm (diferentes casos), y en el recuadro blanco, vemos que aunque estas velocidades ingresadas están por abajo y por encima de nuestros limites, lo que en realidad esta mandando la interfaz al micro son 200 y 400 rpm respectivamente para cada caso. 

\begin{figure}[H]
    \centering
    \includegraphics[width=0.75\textwidth]{ImagenesTT1/Config3.jpg}
    \caption{Ejemplo de la protección para la señales mandadas por la interfaz}
    \label{fig:Config3}
\end{figure}

De igual forma, esto se implementará para las otras dos variables que tenemos, aunque ya no explicaremos los cálculos, se procedió de la misma forma. De manera que en la Figura \ref{fig:Config4} podemos observar la comparación de los valores mínimos permitidos VS los ingresados, y por otro lado en la Figura \ref{fig:Config5}, podemos observar la comparación de los valores máximos permitidos VS los ingresados

\begin{figure}[H]
    \centering
    \includegraphics[width=0.55\textwidth]{ImagenesTT1/Config4.jpg}
    \caption{Comparación valores mínimos permitidos VS los ingresados}
    \label{fig:Config4}
\end{figure}

\begin{figure}[H]
    \centering
    \includegraphics[width=0.55\textwidth]{ImagenesTT1/Config5.jpg.png}
    \caption{Comparación valores máximos permitidos VS los ingresados}
    \label{fig:Config5}
\end{figure}

\subsubsection{Control de temperatura}

El control de temperatura es una parte crucial en muchos sistemas mecatrónicos, especialmente en aplicaciones donde se requiere mantener condiciones térmicas, como en procesos de extrusión o en sistemas de calefacción. Para regular la temperatura proporcionada por la resistencia calefactora en sistemas de extrusión, se opta por implementar un control de potencia. 

Este proceso requiere conocer la temperatura actual para proporcionar retroalimentación, lo cual se logra mediante el uso de un termopar. Además, es necesario detectar el cruce por cero de la señal de corriente alterna (CA), desarrollar un algoritmo de control PID, y ajustar el ángulo de disparo en la puerta de un TRIAC.


\textbf{Detector de Cruce por Cero}

Un Detector de cruce por cero no es más que un dispositivo que detecta el momento exacto en que una señal de corriente alterna (CA) pasa por el nivel cero voltios. Esta información es crucial para controlar dispositivos como TRIACs, que se utilizan en la regulación de potencia en sistemas CA. 

En nuestro caso, el uso de un Detector de cruce por cero permite encender o apagar el TRIAC en el momento en que la corriente es mínima, reduciendo así el ruido eléctrico y mejorando la eficiencia del sistema. Este punto se puede observar en la Figura \ref{fig:Cruce}.

\begin{figure}[h!]
    \centering
    \includegraphics[width=0.4\textwidth]{ImagenesTT1/CrucePorCero.png}
    \caption{Momento de reconocimiento del cruce por cero en la señal AC}
    \label{fig:Cruce}
\end{figure}

\newpage

Sabiendo lo anterior, para que el microcontrolador pueda funcionar correctamente con señales de voltajes mayores a 5V y evitando voltajes negativos, se utiliza un transformador y un puente de diodos en conjunto con un optoacoplador. El puente de diodos se encarga de rectificar la señal, eliminando la componente negativa, mientras que el optoacoplador ajusta el voltaje a 5 $Vpp$, lo que permite al microcontrolador leer la señal. 

La Figura \ref{fig:zero_crossing} muestra el circuito de conexión para este propósito, empleando un optoacoplador 4N25, aunque también puede ser ocupado su equivalente PC817 que aísla las dos etapas del circuito.


\begin{figure}[h!]
    \centering
    \begin{subfigure}[b]{1\textwidth}
        \centering
        \includegraphics[width=0.9\textwidth]{ImagenesTT1/ZeroCrossing2.jpg}
    \end{subfigure}
    \vskip\baselineskip
    \begin{subfigure}[b]{0.8\textwidth}
        \centering
        \includegraphics[width=1\textwidth]{ImagenesTT1/ZeroCrossing1.jpg}
    \end{subfigure}
    \caption{Diagrama de conexión para la detección de cruce por cero.}
    \label{fig:zero_crossing}
\end{figure}

\subsubsection{Sistema de lectura de la temperatura}

Para medir la temperatura de extrusión, se seleccionó un termopar tipo K, el cual es adecuado para un rango de temperaturas que va desde -200°C hasta 1250°C. El termopar funciona generando una diferencia de potencial (voltaje) en respuesta a cambios de temperatura, gracias a la unión de dos metales diferentes. Sin embargo, la señal generada por el termopar es muy pequeña y necesita ser amplificada para que el microcontrolador pueda procesarla adecuadamente.

Para amplificar y acondicionar esta señal, se utiliza el módulo MAX6675. Este módulo no solo amplifica la señal analógica del termopar, sino que también la convierte en una señal digital que el microcontrolador puede leer fácilmente. El MAX6675 incluye un ADC (convertidor analógico-digital) que digitaliza la salida del termopar, además de proporcionar compensación de temperatura para obtener mediciones precisas. La conFiguración y conexión del módulo con el termopar se muestran en la Figura \ref{fig:thermocouple_circuit}.

\begin{figure}[h!]
    \centering
    \includegraphics[width=0.9\textwidth]{ImagenesTT1/TermoparK.jpg}
    \caption{Diagrama de conexión para la lectura de temperatura.}
    \label{fig:thermocouple_circuit}
\end{figure}

\newpage
\subsubsection{Sistema de control de temperatura}

En un sistema de control de temperatura convencional para el cañón extrusor, como hemos dicho, la temperatura es medida mediante un termopar, que genera una señal analógica proporcional a la temperatura detectada. Esta señal analógica es muy pequeña y necesita ser acondicionada y amplificada para poder ser procesada por un microcontrolador. Para ello, se utiliza un convertidor analógico-digital (ADC) que transforma la señal analógica en una señal digital. Esta señal digitalizada es entonces enviada al microcontrolador, donde se procesa para determinar la temperatura actual del cañón extrusor.

Para mantener la temperatura del cañón extrusor dentro del rango deseado, se implementa un controlador PID (Proporcional-Integral-Derivativo). Este tipo de controlador utiliza tres componentes principales para ajustar la salida del sistema y corregir cualquier desviación de la temperatura deseada:

\begin{itemize}
    \item \textbf{Acción Proporcional (P)}: Esta componente incrementa la velocidad de respuesta del sistema y reduce el error en régimen permanente. Sin embargo, puede hacer que el sistema se vuelva más inestable.
    \item \textbf{Acción Integral (I)}: La acción integral reduce el error en régimen permanente al acumular los errores pasados y ajustarlos en consecuencia. No obstante, también puede aumentar la inestabilidad del sistema y afectar ligeramente la velocidad de respuesta.
    \item \textbf{Acción Derivativa (D)}: La componente derivativa mejora la estabilidad del sistema al prever los futuros errores basándose en la tasa de cambio de la señal de error. Aunque mejora la estabilidad, puede reducir ligeramente la velocidad del sistema y no afecta el error en régimen permanente.
\end{itemize}

El controlador PID ajusta la señal que se envía a la resistencia eléctrica del cañón extrusor. Si la temperatura medida es diferente de la temperatura deseada, el controlador PID calcula la corrección necesaria y ajusta la potencia suministrada a la resistencia para alcanzar y mantener la temperatura establecida. La Figura \ref{fig:pid_control} muestra el esquema de un sistema de control en lazo cerrado con un controlador PID, donde la salida del sistema (temperatura) es retroalimentada para ajustar la entrada (potencia suministrada a la resistencia).

\begin{figure}[h!]
    \centering
    \includegraphics[width=0.9\textwidth]{ImagenesTT1/DiagramControl.png}
    \caption{Sistema de control en lazo cerrado con control PID. Fuente: \cite{control-pid}}
    \label{fig:pid_control}
\end{figure}

\newpage

Donde:

\begin{itemize}
    \item $Acondicionador :$ Circuito de Potencia
    \item $Sistema : $ Resistencia calefactora (eléctrica)
    \item $Sensor : $ Termopar tipo K
\end{itemize}

Para lograr una sintonización precisa del modelo, es esencial contar con la planta física, ya que esto permite mejorar notablemente la estimación de las constantes del controlador. De acuerdo con el estudio realizado por Rosales-Dávalos et al. \cite{rosales2020}, se estableció un modelo para una resistencia eléctrica de 200W a 127V, similar a la que se empleará en este sistema. En el mencionado estudio, se caracterizó la planta (resistencia eléctrica) aplicando una tensión de 120V durante un periodo de 60 minutos. 

Para nuestro caso, se caracteriza una resistencia calefactora suministrándole diferentes voltajes y observar la temperatura que entregaba, en donde se observaron los valores entregados de 65°C, 155°C, 190°C, 230°C, 275°C y el máximo 300°C, obteniendo valores parecidos a los mostrados por la Figura \ref{fig:Temp} obtenida en el estudio \cite{rosales2020}.

\begin{figure}[h!]
    \centering
    \includegraphics[width=0.9\textwidth]{ImagenesTT1/Temp.jpg}
    \caption{Curvas características de la planta ejemplo. Fuente: \cite{rosales2020}}
    \label{fig:Temp}
\end{figure}

\newpage

Ahora, como sabemos, una resistencia calefactora se puede aproximar a un diagrama de bloques como el de la Figura \ref{fig:Bloques}, el cual es efectivamente el producto de la aportación resistiva y capacitiva del calefactor y su retroalimentación con ganancia de 0, todo esto planteado en \cite{ogata2003}.

\begin{figure}[h!]
    \centering
    \includegraphics[width=0.9\textwidth]{ImagenesTT1/BloquesCalefactor.jpg}
    \caption{Diagrama a bloques de la resistencia calefactora}
    \label{fig:Bloques}
\end{figure}

Dicho esto, y viendo que es un sistema de primer orden al cual tenemos que aplicarle un PID, su respuesta en función del tiempo va a tener la siguiente forma:

\[
e^{-at} \quad \longrightarrow \quad \frac{1}{s + a}
\]

\[
e^{\frac{t}{\tau}} \quad \longrightarrow \quad \frac{1}{s + \frac{1}{\tau}} \quad = \quad \frac{1}{\tau s + 1}
\]

Por lo tanto, de manera general, la ecuación de nuestra planta que debe caracterizarse tiene la forma:

\[
\frac{k}{\tau s + 1} 
\]

Sabiendo esto, con ayuda de la función de transferencia de la planta y con el software Matlab, obtenemos la función de transferencia para nuestra resistencia calefactora con ayuda del siguiente código:

\begin{lstlisting}[language=Matlab]
maxResTemp = 300;
maxVoltage = 127;
stataticTime = 1250;

% Calculos
k = maxResTemp / maxVoltage;
T = stataticTime / 5;

% Definicion del sistema de transferencia
sys1 = tf(k, [T 1]);

% Respuesta al escalon
step(sys1);
\end{lstlisting}

Obteniendo el modelo siguiente:

\[
G(s) = \frac{k}{Ts + 1} = \frac{4.339}{227.9s + 1}
\]

\newpage

Al simular el modelo y ver su respuesta a lo largo del tiempo, podemos ver su gráfica en la Figura \ref{fig:step}, percatándonos que alcanza su establecimiento en aproximadamente los 1250 s, que es el tiempo aproximado para un establecimiento correcto que mencionan en el estudio \cite{rosales2020}.

\begin{figure}[h!]
    \centering
    \includegraphics[width=0.65\textwidth]{ImagenesTT1/StepGrafica.png}
    \caption{Respuesta al escalón de nuestro sistema.}
    \label{fig:step}
\end{figure}


Utilizando la función de transferencia de la planta y el software Matlab, es posible realizar la sintonización del controlador PID para determinar las constantes \(K_p\), \(K_i\) y \(K_d\) del controlador. Este proceso se lleva a cabo con la ayuda de la herramienta "PID Tuner" de Matlab. En la Figura \ref{fig:pid_tuner} se muestra el diagrama correspondiente a esta tarea.

\begin{figure}[h!]
    \centering
    \includegraphics[width=0.95\textwidth]{ImagenesTT1/DiagramaControlador.png}
    \caption{Diagrama para la sintonización del controlador PID con Matlab.}
    \label{fig:pid_tuner}
\end{figure}

\newpage
Con la ayuda de la herramienta PID Tuner, se obtuvieron las siguientes ganancias: \(K_p = 203\), \(K_i = 7.2\) y \(K_d = 1.04\). La respuesta de la planta con estas ganancias se muestra en la Figura \ref{fig:pid_response}.

\begin{figure}[h!]
    \centering
    \includegraphics[width=0.6\textwidth]{ImagenesTT1/PIDresp.png}
    \caption{Curva característica del controlador PID con las ganancias obtenidas.}
    \label{fig:pid_response}
\end{figure}

En la Figura \ref{fig:pid_response} se muestra la salida del sistema de control en lazo cerrado, donde el sobreimpulso (\(M_p\)) es del 3.14 \% y el tiempo de establecimiento (\(T_s\)) es de 624 segundos. 

\subsubsection{Circuito de control de disparo para el TRIAC}

Para el circuito de potencia que permite regular el disparo para la resistencia eléctrica, se utiliza un MOC3023. Este componente es un optoacoplador, que está diseñado para ser usado con un TRIAC en la interfaz de sistemas lógicos para equipos alimentados por líneas de 115/240 Vac. La función principal del optoacoplador es proporcionar aislamiento eléctrico entre el circuito de control de baja potencia y el circuito de alta potencia.

El MOC3023 contiene un diodo emisor de luz (LED) en su entrada y un detector que controla un TRIAC en su salida. Cuando el LED es activado por una señal de control, el detector activa el TRIAC. 

Esto permite que la señal de control de baja potencia dispare el TRIAC sin un contacto directo, proporcionando un aislamiento eléctrico seguro y reduciendo el riesgo de daños por altas tensiones.

Además, se utiliza un TRIAC Q7008 en el circuito, que es un dispositivo semiconductor que puede controlar la corriente alterna. El TRIAC tiene un voltaje de bloqueo (VDRM) de 600V, lo que lo hace adecuado para aplicaciones de control de potencia en líneas de alta tensión. El TRIAC permite el paso de la corriente en ambas direcciones cuando es disparado, lo que lo convierte en un componente esencial para controlar dispositivos como resistencias eléctricas.

\begin{figure}[h!]
    \centering
    \includegraphics[width=0.8\textwidth]{ImagenesTT1/ControlDisparo.jpg}
    \caption{Diagrama de conexión para regular el ángulo de disparo.}
    \label{fig:triac_control}
\end{figure}

En el diagrama de conexión mostrado en la Figura \ref{fig:triac_control}, se puede observar cómo se integran estos componentes. El microcontrolador envía una señal al MOC3023, que luego dispara el TRIAC Q7008. Esto permite el control preciso del ángulo de disparo del TRIAC, regulando así la cantidad de potencia entregada a la resistencia eléctrica. Este método de control es eficiente y permite ajustes precisos en la salida de potencia, esencial para mantener una temperatura constante en aplicaciones como la extrusión.

\subsubsection{Controlador PID REX-C100 salida SSR}


\subsubsection{Circuito de acondicionamiento}

Para asegurar el funcionamiento adecuado de ciertos componentes electrónicos, es esencial suministrar un voltaje de 5Vdc, con un consumo aproximado de 300mA. Para lograr esto, se emplea un regulador de voltaje LM7805. 

Este dispositivo es parte de un circuito diseñado para acondicionar la señal proveniente de la fuente conmutada mencionada que se mencionará en las próximas páginas, proporcionando así una salida estable de 5V.

El regulador LM7805 puede suministrar hasta 1.5 A de corriente en su salida, lo cual es suficiente para alimentar los componentes electrónicos necesarios en el sistema. Además, el circuito incorpora capacitores que ayudan a eliminar picos de energía y corrientes parásitas, garantizando una señal de salida limpia y estable. La Figura \ref{fig:acondicionamiento} muestra el diseño del circuito de acondicionamiento.

\begin{figure}[h!]
    \centering
    \includegraphics[width=0.8\textwidth]{ImagenesTT1/Acondicionador.jpg}
    \caption{Circuito de acondicionamiento de 24 Vdc a 5Vdc.}
    \label{fig:acondicionamiento}
\end{figure}

Estos componentes funcionan en conjunto para asegurar que el voltaje de salida se mantenga estable en 5V, sin importar las variaciones en la carga o en el voltaje de entrada.

\newpage

\subsubsection{Etapa de potencia para los motores de mezclado y extrusión}

Para los motores de mezclado y extrusión que se usarán en este proyecto, se implementan dos tipos de drivers: el BTS7960 para el motor de mezclado y el A4988 para el motor de extrusión. A continuación, se detalla la configuración y funcionamiento de cada uno de estos drivers.

\textbf{Driver BTS7960 para el Motor de Mezclado}

El driver BTS7960 es un controlador de puente H de alta corriente que es adecuado para motores de corriente continua de alta potencia. Este driver permite controlar la dirección y velocidad del motor mediante señales PWM. Es capaz de manejar corrientes de hasta 43A y voltajes de hasta 27V, lo cual lo hace ideal para aplicaciones de alta potencia como el motor de mezclado en nuestro proyecto. Este modulo se muestra en la Figura \ref{fig:FigBTS}

\begin{figure}[h!]
    \centering
    \includegraphics[width=0.5\textwidth]{ImagenesTT1/FigBTS.jpg}
    \caption{Driver BTS7920}
    \label{fig:FigBTS}
\end{figure}


El driver BTS7960 tiene dos canales que permiten el control de dos señales PWM independientes para controlar la dirección y velocidad del motor. A continuación, se presenta la configuración típica para este driver:

\newpage

\textbf{Pines del microcontrolador (baja corriente)}

\begin{itemize}
    \item \textbf{VCC}: Alimentación del módulo – 5V
    \item \textbf{GND}: Tierra
    \item \textbf{IS-R}: Señal de entrada para detectar alta corriente – Rotación directa
    \item \textbf{IS-L}: Señal de entrada para detectar alta corriente – Rotación inversa
    \item \textbf{EN-R}: Señal de salida para controlar la dirección del motor – Rotación directa
    \item \textbf{EN-L}: Señal de salida para controlar la dirección del motor – Rotación inversa
    \item \textbf{PWM-R}: Señal PWM para controlar la velocidad del motor – Rotación directa
    \item \textbf{PWM-L}: Señal PWM para controlar la velocidad del motor – Rotación inversa
\end{itemize}

\textbf{Pines del motor (alta corriente)}

\begin{itemize}
    \item \textbf{M+}: Positivo del motor
    \item \textbf{M-}: Negativo del motor
    \item \textbf{B+}: Positivo de la batería
    \item \textbf{B-}: Negativo de la batería
\end{itemize}

El diagrama de la Figura \ref{fig:bts7960} ilustra una conexión típica para el driver BTS7960.

\begin{figure}[h!]
    \centering
    \includegraphics[width=0.6\textwidth]{ImagenesTT1/BTS7960.jpg}
    \caption{Diagrama de conexión para el driver BTS7960.}
    \label{fig:bts7960}
\end{figure}

\newpage

\textbf{Driver A4988 para el Motor de Extrusión}

Para el control del motor de extrusión, se utiliza el driver A4988. Este módulo permite controlar motores paso a paso mediante la modulación por ancho de pulso (PWM). El driver A4988 es adecuado para motores bipolares y soporta microstepping de hasta 1/16, lo que permite un control preciso del movimiento del motor. Este modulo se muestra en la Figura \ref{fig:FigA4988}

\begin{figure}[h!]
    \centering
    \includegraphics[width=0.5\textwidth]{ImagenesTT1/FigA4988.jpg}
    \caption{Driver A4988}
    \label{fig:FigA4988}
\end{figure}

El driver A4988 puede proporcionar hasta 2A por bobina y opera motores con voltajes entre 8V y 35V. A continuación, se presenta la configuración típica para este driver:

\begin{itemize}
    \item \textbf{VDD y GND}: Conexiones de alimentación para el circuito de control del driver.
    \item \textbf{VMOT y GND}: Conexiones de alimentación para el motor.
    \item \textbf{STP}: Entrada de señal para el control de los pasos del motor.
    \item \textbf{DIR}: Entrada de señal para controlar la dirección del motor.
    \item \textbf{MS1, MS2, MS3}: Entradas para seleccionar el modo de microstepping.
    \item \textbf{EN}: Entrada para habilitar o deshabilitar el driver.
\end{itemize}

\newpage

El diagrama de la Figura \ref{fig:a4988} muestra una conexión típica para el driver A4988.

\begin{figure}[h!]
    \centering
    \includegraphics[width=0.7\textwidth]{ImagenesTT1/A4988.jpg}
    \caption{Arreglo típico para control de motores a paso con driver A4988.}
    \label{fig:a4988}
\end{figure}

\subsection{Detalle módulo 6 ($M_6$)}

\subsubsection{Módulo Eléctrico}

Para el desarrollo del circuito eléctrico, se ha decidido utilizar un circuito impreso de doble cara. Este diseño incluye todos los componentes necesarios para la etapa de potencia de los motores, el acondicionamiento de señale y la comunicación con la pantalla Nextion de 7 pulgadas. En la Figura \ref{fig:PCB}, se presenta el diseño detallado del circuito.

\begin{figure}[h!]
    \centering
    \includegraphics[width=\textwidth]{ImagenesTT1/PCB.jpg}
    \caption{Diseño del circuito impreso para controlar el sistema.}
    \label{fig:PCB}
\end{figure}

Por otro lado en la Figura \ref{fig:Esquematico}, podemos ver el diagrama esquemático realizado para las conexiones y simulaciones de todo el circuito electrónico.

\begin{figure}[h!]
    \centering
    \includegraphics[width=\textwidth]{ImagenesTT1/Circuitoooo.png}
    \caption{Conexión eléctrica para el circuito impreso.}
    \label{fig:Esquematico}
\end{figure}

\newpage
\subsubsection{Fuente de alimentación}

Para determinar la capacidad de la fuente de alimentación necesaria para el sistema, se han listado en la tabla \ref{tab:Fuente} los requerimientos eléctricos de cada componente que debe ser alimentado. Esta tabla proporciona detalles sobre el consumo de corriente y el voltaje de operación de cada elemento, lo cual es esencial para dimensionar adecuadamente la fuente de alimentación y garantizar un funcionamiento óptimo del sistema.

\begin{table}[h!]
\centering
\begin{tabular}{|p{2.5cm}|p{2.6cm}|p{1cm}|p{1cm}|p{1cm}|p{2cm}|p{2cm}|p{1.5cm}|}
\hline
\textbf{Ítem} & \textbf{Descripción} & \textbf{Cant} & \textbf{V [V]} & \textbf{I [A]} & \textbf{Corriente Total [A]} & \textbf{Corriente Mínima Fuente 120\% [A]} & \textbf{Voltaje de la Fuente [V]} \\ \hline
Arduino & UNO R3 & 1 & 5 & 0.6853 & 0.6853 & 0.82236 & 5 \\ \hline
Controlador de motor paso a paso & A4988 & 1 & 5 & 0.224 & 0.224 & 0.2688 & 5 \\ \hline
Driver H & BTS7960 & 1 & 5 & 0.003 & 0.003 & 0.0036 & 5 \\ \hline
Optoacoplador & 4N25 & 1 & 5 & 0.15 & 0.15 & 0.18 & 5 \\ \hline
Módulo de temperatura SPI & MAX6675 & 1 & 5 & 0.0015 & 0.0015 & 0.0018 & 5 \\ \hline
Pantalla táctil & Nextion & 1 & 5 & 0.5 & 0.5 & 0.6 & 5 \\ \hline
Ventilador & PIG5010X12H & 3 & 12 & 0.12 & 0.36 & 0.432 & 12 \\ \hline
Motor paso a paso & Nema 23 & 1 & 12 & 2.4 & 2.4 & 2.88 & 12 \\ \hline
Motor DC & F-4D60-24 & 1 & 24 & 4.5 & 4.5 & 5.4 & 24 \\ \hline
\multicolumn{5}{|c|}{\textbf{Corriente Total}} & \textbf{8.8238} & \textbf{10.58856} & \\ \hline
\end{tabular}
\caption{Requerimientos para la selección de la fuente de alimentación}
\label{tab:Fuente}
\end{table}

\newpage
\textbf{Fuente seleccionada}

Se ha optado por una fuente conmutada de 24V y 10A. Esta elección se basa en la corriente total requerida por el sistema, incrementada en un 20\% para asegurar un margen de seguridad. La corriente total mínima requerida por los componentes es de aproximadamente 8.82 A. Al aplicar el margen de seguridad del 20\%, se obtiene una corriente de 10.58 A. La fuente seleccionada proporciona hasta 10A, lo cual es adecuado para las necesidades del sistema.
\newpage
\textbf{Especificaciones de la fuente}
\begin{itemize}
    \item \textbf{Tensión de entrada:} 110-220V AC
    \item \textbf{Tensión de salida:} 24V DC
    \item \textbf{Corriente de salida:} 10A
    \item \textbf{Potencia de salida:} 240W
    \item \textbf{Peso:} 629g
    \item \textbf{Precio:} \$231
\end{itemize}

La elección de esta fuente no solo cumple con los requisitos técnicos, sino que también es económica y de fácil integración en el sistema.

\begin{table}[H]
\centering
\begin{tabular}{|>{\centering\arraybackslash}p{2cm}|>{\centering\arraybackslash}p{4cm}|>{\centering\arraybackslash}p{2cm}|>{\centering\arraybackslash}p{4cm}|>{\centering\arraybackslash}p{2cm}|}
\hline
\textbf{Tensión [V]} & \textbf{Corriente mínima [A]} & \textbf{Corriente 120\% [A]} & \textbf{Corriente comercial [A]} & \textbf{Potencia [W]} \\ \hline
24 & 8.82 & 10.58 & 10 & 240 \\ \hline
\end{tabular}
\caption{Elección de fuente de alimentación}
\end{table}

\section{Integración del sistema mecatrónico}

En la Figura \ref{fig:sistema_mecatronico} se muestran todos los módulos y piezas integradas para el sistema mecatrónico. En este deberá estar presente un operador para iniciar y detener el proceso, seleccionará los datos iniciales de la mezcla, podrá ver parámetros importantes en una pantalla, y por último guiará el filamento una vez que termine la extrusión para sus estudios correspondientes en el laboratorio del CIIEMAD.

\begin{figure}[H]
    \centering
    \includegraphics[width=\textwidth]{ImagenesTT1/VistaLateral.png}
    \caption{Vista lateral}
    \label{fig:VistaLateral}
\end{figure}

\begin{figure}[H]
    \centering
    \includegraphics[width=\textwidth]{ImagenesTT1/sistema_mecatronico.png}
    \caption{Integración de sistema mecatrónico. Vista isométrica}
    \label{fig:sistema_mecatronico}
\end{figure}

\begin{figure}[H]
    \centering
    \includegraphics[width=\textwidth]{ImagenesTT1/VistaAlzada.png}
    \caption{Vista alzada}
    \label{fig:VistaAlzada}
\end{figure}

\subsection{Diagrama de funcionamiento}

Para una integración efectiva del sistema, se ha definido un modo de operación claro, representado en el diagrama de la Figura \ref{fig:diagrama_funcionamiento}. Este diagrama detalla el proceso desde la preparación del pellet hasta la extracción del filamento finalizado

\begin{figure}[h]
    \centering
    \includegraphics[width=\textwidth]{ImagenesTT1/DiagramaSistema.jpg}
    \caption{Diagrama de funcionamiento del equipo}
    \label{fig:diagrama_funcionamiento}
\end{figure}

\textbf{Descripción del Proceso}

\begin{itemize}
    \item \textbf{Usuario:}
    \begin{itemize}
        \item \textit{Inicio y preparación:} El usuario comienza preparando la materia prima e ingresándola en la tolva.
    \end{itemize}
    \item \textbf{Interfaz de Usuario:}
    \begin{itemize}
        \item \textit{ConFiguración:} Se seleccionan la velocidad de mezclado, el tiempo y la temperatura de extrusión a través de la interfaz.
        \item \textit{Monitoreo:} Se muestra el tiempo restante y la temperatura de extrusión.
    \end{itemize}
    \item \textbf{Módulo de Mezclado:}
    \begin{itemize}
        \item \textit{Parámetros y operación:} Se establecen los parámetros de operación y se inicia el giro del motor.
        \item \textit{Contador:} El tiempo se cuenta hasta alcanzar el tiempo establecido.
    \end{itemize}
    \item \textbf{Módulo de Precalentado:}
    \begin{itemize}
        \item \textit{Calentamiento:} Se eleva la temperatura a 50 °C durante el mezclado.
        \item \textit{Activación:} El usuario presiona un botón para activar la electroválvula.
    \end{itemize}
    \item \textbf{Módulo de Extrusión:}
    \begin{itemize}
        \item \textit{Extrusión:} La temperatura se eleva y se monitorea mediante un sensor.
        \item \textit{Activación del husillo:} Al alcanzar la temperatura establecida, se gira el motor del husillo.
    \end{itemize}
    \item \textbf{Módulo de Enfriamiento:}
    \begin{itemize}
        \item \textit{Enfriamiento:} El filamento es guiado por ventiladores que se encienden a máxima potencia para finalizar el proceso.
    \end{itemize}
\end{itemize}

\newpage
\clearpage

\section{Implementación del Sistema}

\subsection{Modificaciones al diseño}
En un inicio, se desarrolló un diseño conceptual ideal basado en fórmulas y ecuaciones, el cual, en teoría, cumplía con los requerimientos planteados. No obstante, en la implementación práctica, fue necesario realizar diversos ajustes en todas las etapas del sistema, algunos de ellos de considerable importancia. Estos cambios respondieron principalmente a la necesidad de reducir costos y facilitar el proceso de manufactura. A continuación, se detallan las modificaciones realizadas en cada módulo del sistema.

\subsubsection{Modificaciones al módulo de mezclado}
En el módulo de mezclado, se realizó un cambio relevante en la geometría de la tolva, pasando de una forma cónica a una prismática. Aunque la geometría cónica fue considerada inicialmente por sus ventajas en el flujo de materiales, se identificaron complicaciones en la fabricación, especialmente en lo que respecta a las técnicas de soldadura. Dado que las tolvas cónicas presentan dificultades significativas para ser soldadas, se optó por rediseñar la tolva en forma prismática para facilitar su manufactura.

El diseño prismático permite el corte y plegado de las láminas de forma más sencilla y precisa, y posibilita el uso de soldadura en ángulos rectos, lo cual es mucho más accesible y seguro en comparación con la soldadura en superficies curvas. Esta configuración prismática no solo reduce la cantidad de operaciones de ensamblaje, sino que también minimiza el riesgo de defectos en las uniones soldadas, asegurando una estructura más confiable y robusta.

Aunque el cambio de una tolva cónica a una prismática implica una menor eficiencia en el flujo natural de materiales, esta decisión fue tomada para priorizar la facilidad de manufactura, la simplicidad en los procesos de soldadura y la reducción de costos asociados al ensamblaje. En consecuencia, la tolva prismática cumple con los requerimientos estructurales y funcionales del diseño, al tiempo que optimiza los recursos de producción.

Fotoooooooooo tolva

En la Figura X, se observa la integración de una estructura completa de soporte para la tolva, la cual incluye placas en forma de "L" dispuestas estratégicamente. Estas placas cumplen la función de unir la tolva a los soportes laterales, proporcionando estabilidad adicional y evitando que el peso de la tolva recaiga directamente sobre la electroválvula. Además, estas placas sirven como punto de anclaje para el soporte del motor que se ubicará en la parte superior de la tolva.

El diseño modular permite que la estructura sea desmontable, facilitando tanto el mantenimiento como el acceso a todos los elementos del sistema de mezclado. La inclusión de soportes verticales y refuerzos horizontales distribuye las cargas de manera uniforme, asegurando que no haya tensiones concentradas en componentes críticos, como la electroválvula. Este enfoque asegura la estabilidad del conjunto, proporcionando un sistema confiable y fácil de ensamblar para la aplicación final.

Fotoooo final de mezclado

\subsubsection{Modificaciones al módulo de precalentado}
En la etapa de precalentado, se incorporaron compartimientos en las caras internas y en la parte superior de la tolva. Este diseño permite mantener el volumen efectivo de la tolva prácticamente sin alteraciones.

La adición de estos compartimientos internos optimiza la transmisión de calor hacia la mezcla, ya que el calor se transfiere directamente a través de las paredes internas en contacto con el material. Así mismo al evitar el desbaste externo de los compartimientos planteados anteriormente, se elimina la barrera térmica adicional, logrando así una mayor eficiencia en la transferencia de calor y reduciendo el tiempo necesario para alcanzar la temperatura deseada en la mezcla. Este diseño no solo mejora el rendimiento térmico de la tolva, sino que también simplifica los procesos de fabricación y preserva la integridad estructural del equipo.

Fotoooooooooo interior tolva

\subsubsection{Modificaciones al módulo de extrusión}
En esta etapa, el cambio más significativo se realizó en la transmisión que conecta el motor con el husillo. Inicialmente, se contempló el uso de un tren de engranajes para esta función; sin embargo, esta opción presentaba mayores desafíos tanto en términos de mantenimiento como de fabricación. Por este motivo, se optó por reemplazar dicho sistema con un acoplador flexible de eje de ciruela de aluminio, adaptado a las medidas específicas del eje del motor y del eje del husillo (12 mm - 20 mm). Asimismo, se mantuvo la elección del motor Nema 23 propuesto desde el diseño inicial, aunque se incorporó un motorreductor con relación 15:1, dado que la velocidad de extrusión requerida es relativamente baja. (Figura x)

FOTOOOOOOOOOOOOO

Asimismo, los soportes también fueron objeto de modificaciones debido a los ajustes previamente mencionados. Actualmente, se utiliza un soporte específico para el motor y otro para el cañón extrusor. En el caso del cañón extrusor, se fabricó una brida con rosca interna para asegurar su sujeción. Adicionalmente, para reducir la carga sobre el motor y mejorar la distribución de peso, se incorporó un segundo soporte que contribuye a distribuir las cargas. (Figura x)

FOTOOOOOOOOOOOOOOOOOOOOOOO

\subsubsection{Modificaciones al módulo de enfriamiento}
En este módulo, se añadió un ventilador adicional con el propósito de prolongar el recorrido del filamento, permitiendo así un tiempo de enfriamiento mayor. Esto con el objetivo de reducir el riesgo de deformaciones al solidificarse más rápido.

FOTOOOOOOOOOOOOOOOOOO

\subsubsection{Modificaciones al módulo de monitoreo y control}


\subsection{Implementación del módulo de mezclado}

\subsubsection{Medición de rpm del motor de mezclado}
\subsubsection{Aquí las pruebas necesarias e implementaciones de este módulo}

\subsection{Implementación del módulo de extrusión}
El módulo físico de extrusión se muestra en la figura x.

FOTOOOOOOOOOOOOOOOOOOOOO

\subsection{Verificación del módulo de extrusión}
\subsubsection{Implementación de la variación de velocidad de rotación del husillo extrusor}
\textbf{Objetivo:} Verificar que la frecuencia generada por el microcontrolador tenga una variación de máximo 5\% respecto a lo calculado teóricamente.

\textbf{Descripción:} Se hace uso del módulo de monitoreo y control para configurar dos pruebas, una verificación para una velocidad de 8 rpm y otra para 20 rpm, con ayuda de un osciloscopio se comprueba que sea generada la frecuencia deseada con un ancho de pulso del 5\%.

\textbf{Resultados:} Los resultados deseados junto con los obtenidos y su respectivo error, se muestran en las figuras.

\subsubsection{Implementación de variación de temperatura en el cañón extrusor}
\textbf{Objetivo:} Verifica la diferencia de temperatura configurada en el controlador de temperatura REX-C100 y el interior del cañón extrusor sea mínima.

\textbf{Descripción:} Se hace uso del módulo de monitoreo y control para configurar.

\textbf{Resultados:} Los resultados deseados junto con los obtenidos y su respectivo error, se muestran en las figuras.

\subsection{Implementación del módulo de enfriamiento}
El módulo físico de enfriamiento se muestra en la figura x.

FOTOOOOOOOOOOOOOOOOOOOOO

\subsection{Verificación del módulo de enfriamiento}
\subsubsection{Implementación de la variación de velocidad de los ventiladores}
\textbf{Objetivo:} Verificar que los valores de la programación coincidan con los reales

\textbf{Descripción:} Se hace uso del .

\textbf{Resultados:} Los resultados deseados junto con los obtenidos y su respectivo error, se muestran en las figuras.

\subsection{Implementación del módulo de monitoreo y control}

\subsubsection{Aquí las pruebas necesarias e implementaciones de este módulo}
\subsubsection{Aquí las pruebas necesarias e implementaciones de este módulo}




\newpage
\clearpage

\section{Análisis de Resultados}

El presente análisis se enfoca en la validación y rendimiento del sistema diseñado para la obtención de filamentos de biopolímeros y polímeros sintéticos. A través de simulaciones y pruebas prácticas, se evaluaron distintos módulos y componentes críticos del sistema, asegurando que cada elemento cumple con las especificaciones técnicas requeridas para el proceso de extrusión.

\textbf{Módulo de mezclado}

El módulo de mezclado se diseñó para asegurar una homogeneidad adecuada de la mezcla de biopolímeros y polímeros sintéticos. Utilizando un motorreductor capaz de operar a velocidades entre 200 - 400 rpm durante 1 - 2.5 horas, para lograr una mezcla homogénea, sin grumos. Las validaciones realizadas con Solidworks® confirmaron que el diseño de la turbina, eje y acople proporcionan una mezcla uniforme utilizando el motorreductor seleccionado en la sección 10.3.4 según el estudio de Flow Simulation que se muestra en la Figura \ref{fig:FlowMezcladoDetalle}, así mismo quedó en evidencia que nuestras piezas en conjunto al someterse al par de torsión aplicado por el motorreductor soportarán las tensiones a las que estarán sometidas, y esto se puede ver a detalle en las Figuras \ref{fig:TensionesMezclado} y \ref{fig:MaximaTensionMezclado}.

\textbf{Módulo de precalentado}

El módulo de precalentado incluyó el rediseño de la tolva con ranuras para los calefactores PTC que elevan la temperatura de la mezcla a 50°C. Estos elementos calefactores fueron seleccionados por su eficiencia energética y capacidad de autorregulación, garantizando un calentamiento uniforme y seguro. Así mismo, se agregó un diseñó de una puerta con bisagra para el compartimiento de los elementos calefactores, dicha puerta cuenta con un aislante para protección del usuario.

\textbf{Módulo de extrusión}

El módulo de extrusión transforma la mezcla calentada en filamento. Se diseñó un husillo extrusor con base en cálculos detallados, asegurando un flujo continuo y eficiente del material . Las simulaciones en Solidworks® de la sección 10.5.6 mostraron que el husillo y el cañón extrusor mantienen una distribución de calor homogénea, crucial para la correcta fusión del material .

El análisis de esfuerzos realizado en los soportes del cilindro y el husillo extrusor confirmó que los materiales seleccionados pueden soportar las altas temperaturas y presiones del proceso sin deformaciones significativas. Este análisis es fundamental para asegurar la integridad estructural del módulo durante la operación.

\textbf{Módulo de Enfriamiento}

El sistema de enfriamiento por aire se diseñó con tres ventiladores posicionados estratégicamente. Se tomó como referencia la metodología de un autor especializado para determinar la configuración óptima del sistema de enfriamiento. El cálculo del flujo de aire necesario mostró que cada ventilador debe tener un flujo mayor o igual a 45 CFM para asegurar un enfriamiento efectivo del filamento.

El enfriamiento instantáneo a la salida de la matriz es esencial para mejorar las propiedades mecánicas del filamento al enfriarse rápidamente y solidificarse, evitando deformaciones o cambios en las dimensiones del producto. La configuración del sistema de enfriamiento se validó utilizando los criterios establecidos por el autor de referencia, asegurando que el filamento pasa por un canal agujereado que permite su correcta refrigeración.

\textbf{Módulo de monitoreo y control}

El módulo de monitoreo y control se encarga de visualizar y manipular los parámetros actuales de trabajo de la máquina. A través de una interfaz gráfica de usuario implementada en una pantalla táctil Nextion, se permite el monitoreo en tiempo real de parámetros críticos como velocidad, tiempo y temperatura, así como el control manual para ajustar estos parámetros según las necesidades del proceso.

El microcontrolador, elegido mediante un análisis exhaustivo, se encarga de procesar la información de las entradas del usuario y diversos sensores, ejecutando los comandos necesarios para el control y monitoreo de los distintos subsistemas. La elección del Arduino UNO se basó en su equilibrio entre costo, facilidad de programación y capacidad para manejar las conexiones necesarias con la pantalla Nextion. Además, se realizaron simulaciones de la implementación de la parte electrónica utilizando software especializado, validando el funcionamiento del sistema de control y monitoreo. Se implementó un controlador PID para mantener la temperatura constante en el módulo de extrusión, el cual, las pruebas mostraron que el controlador PID responde eficazmente a las variaciones de temperatura, manteniéndola dentro de los rangos establecidos.


\subsection{Estimación de costos}

El costo general del sistema se puede ver en la tabla \ref{Tab:CostoGeneral}, donde se muestra un resumen del presupuesto para cada modulo, aunque en las tablas mostradas anteriormente a esta, se desglosa cada subsistema y el precio de los componentes que lo integran.

\begin{longtable}{|c|l|c|c|}
\hline
\multicolumn{4}{|c|}{\textbf{Módulo de Mezclado}} \\ \hline
\textbf{\centering Cantidad} & \textbf{\centering Objeto} & \textbf{\centering Precio unitario} & \textbf{\centering Precio total} \\ \hline
1 & Motor Mezclado F-4D60-24 & \$1,495.17 & \$1,495.17 \\ \hline
4 & Elementos calefactores AC DC 12 V de 70 °C & \$40.00 & \$160.00 \\ \hline
1 & Electroválvula 1 in & \$1,200.00 & \$1,200.00 \\ \hline
1 & Placa de Acero Inoxidable AISI 304 de 1" & \$2,000.00 & \$2,000.00 \\ \hline
\multicolumn{3}{|r|}{\textbf{Total}} & \$4,855.17 \\ \hline
\caption{Desglose y costo total del subsistema de mezclado.}
\end{longtable}

\begin{longtable}{|c|l|c|c|}
\hline
\multicolumn{4}{|c|}{\textbf{Módulo monitoreo y Módulo eléctrico}} \\ \hline
\textbf{\centering Cantidad} & \textbf{\centering Objeto} & \textbf{\centering Precio unitario} & \textbf{\centering Precio total} \\ \hline
1 & Arduino UNO o Arduino MEGA & \$200.00 & \$200.00 \\ \hline
1 & Controlador BTS7960 & \$180.00 & \$180.00 \\ \hline
3 & Controlador A4988 & \$40.00 & \$120.00 \\ \hline
1 & Hilo de Nicromo & \$80.00 & \$80.00 \\ \hline
1 & Fuente conmutada 24V 10A & \$231.00 & \$231.00 \\ \hline
3 & Optoacoplador 4N25 & \$10.00 & \$30.00 \\ \hline
1 & Módulo Max6675 & \$100.00 & \$100.00 \\ \hline
1 & Pantalla Nextion 7" & \$2,400.00 & \$2,400.00 \\ \hline
1 & Otros componentes electrónicos y conectores & \$300.00 & \$300.00 \\ \hline
1 & Piezas de soporte & \$500.00 & \$500.00 \\ \hline
1 & Pieza para base de la fuente & \$500.00 & \$500.00 \\ \hline
1 & Cable & \$100.00 & \$100.00 \\ \hline
1 & Tornillería & \$100.00 & \$100.00 \\ \hline
\multicolumn{3}{|r|}{\textbf{Total}} & \$4,641.00 \\ \hline
\caption{Desglose y costo total del subsistema de monitoreo/control y eléctrico.}
\end{longtable}

\newpage

\begin{longtable}{|c|l|c|c|}
\hline
\multicolumn{4}{|c|}{\textbf{Módulo de Extrusión}} \\ \hline
\textbf{\centering Cantidad} & \textbf{\centering Objeto} & \textbf{\centering Precio unitario} & \textbf{\centering Precio total} \\ \hline
1 & Nema 23 & \$600.00 & \$600.00 \\ \hline
1 & Base Nema 23 & \$190.00 & \$190.00 \\ \hline
1 & Rodamiento 0.85" & \$250.00 & \$250.00 \\ \hline
1 & Base para el rodamiento & \$590.00 & \$590.00 \\ \hline
1 & Barra redonda AISI 304 de 1.5" x 1m de largo & \$2,500.00 & \$2,500.00 \\ \hline
1 & Engranaje & \$130.00 & \$130.00 \\ \hline
1 & Piñón & \$110.00 & \$110.00 \\ \hline
2 & Soportes para el cilindro & \$650.00 & \$1,300.00 \\ \hline
\multicolumn{3}{|r|}{\textbf{Total}} & \$4,370.00 \\ \hline
\caption{Desglose y costo total del subsistema de extrusión.}
\end{longtable}

\begin{longtable}{|c|l|c|c|}
\hline
\multicolumn{4}{|c|}{\textbf{Módulo de Enfriamiento}} \\ \hline
\textbf{\centering Cantidad} & \textbf{\centering Objeto} & \textbf{\centering Precio unitario} & \textbf{\centering Precio total} \\ \hline
1 & Canal de enfriamiento & \$60.00 & \$60.00 \\ \hline
4 & Base para la altura & \$110.00 & \$440.00 \\ \hline
6 & Carcasa para el ventilador & \$190.00 & \$1,140.00 \\ \hline
2 & Base para el canal & \$155.00 & \$310.00 \\ \hline
3 & Ventiladores PIG5010X12H & \$120.00 & \$360.00 \\ \hline
1 & Perfil de aluminio 20x20 1 m & \$150.00 & \$150.00 \\ \hline
1 & Tornillería & \$100.00 & \$100.00 \\ \hline
\multicolumn{3}{|r|}{\textbf{Total}} & \$2,460.00 \\ \hline
\caption{Desglose y costo total del subsistema de enfriamiento.}
\end{longtable}


\begin{longtable}{|c|l|c|c|}
\hline
\multicolumn{4}{|c|}{\textbf{Módulo de Embobinado}} \\ \hline
\textbf{\centering Cantidad} & \textbf{\centering Objeto} & \textbf{\centering Precio unitario} & \textbf{\centering Precio total} \\ \hline
1 & Base de sujeción guías & \$120.00 & \$120.00 \\ \hline
1 & Base de sujeción balero & \$95.00 & \$95.00 \\ \hline
2 & Varilla lisa 8mm 0.3m & \$73.00 & \$146.00 \\ \hline
1 & Motor Nema 17 & \$200.00 & \$200.00 \\ \hline
1 & Polea dentada GT2 T20 W6 & \$21.00 & \$21.00 \\ \hline
1 & Base sujeción para motor & \$98.00 & \$98.00 \\ \hline
4 & Soporte lineal SK8 & \$30.00 & \$120.00 \\ \hline
1 & Correa dentada GT2 6mm 1m & \$40.00 & \$40.00 \\ \hline
2 & Plataforma balero lineal SCS8UU & \$98.00 & \$196.00 \\ \hline
2 & Rueda Derlin Dual V & \$31.00 & \$62.00 \\ \hline
1 & Base de sujeción para polea & \$85.00 & \$85.00 \\ \hline
1 & Polea aluminio GT2 & \$50.00 & \$50.00 \\ \hline
1 & Perfil Aluminio 25cm & \$79.00 & \$79.00 \\ \hline
1 & Motor Nema 17 c/ caja de engranes & \$825.00 & \$825.00 \\ \hline
4 & Ángulo de ajuste para perfil 2020 & \$11.00 & \$44.00 \\ \hline
1 & Cople D18L25 & \$35.00 & \$35.00 \\ \hline
1 & Celda de carga 5 Kg & \$72.00 & \$72.00 \\ \hline
1 & Varilla Roscada 8mm 0.2m & \$30.00 & \$30.00 \\ \hline
1 & Pieza de sujeción perfil aluminio & \$65.00 & \$65.00 \\ \hline
1 & Pieza sujeción motorreductor & \$70.00 & \$70.00 \\ \hline
\multicolumn{3}{|r|}{\textbf{Total}} & \$2,458.00 \\ \hline
\caption{Desglose y costo total del subsistema de embobinado}
\end{longtable}

\textbf{Resumen de Costos}

\begin{table}[h!]
\centering
\begin{tabular}{| m{6cm} | m{4cm} |}
\hline
\textbf{Módulo} & \textbf{Total} \\ \hline
Módulo de Mezclado & \$4,855.17 \\ \hline
Módulo Monitoreo y Módulo Eléctrico & \$4,641.00 \\ \hline
Módulo de Extrusión & \$4,370.00 \\ \hline
Módulo de Enfriamiento & \$2,460.00 \\ \hline
Módulo de Embobinado & \$2,458.00 \\ \hline
\textbf{Gran Total} & \textbf{\$18,784.17} \\ \hline
\end{tabular}
\caption{Resumen de costos de los diferentes módulos.}
\label{Tab:CostoGeneral}
\end{table}

\subsection{Análisis de valor}

\newpage
\section{Conclusiones}

Hablando en referente al objetivo principal de este proyecto el cual fue "Diseñar y construir un sistema semiautomático para mezclar biopolímeros y polímeros sintéticos y extruirlos para producir filamentos", este trabajo ha cumplido con la primera parte del objetivo, logrando el diseño y construcción de un sistema capaz de realizar la mezcla y extrusión de estos materiales, y validando la integración de cada módulo que conforma el sistema.

Para alcanzar el objetivo general, se dividió en objetivos específicos que facilitaron el control y organización del progreso del proyecto. Estos objetivos específicos permitieron definir con mayor precisión el alcance del sistema a desarrollar y la complejidad de su construcción.

La parte mas complicada fue seleccionar los materiales adecuados para la estructura del sistema, en este parte se realizó una exhaustiva investigación y evaluación de diversos materiales para la base del sistema mecatrónico, cuya función principal es soportar los módulos del sistema. Dada la magnitud de la base, fue un desafío encontrar materiales económicos que cumplieran con los requisitos necesarios.

Aunque el presupuesto inicial del proyecto puede parecer alto, el ahorro a largo plazo justifica la inversión, ya que se reduce la necesidad de comprar bobinas de filamento comercial. Asimismo, el diseño se pensó para realizar un mantenimiento fácil de los componentes mecánicos y electrónicos para asegurar su durabilidad y funcionamiento óptimo.

\section{Recomendaciones y trabajo a futuro}

\subsection{Modulo de embobinado}
\subsection{Estructura del sistema}


\newpage
\section{Apéndices}

\subsection{Apéndice 1 - Infraestructura}

En la tabla \ref{tabla:RH} se enumeran los recursos humanos disponibles para el desarrollo del proyecto, detallando el nombre, rol (asesor o estudiante), institución de origen y el tiempo destinado a las actividades del proyecto. 

\begin{table}[!h]
\centering
\small
\renewcommand{\arraystretch}{1.5}
\setlength{\tabcolsep}{10pt}
\begin{tabular}{|p{7cm}|p{1.5cm}|p{2cm}|p{3.5cm}|}
\hline
\rowcolor{gray!50}
\textbf{Recursos Humanos} & \textbf{STEM} & \textbf{Institución} & \textbf{Tiempo destinado} \\
\hline
{Barrera Ramírez Fernando Manuel (E)} & STEM & UPIITA & 600 horas \\
\hline
{Melchor Solis Marco Antonio (E)} & STEM & UPIITA & 600 horas \\
\hline
{Rivera Zúñiga Enrique Maximiliano (E)} & STEM & UPIITA & 600 horas \\
\hline
Gutierrez Begovich David Arturo (A) & STE & UPIITA  & 60 horas \\
\hline
García Serrano Luz Arcelia (A) & STE & CIIEMAD & 60 horas \\
\hline
López Alarcón Erick (A) & STE & UPIITA  & 60 horas \\
\hline
\end{tabular}
\caption{Recursos Humanos disponibles para el desarrollo del proyecto. (A: Asesor; E: Estudiante)}
\label{tabla:RH}
\end{table}

Además, en la tabla \ref{tabla:Infra} se describe la infraestructura necesaria para el desarrollo del proyecto, especificando los recursos físicos y tecnológicos disponibles, su descripción, la cantidad necesaria y el uso previsto. 

\begin{table}[!h]
\centering
\small
\renewcommand{\arraystretch}{1.5}
\setlength{\tabcolsep}{10pt}
\begin{tabular}{|p{3.5cm}|p{5cm}|p{1cm}|p{4cm}|}
\hline
\rowcolor{gray!50}
\textbf{Infraestructura} & \textbf{Descripción} & \textbf{Cant.} & \textbf{Uso} \\
\hline
Taller de máquinas y herramientas & Torno, fresadora, máquina CNC, herramientas de corte y doblado, estación de soldadura, herramientas de medición & 1 & Se manufacturará la estructura, tornillo, matriz de extrusora, etc. \\
\hline
Laboratorio de electrónica & Osciloscopio, generador de funciones, fuentes de alimentación, multímetro, cautín & 1 & Se harán los experimentos y se probarán los circuitos eléctricos \\
\hline
\end{tabular}
\caption{Infraestructura para el desarrollo del proyecto}
\label{tabla:Infra}
\end{table}

\newpage
\subsection{Apéndice 2 - Módulo de embobinado}

Complementando el trabajo que se va a realizar según nuestros objetivos, se pretende agregar un mecanismo para que el filamento extruido sea embobinado en carretes comerciales. Esto con el fin de que si es posible alcanzar a desarrollar esta etapa, el filamento generado por nuestro sistema pueda ser implementado de manera inmediata a la aplicación de impresiones 3D. Cabe mencionar que se recuperó el sistema de embobinado que han realizado en otros proyectos, por lo que el diseño no es de propia autoría pero si las modificaciones pertinentes para su implementación en nuestro sistema.

El embobinado del filamento se lleva a cabo a través de dos acciones mecánicas fundamentales: rotar y trasladar el material. La Figura \ref{fig:MovimientosCarrete} muestra gráficamente estos movimientos esenciales en los carretes.

\begin{figure}[H]
    \centering
    \includegraphics[width=0.6\linewidth]{ImagenesTT1/MovimientosCarrete.png}
    \caption{Movimientos para embobinar filamento en carretes \cite{extrusoraPLA}}
    \label{fig:MovimientosCarrete}
\end{figure}

\textbf{Mecanismo de movimiento lineal}

Para generar el movimiento lineal se implementará una transmisión a través de correa dentada. En la Figura \ref{fig:ComponentesML} se muestran los componentes del mecanismo a utilizar.

\begin{figure}[H]
    \centering
    \includegraphics[width=0.65\linewidth]{ImagenesTT1/ComponentesML.png}
    \caption{Componentes del mecanismo para movimiento lineal \cite{extrusoraPLA}}
    \label{fig:ComponentesML}
\end{figure}

\vspace{-1cm}

En la Tabla \ref{tab:MecanismoLineal} se describen los componentes y la cantidad necesaria para este mecanismo:

\begin{table}[H]
\centering
\begin{tabular}{|c|l|c|}
\hline
\textbf{Etiqueta} & \textbf{Nombre} & \textbf{Cantidad} \\ \hline
A1 & Base sujeción para guías & 1 \\ \hline
A2 & Base sujeción para balero & 1 \\ \hline
A3 & Varilla lisa 8mm (\(L = 0.3m\)) & 2 \\ \hline
A4 & Motor a pasos & 1 \\ \hline
A5 & Polea dentada GT2 para banda 6mm & 1 \\ \hline
A6 & Base sujeción para motor & 1 \\ \hline
A7 & Soporte de eje óptico lineal SK8 8mm & 1 \\ \hline
A8 & Correa dentada GT2 6mm (\(L > 0.8m\)) & 1 \\ \hline
A9 & Plataforma balero lineal SCS8UU & 2 \\ \hline
A10 & Rueda Derlin Dual V & 2 \\ \hline
A11 & Base de sujeción para polea & 1 \\ \hline
A12 & Polea de aluminio 20T 6W & 1 \\ \hline
\end{tabular}
\caption{Tabla de descripción de componentes para mecanismo lineal.}
\label{tab:MecanismoLineal}
\end{table}

La elección de componentes para este mecanismo se inspira en las partes comúnmente utilizadas en ejes de máquinas CNC o impresoras 3D, que requieren una gran precisión en sus movimientos lineales. Específicamente, este diseño se utiliza para desplazarse a lo largo del eje del carrete de filamento. Se han desarrollado cuatro bases, identificadas como A1, A2, A6, y A11, para asegurar el soporte de ciertos componentes. Se hará uso de la impresión 3D como método de manufactura junto con el PLA como material de impresión.

Después de decidir sobre el sistema de transmisión y los materiales a utilizar, se elige un motor adecuado para garantizar la precisión requerida en el movimiento lineal que facilitará un embobinado eficiente. Debido a su uso frecuente en este tipo de aplicaciones, se opta por un motor paso a paso NEMA 17. A continuación, se determinará el par nominal necesario del motor, utilizando las directrices del Manual SureStep, como se ilustra en la Figura \ref{fig:ConsideracionesParMotor} para orientar los cálculos relacionados con el par motor.

\begin{figure}[H]
    \centering
    \includegraphics[width=0.9\linewidth]{ImagenesTT1/ConsideracionesParMotor.png}
    \caption{Esquema de consideraciones para el cálculo del par motor \cite{extrusoraPLA}}
    \label{fig:ConsideracionesParMotor}
\end{figure}

\vspace{-0.5cm}

Utilizando un software específico, se ha estimado el peso de los componentes individuales del carro móvil del mecanismo. Estos valores se presentan en la Tabla \ref{tab:peso-piezas-carro-movil}

\begin{table}[h]
\centering
\begin{tabular}{|c|c|}
\hline
\textbf{Etiqueta} & \textbf{Peso [Kg]} \\ \hline
A2 & 0.033 \\ \hline
A3 & 0.08 \\ \hline
A4 & 0.025 \\ \hline
A11 & 0.082 \\ \hline
A12 & 0.064 \\ \hline
\textbf{Total} & \textbf{0.284} \\ \hline
\end{tabular}
\caption{Peso de las piezas del carro móvil del mecanismo de desplazamiento.}
\label{tab:peso-piezas-carro-movil}
\end{table}

A continuación, se detallan los datos del actuador y los componentes:

\begin{itemize}
    \item Peso de la masa = 350g o 0.35kg
    \item Diámetro de la polea = 12.25mm
    \item Grueso de la polea = 6mm
    \item Material de la polea: Aluminio
    \item Distancia de movimiento = 0.25m
    \item Eficiencia de la polea y de la correa = 0.8
    \item Coeficiente de fricción del rodamiento = 0.003
\end{itemize}

Para determinar el torque necesario para mover la carga, se comienza por calcular la inercia total del sistema, utilizando la ecuación \ref{eq:total_inertia}.

\begin{equation}
J_{\text{total}} [\text{Kg} \cdot \text{m}^2] = J_{\text{motor}} + J_{\text{polea}} + J_{\text{carga}}
\label{eq:total_inertia}
\end{equation}

donde la inercia de la polea (recordando que son dos poleas) se puede calcular como sigue en la ecuación \ref{eq:polea_inertia}.

\begin{equation}
J_{\text{polea}} = (\pi \cdot L \cdot p \cdot r^4) \cdot 2
\label{eq:polea_inertia}
\end{equation}


Donde:
\begin{itemize}
    \item $L$ = grueso de la polea en [m].
    \item $p$ = densidad del aluminio en [$\text{Kg/m}^3$].
    \item $r$ = radio de la polea en [m].
\end{itemize}

La inercia de la polea se calcula con la siguiente fórmula:
\begin{equation}
J_{\text{polea}} \approx \left( \pi \cdot 0.006 \cdot 2700 \cdot (0.006125)^4 \right) \cdot 2
\label{eq:inertia_polea}
\end{equation}

Esta ecuación simplifica a:
\begin{equation}
J_{\text{polea}} \approx \left( \pi \cdot 0.006 \cdot 2700 \cdot 0.00000001407422 \right) \approx 1.432 \cdot 10^{-7} \, [\text{Kg} \cdot \text{m}^2]
\label{eq:inertia_polea_simplified}
\end{equation}


El cálculo de la inercia de la carga se calcula en la ecuación \ref{eq:carga_inertia}.

\begin{equation}
J_{\text{carga}} = \text{masa} \cdot r^2 = 0.35 \cdot 0.006125^2 = 1.3130 \cdot 10^{-5} \, [\text{Kg} \cdot \text{m}^2]
\label{eq:carga_inertia}
\end{equation}

La inercia de la carga y las poleas reflejadas al eje del motor se calculan a continuación, y se proponen las mejoras al sistema:

\[
J_{\text{poleas+carga}} [\text{Kg} \cdot \text{m}^2] = 1.432 \cdot 10^{-7} + 1.3130 \cdot 10^{-5} = 1.327 \cdot 10^{-5} \, [\text{Kg} \cdot \text{m}^2]
\]

Posterior a esto, se propone el uso de un motor a pasos NEMA 17, de 34mm, 40 N-cm y 1.7 A, el cual tiene una inercia de:
\begin{equation}
J_{\text{motor}} = 57 \, [\text{g-cm}^2] = 0.0000057 \, [\text{Kg-m}^2]
\label{eq:J_motor}
\end{equation}

El torque de aceleración requerido del motor se muestra a continuación:
\begin{equation}
T_{\text{acc}} \approx \left(1.327 \cdot 10^{-5} + 0.0000057\right) \cdot \frac{(60/2\pi)^2}{60} \cdot 2\pi \cdot 1.89 \cdot 10^{-5} \, [\text{N-m}]
\label{eq:torque_acc}
\end{equation}

Para calcular la fuerza total que se ejerce sobre el eje del motor, se hace uso de la ecuación 3.20, y el cálculo se muestra a continuación:
\begin{equation}
F_{\text{total}} = 0 + 0.003 \cdot 0.35 \cdot \cos(0) \cdot 9.81 + 0.35 \cdot 9.81 = 3.44 \, [\text{N}]
\label{eq:force_total}
\end{equation}

Para continuar con el cálculo, se necesita establecer el par resistivo a proporcionar por el motor, esto se hace con la ecuación 3.31.
\begin{equation}
T_{\text{resist}} = F_{\text{total}} \cdot r
\label{eq:torque_resist}
\end{equation}

Para nuestro caso de análisis, el cálculo se presenta a continuación:
\begin{equation}
T_{\text{resist}} = 3.44 \cdot 0.006125 = 0.02107 \, [\text{N-m}]
\label{eq:calc_torque_resist}
\end{equation}

Por lo tanto, el torque necesario para mover el sistema se calcula en la ecuación 3.32.
\begin{equation}
T_{\text{motor}} = T_{\text{acc}} + T_{\text{resist}} = 1.89 \cdot 10^{-5} + 0.02107 = 0.0208 \, [\text{N-m}] = 2.1088 \, [\text{N-cm}]
\label{eq:total_motor_torque}
\end{equation}

Por lo tanto, el torque requerido para desplazar la carga del sistema es mucho menor
al torque que puede proporcionar el motor, por lo tanto el uso de un motor a pasos
tipo NEMA 17 de 33mm, 40 N-cm y 1.7 A es apto para este mecanismo.

\textbf{Mecanismo de rotación}

Para facilitar el movimiento del bobinado mientras el filamento transita axialmente, es viable emplear diversos actuadores y sistemas en combinación. Al optimizar previamente el movimiento de embobinado, se pueden prevenir superposiciones indeseadas que podrían complicar tanto el almacenamiento como la funcionalidad del filamento cuando se usa en impresoras 3D.

Se ha escogido un motor de pasos debido a su superior par motor y la facilidad de controlar su movimiento en comparación con otros tipos de motores. Este motor maneja mayores cargas, por lo que se ha decidido integrar un motor de pasos con un reductor de engranajes para aumentar su par motor.

Para confirmar la elección del motor de pasos, es esencial calcular el par motor nominal utilizando la metodología descrita en el Manual SureStep, referenciada en la Figura \ref{fig:ConsideracionesParMotorRot}, que ilustra el enfoque para calcular el par del motor.


\begin{figure}[H]
    \centering
    \includegraphics[width=0.7\linewidth]{ImagenesTT1/ConsideracionesParMotorRot.png}
    \caption{Esquema de consideraciones para el cálculo del par motor \cite{extrusoraPLA}}
    \label{fig:ConsideracionesParMotorRot}
\end{figure}
A continuación se presentan los datos del actuador y los componentes relevantes:
\begin{itemize}
    \item Peso de la masa = 1.25 Kg.
    \item Peso del acople = 15 g = 0.015 Kg.
    \item Diámetro del tornillo = 8mm.
    \item Longitud del tornillo = 150mm.
    \item Paso del tornillo = 1.25mm/rev (pitch = 0.8 rev/mm o 800 rev/m).
    \item Densidad del acero = 7700 kg/m$^3$.
    \item Reductor de engranajes: 5.18:1.
\end{itemize}

A partir de estos datos, la inercia reflejada en el eje del motor se calcula con la ecuación:
\begin{equation}
J_{\text{total}} [\text{Kg} \cdot \text{m}^2] = J_{\text{motor}} + J_{\text{reductor}} + \frac{J_{\text{acople}} + J_{\text{tornillo}} + J_{\text{carga}}}{i^2}
\label{eq:inertia_total}
\end{equation}

Dado que $i$ representa la razón de reducción del reductor de velocidad, y considerando prácticas anteriores, idealmente se considera la inercia del acople y el reductor como despreciable. Luego, la inercia de la carga se calcula con la fórmula:

\begin{equation}
J_{\text{carga}} = \frac{1.25}{0.9} \left(\frac{1}{2\pi \cdot 800}\right)^2 = 5.497 \cdot 10^{-8} \, [\text{Kg} \cdot \text{m}^2]
\label{eq:carga_inertia}
\end{equation}

Para calcular la inercia del tornillo se utiliza la ecuación:
\begin{equation}
J_{\text{tornillo}} \approx \frac{\pi \cdot 0.15 \cdot 7700 \cdot 0.0004^2}{2} \approx 4.644 \cdot 10^{-11} \, [\text{Kg} \cdot \text{m}^2]
\label{eq:tornillo_inertia}
\end{equation}

La inercia total de la carga y el tornillo que se refleja en el eje del motor es:
\begin{equation}
J_{\text{carga+tornillo}} = 5.497 \cdot 10^{-8} + 4.644 \cdot 10^{-11} = 5.501 \cdot 10^{-8} \, [\text{Kg} \cdot \text{m}^2]
\label{eq:total_inertia}
\end{equation}

Posteriormente, se requiere calcular el torque necesario para acelerar esta inercia. Dado que este proceso es lento, con velocidades menores a 50 rpm, se propone que el motor permita acelerar una carga desde 0 a 30 RPM en 3 segundos. El cálculo del torque para acelerar la carga se efectúa utilizando la siguiente ecuación:
\begin{equation}
T_{\text{acel}} \approx 5.501 \cdot 10^{-8} \cdot \left(\frac{30 \cdot 2\pi}{60}\right)^2 \approx 5.761 \cdot 10^{-8} \, [\text{N-m}]
\label{eq:torque_acel}
\end{equation}

Después, se necesita determinar el torque resistivo en la operación de movimiento para estimar este valor por fórmulas similares. Para esto, se hace uso de la ecuación siguiente:
\begin{equation}
T_{\text{resist}} = \frac{\frac{F_{\text{total}}}{2\pi \cdot P}}{i} \quad [\text{N-m}]
\label{eq:torque_resist}
\end{equation}

 Para calcular la fuerza total que se ejerce sobre el eje del motor, se expresa en la ecuación \ref{eq:total_force1}:
\begin{equation}
F_{\text{total}} = F_{\text{ext}} + F_{\text{fricción}} + F_{\text{gravedad}} = F_{\text{ext}} + \mu \cdot \text{masa} \cdot \cos(\theta) \cdot 9.81 + F_{\text{gravedad}} \quad [\text{N}]
\label{eq:total_force1}
\end{equation}

 Debido a que no se aplican fuerzas externas, y el peso del tornillo es de aproximadamente 42.35 g (obtenido a partir del software CAD), el cálculo de las fuerzas se expresa en la ecuación \ref{eq:total_force}:
\begin{equation}
F_{\text{total}} = 0 + 0.58 \cdot 1.5 \cdot \cos(0) \cdot 9.81 + 1.5 \cdot 9.81 = 23.24 \, [\text{N}]
\label{eq:total_force}
\end{equation}

 Se necesita determinar el torque resistivo en la operación de movimiento, para estimar este valor por fórmulas similares, se hace uso de la ecuación \ref{eq:torque_resist1}:
\begin{equation}
T_{\text{resist}} = \frac{F_{\text{total}}}{2\pi \cdot P} \quad [\text{N-m}]
\label{eq:torque_resist1}
\end{equation}

 El torque resistivo para este sistema se calcula con la ecuación \ref{eq:torque_resist2}::
\begin{equation}
T_{\text{resist}} = \frac{23.24}{2\pi \cdot 800} \cdot \frac{1}{5.18} = 8.9255 \cdot 10^{-4} \, [\text{N-m}]
\label{eq:torque_resist2}
\end{equation}

El torque a ser suministrado por el motor se muestra en la ecuación 3.41:
\begin{equation}
T_{\text{movimiento}} = T_{\text{acel}} + T_{\text{resist}} = 5.7610^{-8} + 8.9255 \cdot 10^{-4} \approx 8.92 \cdot 10^{-4} \, [\text{N-m}]
\label{eq:motor_torque}
\end{equation}

Lo anterior se refiere al torque requerido del motor antes de que se incluya la inercia del mismo. Para este caso se propone el uso de un motor tipo NEMA 17 de 40mm, con un torque máximo de 1.68 Nm y 1.7 A de corriente por fase. A continuación, se propone validar que el motor propuesto cumpla los requerimientos antes calculados.

Este motor tiene una inercia aproximada de:
\[
J_{\text{motor}} = 57 \, [\text{g-cm}^2] = 0.0000057 \, [\text{Kg-m}^2]
\]

A partir de esto, el torque de aceleración del motor es modificado y se muestra en la ecuación:
\begin{equation}
T_{\text{acel}} = \left(5.501 \times 10^{-8} + 0.0000057\right) \cdot \left(\frac{30}{3} \cdot \frac{2\pi}{60}\right)^2 \approx 6.02 \times 10^{-6} \, [\text{N-m}]
\label{eq:torque_acel}
\end{equation}

Mientras que el torque real del motor se calcula como sigue en la ecuación:
\begin{equation}
T_{\text{motor}} = T_{\text{acel}} + T_{\text{resist}} = 6.02 \times 10^{-6} + 8.92 \times 10^{-4} = 8.98 \times 10^{-4} \, [\text{N-m}]
\label{eq:torque_motor}
\end{equation}

Por lo tanto, para este sistema es suficiente el uso de un motor a pasos tipo NEMA 17 de 40mm, 1.68 Nm y 1.7 A, el cual a partir del análisis del par que puede entregar cumple con los requisitos del sistema.

\textbf{Estructura para el Sistema Rotativo}

Tras la selección del motor, el siguiente paso es desarrollar una estructura que no solo sostenga y permita el giro de la bobina sobre el eje del motor, sino que también determine su peso. Se ha elegido un perfil de aluminio para este componente debido a su resistencia a la corrosión, bajo peso, coste económico y la modularidad que ofrece. La selección de este material se debe también a la disponibilidad de numerosos accesorios compatibles y a que su diseño estructural es adecuado para soportar cargas axiales, de flexión y torsión que pueden llegar a los 2 kg aproximadamente. La estructura diseñada se muestra en la Figura \ref{fig:EstructuraRotacional}.

\begin{figure}[H]
    \centering
    \includegraphics[width=0.5\linewidth]{ImagenesTT1/EstructuraRotacional.png}
    \caption{Estructura para el movimiento rotacional de la bobina}
    \label{fig:EstructuraRotacional}
\end{figure}

\vspace{-0.5 cm}

En la Figura \ref{fig:EstructuraRotacional} se ilustra el uso de una célula de carga de 5 kg que se emplea para registrar el peso de la bobina de filamento mientras se enrolla. Basándose en este arreglo, es esencial fabricar dos piezas clave para este sistema: una que asegura la célula de carga al perfil de aluminio y otra que conecta el motor con la célula de carga. La selección de los material será PLA utilizando técnicas de impresión 3D para su producción.

\textbf{Análisis de esfuerzos para las piezas del módulo de embobinado}

Para garantizar la adecuación de las piezas fabricadas para este módulo, que se ilustran en la Figura \ref{fig:EstructuraRotacional}, se llevó a cabo un análisis estático usando el software SolidWorks\textregistered. Este proceso evalúa los esfuerzos en cada uno de los componentes involucrados. 

\textbf{Sujeción de motor}

El análisis inicial se centró en la pieza que mantiene el motor NEMA 17, que está conectado a la bobina de filamento y que soportará una carga máxima de 2.25 kg. Los detalles de este análisis se presentan en la Figura \ref{fig:SujecionMotor}.

\begin{figure}[H]
    \centering
    \includegraphics[width=0.7\linewidth]{ImagenesTT1/SujecionMotor.png}
    \caption{Esfuerzo en el soporte de sujeción del motor}
    \label{fig:SujecionMotor}
\end{figure}

Según se observa en los resultados presentados en la Figura \ref{fig:SujecionMotor}, la pieza en cuestión es capaz de soportar la carga asignada sin sobrepasar la resistencia máxima del material utilizado.

\textbf{Soporte de la celda de carga}

Continuando con los estudios de resistencia, el análisis subsiguiente examina la pieza que fija la celda de carga al perfil de aluminio. Dicho componente está diseñado para tolerar una carga máxima de 2.5 kg, con los resultados de este análisis exhibidos en la Figura \ref{fig:SujeciónCelda}.

\begin{figure}[H]
    \centering
    \includegraphics[width=0.5\linewidth]{ImagenesTT1/SujeciónCelda.png}
    \caption{Esfuerzo en el soporte de sujeción a la celda}
    \label{fig:SujeciónCelda}
\end{figure}

Por lo que, analizando los resultados del análisis en la Figura \ref{fig:SujeciónCelda} la pieza puede
soportar la carga sin generar algún esfuerzo mayor al último del material.



\newpage

\subsection{Apéndice 3 - Código para la comunicación del sistema}

\textbf{Importación de librerías y definición de pines}

En esta sección se importan las librerías necesarias para el funcionamiento del proyecto. La librería Nextion.h permite la comunicación con la pantalla táctil Nextion, mientras que GyverMAX6675.h es utilizada para leer datos del sensor de temperatura MAX6675. También se definen los pines utilizados para diferentes funciones del sistema, como controlar la velocidad, tiempo, temperatura, y los estados del motor y LED.

\begin{algorithm}[H]
\SetAlgoLined

\#include "Nextion.h" \;
\#include GyverMAX6675.h \;

\#define CLK\_PIN 13 // Pin para SCK del sensor MAX6675 \;
\#define DATA\_PIN 12 // Pin para SO del sensor MAX6675 \;
\#define CS\_PIN 10 // Pin para CS del sensor MAX6675 \;

const int speedPin = 11; // Pin para el control de velocidad \;
const int timePin = 9; // Pin para el control de tiempo \;
const int ledPin = 7; // Pin para el LED indicador \;
const int tempPin = 6; // Pin para el control de temperatura \;
const int stepPin = 5; // Pin para el control de pasos del motor \;
const int dirPin = 4; // Pin para la dirección del motor \;
const int FiringPin = 3; // Pin para el ángulo de disparo \;
const int zeorCPin = 2; // Pin para la detección de cruce por cero \;

\caption{Importación de librerías y definición de pines}
\label{al:ImportacionDefinicionPines}
\end{algorithm}

\textbf{Definición de constantes y variables}

Se definen constantes que establecen los límites máximos y mínimos para diferentes parámetros como la velocidad del motor, el tiempo de mezcla, y la temperatura. Estas constantes aseguran que los valores utilizados en el programa se mantengan dentro de rangos seguros y operativos. También se declaran variables que se utilizarán para almacenar y manipular datos durante la ejecución del programa, incluyendo parámetros para el control PID y banderas para el estado del sistema.

\begin{algorithm}[H]
\SetAlgoLined

const float maxRpmMixMotor = 600.0; \;
const float maxRpmMixMixer = 400.0; \;
const float minRpmMixMixer = 10.0; \;
const int minTimeMix = 2; // Minutos reales mínimos \;
const int maxTimeMix = 5; // Minutos reales máximos \;
const int minTemp = 170; \;
const float maxTemp = 210; \;
const float maxTempRes = 300; \;
const long interval = 60; // Intervalo en milisegundos para medir un minuto \;
const long stepInterval = 5; // Intervalo en milisegundos para medir pasos \;
const int maximum\_firing\_delay = 7; \;
const int temp\_read\_Delay = 100; // Retardo de lectura del PID \;

int kp = 203; int ki= 7.2; int kd = 1.04; \;
int PID\_p = 0; int PID\_i = 0; int PID\_d = 0; \;
bool miXButtonFlag = false; \;
bool extrudeFlag = false; \;
unsigned long previousMillis = 0; \;
unsigned long previousMillisStep = 0; \;
unsigned long previousMillisMesTemp = 0; \;
int stepState = LOW; \;
long delaytime = 0; \;
$uint32_t$ setTempVal = 0; \;
int real\_temperature = 0; \;
float PID\_error = 0; \;
float previous\_error = 0; \;
float elapsedTime, Time, timePrev; \;
int PID\_value = 0; \;

\caption{Definición de constantes y variables}
\label{al:DefinicionConstantesVariables}
\end{algorithm}

\textbf{Declaración de objetos Nextion y configuración}

Aquí se declaran los objetos correspondientes a los componentes de la pantalla Nextion que se utilizarán para interactuar con el usuario. Estos incluyen botones de estado dual, campos numéricos, y textos. Los objetos se registran en una lista para poder manejar eventos táctiles. También se inicializa el sensor de temperatura configurando los pines adecuados.

\begin{algorithm}[H]
\SetAlgoLined

NexDSButton bt0 = NexDSButton(0, 1, "bt0"); \;
NexDSButton bt1 = NexDSButton(0, 9, "bt1"); \;
NexNumber n0 = NexNumber(0, 2, "n0"); \;
NexNumber n1 = NexNumber(0, 4, "n1"); \;
NexNumber n2 = NexNumber(0, 6, "n2"); \;
NexNumber n3 = NexNumber(0, 14, "n3"); \;
NexNumber n5 = NexNumber(0, 12, "n5"); \;
NexText t6 = NexText(0, 13, "t6"); \;
char buffer[100] = {0}; \;


NexTouch *nex\_listen\_list[] = \{
    &n0, \;
    &n1, \;
    &n2, \;
    &n5, \;
    &bt0, \;
    &bt1, \;
    NULL \;
\};

GyverMAX6675 CLK\_PIN, DATA\_PIN, CS\_PIN sens; \;

\caption{Declaración de objetos Nextion y configuración}
\label{al:DeclaracionObjetosNextion}
\end{algorithm}

\textbf{Callbacks para botones Nextion}

Las funciones bt0PopCallback y bt1PopCallback son los callbacks para los botones de la pantalla Nextion. Estas funciones se ejecutan cuando los botones son presionados. En el caso de bt0PopCallback, se controla el inicio y detención del proceso de mezcla, ajustando la velocidad y el tiempo de mezcla según los valores obtenidos de la pantalla. bt1PopCallback controla el proceso de extrusión, ajustando la temperatura de extrusión.

\begin{algorithm}[H]
\SetAlgoLined

void bt0PopCallback(void *ptr) \{
    $uint32_t$ dual\_state; \;
    $uint32_t$ speedVal; \;
    $uint32_t$ timeVal; \;
    bt0.getValue(&dual\_state); \;
    t6.setText("Mixing!"); \;

\caption{Callbacks para botones Nextion - Parte 1}
\label{al:CallbacksBotonesNextion}
\end{algorithm}

\begin{algorithm}[H]
\SetAlgoLined
    
    \If{dual\_state} \{
        miXButtonFlag = true; \;
        n0.getValue(&speedVal); \;
        n1.getValue(&timeVal); \;
        \If{speedVal > maxRpmMixMixer} \{
            speedVal = maxRpmMixMixer; \;
        \} \;
        \ElseIf{speedVal < minRpmMixMixer} \{
            speedVal = minRpmMixMixer; \;
        \} \;
        analogWrite(speedPin, speedVal/maxRpmMixMotor *255.0); \;
        n0.setValue(speedVal); \;
        \If{timeVal $>$= maxTimeMix} \{
            timeVal = maxTimeMix; \;
        \} \;
        \ElseIf{timeVal $<$= minTimeMix} \{
            timeVal = minTimeMix; \;
        \} \;
        n1.setValue(timeVal); \;
        n5.setValue(timeVal); \;
        digitalWrite(ledPin, HIGH); \;
    \} \;

\caption{Callbacks para botones Nextion - Parte 2}
\label{al:CallbacksBotonesNextion}
\end{algorithm}

\begin{algorithm}[H]
\SetAlgoLined

    \Else \{
        miXButtonFlag = false; \;
        digitalWrite(ledPin, LOW); \;
    \} \;
\} \;

void bt1PopCallback(void *ptr) \{
    uint32_t dual\_state; \;
    bt1.getValue(&dual\_state); \;
    \If{dual\_state} \{
        t6.setText("Extruding!"); \;
        n2.getValue(&setTempVal); \;
        extrudeFlag = true; \;
        \If{setTempVal $>$ maxTemp} \{
            setTempVal = maxTemp; \;
        \} \;
        \ElseIf{setTempVal $<$ minTemp} \{
            setTempVal = minTemp; \;
        \} \;
        n2.setValue(setTempVal); \;
        analogWrite(tempPin, setTempVal/210.0 * 255.0); \;
    \} \;
    \Else \{
        extrudeFlag = false; \;
        digitalWrite(FiringPin, LOW); \;
    \} \;
\} \;

\caption{Callbacks para botones Nextion - Parte 3}
\label{al:CallbacksBotonesNextion}
\end{algorithm}

\textbf{Configuración inicial en setup()}

La función setup se ejecuta una vez al inicio del programa. Aquí se inicializan los componentes de la pantalla Nextion y se registran los callbacks para los botones. También se configuran los pines como entradas o salidas según su función, y se establece el estado inicial de ciertos pines, como la dirección del motor y el estado de disparo.

\begin{algorithm}[H]
\SetAlgoLined

void setup() \{
    nexInit(); \;
    bt0.attachPop(bt0PopCallback, &bt0); \;
    bt1.attachPop(bt1PopCallback, &bt1); \;
    pinMode(ledPin, OUTPUT); \;
    pinMode(speedPin, OUTPUT); \;
    pinMode(timePin, OUTPUT); \;
    pinMode(tempPin, OUTPUT); \;
    pinMode(stepPin, OUTPUT); \;
    pinMode(dirPin, OUTPUT); \;
    pinMode(FiringPin, OUTPUT); \;
    pinMode(zeorCPin, INPUT); \;
    digitalWrite(dirPin, HIGH); \;
    digitalWrite(FiringPin, LOW); \;
\} \;

\caption{Configuración inicial en setup()}
\label{al:ConfiguracionInicialSetup}
\end{algorithm}

\textbf{Bucle principal en loop()}

La función loop se ejecuta continuamente después de la configuración inicial. En este bucle, se manejan los eventos de la pantalla Nextion mediante nexLoop, y se llaman a otras funciones como Timer, Extrude, y TempMes para gestionar la lógica del sistema, incluyendo la temporización, la extrusión y la medición de temperatura.

\begin{algorithm}[H]
\SetAlgoLined

void loop() \{
    nexLoop(nex\_listen\_list); \;
    Timer(); \;
    Extrude(); \;
    TempMes(); \;
\} \;

\caption{Bucle principal en loop()}
\label{al:BuclePrincipalLoop}
\end{algorithm}

\textbf{Función Timer()}

La función Timer controla el temporizador de la mezcla. Si la bandera $miXButtonFlag$ está activa, se verifica si ha pasado el intervalo de tiempo definido. Si es así, se actualiza el valor del tiempo restante en la pantalla y se decrece en uno. Cuando el tiempo llega a 1, se detiene el proceso de mezcla y se apaga el motor.

\begin{algorithm}[H]
\SetAlgoLined

void Timer() \{
    \If{miXButtonFlag == true} \{
        unsigned long currentMillis = millis(); \;
        uint32_t timeVal; \;
        \If{currentMillis - previousMillis >= stepInterval} \{
            previousMillis = currentMillis; \;
            n5.getValue(&timeVal); \;
            n5.setValue(timeVal-1); \;
            \If{timeVal <= 1} \{
                t6.setText("Extrude"); \;
                miXButtonFlag = false; \;
                analogWrite(speedPin, 0); \;
                bt0.setValue(0); \;
            \} \;
        \} \;
    \} \;
\} \;

\caption{Función Timer()}
\label{al:FuncionTimer}
\end{algorithm}

\textbf{Función Extrude()}

La función Extrude controla el proceso de extrusión. Si la bandera $extrudeFlag$ está activa, se llaman a las funciones $StepMotor$ y $SetTemperature$ para manejar el movimiento del motor y ajustar la temperatura de extrusión según los valores establecidos.

\begin{algorithm}[H]
\SetAlgoLined

void Extrude() \{
    \If{extrudeFlag == true} \{
        StepMotor(); \;
        SetTemperature(); \;
    \} \;
\} \;

\caption{Función Extrude()}
\label{al:FuncionExtrude}
\end{algorithm}

\textbf{Función StepMotor()}

La función $StepMotor$ controla el movimiento del motor paso a paso. Utiliza un intervalo de tiempo para alternar el estado del pin del motor, lo que hace que el motor gire. Si el tiempo transcurrido desde la última actualización es mayor que el intervalo definido, se cambia el estado del pin del motor.

\begin{algorithm}[H]
\SetAlgoLined

void StepMotor() \{
    unsigned long currentMillis = millis(); \;
    \If{currentMillis - previousMillisStep >= stepInterval} \{
        previousMillisStep = currentMillis; \;
        \If{stepState == LOW} \{
            stepState = HIGH; \;
        \} \;
        \Else \{
            stepState = LOW; \;
        \} \;
        digitalWrite(stepPin, stepState); \;
    \} \;
\} \;

\caption{Función StepMotor()}
\label{al:FuncionStepMotor}
\end{algorithm}

\textbf{Función SetTemperature()}

La función $SetTemperature$ ajusta la temperatura del sistema de extrusión. Espera un cruce por cero en la señal de corriente alterna y luego introduce un retraso calculado basado en el valor PID. Luego, envía un pulso al pin de disparo para ajustar la temperatura.

\begin{algorithm}[H]
\SetAlgoLined

void SetTemperature() \{
    \If{pulseIn(zeorCPin, HIGH)} \{
        delay(maximum\_firing\_delay - PID\_value); \;
        digitalWrite(FiringPin, HIGH); \;
        delayMicroseconds(10); \;
        digitalWrite(FiringPin, LOW); \;
    \} \;
\} \;

\caption{Función SetTemperature()}
\label{al:FuncionSetTemperature}
\end{algorithm}

\textbf{Función TempMes()}

La función $TempMes$ mide la temperatura actual del sistema. Si ha pasado el intervalo de tiempo definido desde la última medición, se lee la temperatura del sensor y se actualiza en la pantalla. También se llama a la función PID para ajustar el valor PID basado en la temperatura medida.

\begin{algorithm}[H]
\SetAlgoLined

void TempMes() \{
    unsigned long currentMillis = millis(); \;
    \If{currentMillis - previousMillisMesTemp $>$= temp\_read\_Delay} \{
        previousMillisMesTemp = currentMillis; \;
        \If{sens.readTemp()} \{
            real\_temperature = sens.getTemp(); \;
            n3.setValue(real\_temperature); \;
            PID(); \} \} \} \;

\caption{Función TempMes()}
\label{al:FuncionTempMes}
\end{algorithm}

\textbf{Función PID()}

La función PID ajusta el valor PID para el control de temperatura. Calcula el error entre la temperatura deseada y la temperatura real, y luego calcula los valores proporcional, integral y derivativo para ajustar el valor PID. Este valor se utiliza para controlar el tiempo de disparo y mantener la temperatura estable.

\begin{algorithm}[H]
\SetAlgoLined

void PID() \{
    PID\_error = setTempVal - real\_temperature; \;
    \If{PID\_error $>$ 30} \{
        PID\_i = 0; \;
    \} \;
    PID\_p = kp * PID\_error; \;
    PID\_i = PID\_i + (ki * PID\_error); \;
    timePrev = Time; \;
    Time = millis(); \;
    elapsedTime = (Time - timePrev) / 1000; \;
    PID\_d = kd * ((PID\_error - previous\_error) / elapsedTime); \;
    PID\_value = PID\_p + PID\_i + PID\_d; \;
    \If{PID\_value $<$ 0} \{
        PID\_value = 0; \;
    \} \;
    \ElseIf{PID\_value $>$ maximum\_firing\_delay} \{
        PID\_value = maximum\_firing\_delay; \;
    \} \;
    previous\_error = PID\_error; \;
\} \;

\caption{Función PID()}
\label{al:FuncionPID}
\end{algorithm}

\clearpage
\newpage

\subsection{Apéndice 4 - Planos normalizados}

\begin{figure}
    \centering
    \includegraphics[width=1.2\linewidth, angle=90]{ImagenesTT1/Soporte Motor.pdf}
\end{figure}

\begin{figure}
    \centering
    \includegraphics[width=1.2\linewidth, angle=90]{ImagenesTT1/AcopleMezclado.pdf}
\end{figure}

\begin{figure}
    \centering
    \includegraphics[width=1.2\linewidth, angle=90]{ImagenesTT1/EjeMezclado.pdf}
\end{figure}

\begin{figure}
    \centering
    \includegraphics[width=1.2\linewidth, angle=90]{ImagenesTT1/Turbina Rushton.pdf}
\end{figure}

\begin{figure}
    \centering
    \includegraphics[width=1.2\linewidth, angle=90]{ImagenesTT1/PuertaPrecalentado.pdf}
\end{figure}

\begin{figure}
    \centering
    \includegraphics[width=1.2\linewidth, angle=90]{ImagenesTT1/Tolva.pdf}
\end{figure}

\begin{figure}
    \centering
    \includegraphics[width=1.2\linewidth, angle=90]{ImagenesTT1/Base rodamiento simple.pdf}
\end{figure}

\begin{figure}
    \centering
    \includegraphics[width=1.2\linewidth, angle=90]{ImagenesTT1/Altura.pdf}
\end{figure}

\begin{figure}
    \centering
    \includegraphics[width=1.2\linewidth, angle=90]{ImagenesTT1/Base canal.pdf}
\end{figure}

\begin{figure}
    \centering
    \includegraphics[width=1.2\linewidth, angle=90]{ImagenesTT1/Base nema 23.pdf}
\end{figure}

\begin{figure}
    \centering
    \includegraphics[width=1.2\linewidth, angle=90]{ImagenesTT1/Base ventilador.pdf}
\end{figure}

\begin{figure}
    \centering
    \includegraphics[width=1.2\linewidth, angle=90]{ImagenesTT1/Carcasa ventilador.pdf}
\end{figure}

\begin{figure}
    \centering
    \includegraphics[width=1.2\linewidth, angle=90]{ImagenesTT1/Husillo.pdf}
\end{figure}

\begin{figure}
    \centering
    \includegraphics[width=1.2\linewidth, angle=90]{ImagenesTT1/Matriz.pdf}
\end{figure}

\begin{figure}
    \centering
    \includegraphics[width=1.2\linewidth, angle=90]{ImagenesTT1/Cilindro.pdf}
\end{figure}

\begin{figure}
    \centering
    \includegraphics[width=1.2\linewidth, angle=90]{ImagenesTT1/Soportes.pdf}
\end{figure}

\begin{figure}
    \centering
    \includegraphics[width=1.2\linewidth, angle=90]{ImagenesTT1/Canal.pdf}
\end{figure}

\clearpage
\newpage

\begin{thebibliography}{5}

    \bibitem{dang2022} 
    B. Dang et al., 
    ``Current application of algae derivatives for bioplastic production: a review,'' 
    \textit{Bioresource Technology}, vol. 347, p. 126698, mar. 2022, 
    doi: 10.1016/j.biortech.2022.126698.

    \bibitem{nationalgeographic} 
    National Geographic, 
    ``¿A quién dañan las impresoras 3D? - National Geographic en español,'' 
    \textit{National Geographic en Español}, 18 de noviembre de 2022. 
    Disponible en: \url{https://www.ngenespanol.com/ciencia/impacto-ambiental-impresora-3d-tercera-dimension/#:~:text=Los%20residuos%20pl%C3%A1sticos%20generados%20por,en%20cuestiones%20ambientales%2C%20Christian%20L%C3%B6lkes}.

    \bibitem{rajpoot2022} 
    A. S. Rajpoot, T. Choudhary, H. Chelladurai, T. N. Verma, y V. Shende, 
    ``A comprehensive review on bioplastic production from microalgae,'' 
    \textit{Materials Today: Proceedings}, vol. 56, pp. 171-178, ene. 2022, 
    doi: 10.1016/j.matpr.2022.01.060.

    \bibitem{TablaAntecedentes_Ref1}
    A. Cepeda Alarcón and A. Salas Espino, "Máquina extrusora de filamento termoplástico PLA para la impresión 3D," UPIITA, México, Diciembre 2022.

    \bibitem{TablaAntecedentes_Ref2}
    J. Hernández y J. Valle, «Implementación de una máquina semi automática extrusora de filamento de plástico para impresoras 3D», ESIME, Ciudad de México, 2018.

    \bibitem{TablaAntecedentes_Ref3}
    O. Monroy Santos y J. A. Mora Sánchez, "Diseño de una máquina para el proceso de moldeo por extrusión a partir de pellets de polietileno," ESIME, Ciudad de México, México, junio 2022.

    \bibitem{TablaAntecedentes_Ref4}
    «Máquina extrusora con bobinado de filamento para impresora 3D», Filamentcycle. [En línea]. Disponible en: https://filamentcycle.com/fullextruder/

    \bibitem{white2003} 
    J.L. White, 
    ``Introduction to Extrusion,'' 
    \textit{Hanser Gardner Publications}, 2003.

    \bibitem{tadmor2006} 
    Z. Tadmor y C.G. Gogos, 
    ``Principles of Polymer Processing,'' 
    \textit{John Wiley \& Sons}, 2006.

    \bibitem{mohanty2002} 
    A.K. Mohanty, M. Misra, y L.T. Drzal, 
    ``Natural Fibers, Biopolymers, and Biocomposites,'' 
    \textit{CRC Press}, 2002.

    \bibitem{wagner2009} 
    J.R. Wagner y R. Neugebauer, 
    ``Handbook of Extrusion Technology,'' 
    \textit{William Andrew}, 2009.

    \bibitem{ref18}
    C. y. C. de A. T. a. P. de A. de yuca y maíz., “Juan Armando Gaitán Camacho,” \textit{Edu.co}. [En línea]. Disponible en: \url{https://repositorio.uniandes.edu.co/server/api/core/bitstreams/ab042059-fa63-4830-8253-10a353762425/content}.

    \bibitem{ref19}
    L. M. P. F. P. Del modo, “Plásticos sin petróleo,” \textit{Uned.es}. [En línea]. Disponible en: http://e-spacio.uned.es/fez/eserv/bibliuned:revista100cias-2006-numero9-5075/Plasticos\_sin\_petroleo.pdf

    \bibitem{ref20}
    D. León, I. Peña, J. Avendaño, y J. J. Alvarado, “PRODUCCION DE AGAR-AGAR EN COSTA RICA A PARTIR DE GRACILARIA FORTISSIMA,” \textit{Binasss.sa.cr}. [En línea]. Disponible en: \url{https://repositorio.binasss.sa.cr/repositorio/bitstream/handle/20.500.11764/3845/art2v5n2.pdf?sequence=1&isAllowed=y}

    \bibitem{ref21}
    \textit{Gov.ar}. [En línea]. Disponible en: \url{https://ri.conicet.gov.ar/bitstream/handle/11336/83026/CONICET_Digital_Nro.17da686e-25be-47ca-83ea-fa7fbabf829a_A.pdf?sequence=2&isAllowed=y}

    \bibitem{ref22}
    C. De, I. Ambiental, O. L. Christian, I. J. Braulio, y A. Pinos, “UNIVERSIDAD POLITÉCNICA SALESIANA SEDE CUENCA,” \textit{Edu.ec}. [En línea]. Disponible en: \url{https://dspace.ups.edu.ec/bitstream/123456789/18680/1/UPS-CT008736.pdf}.

    \bibitem{ref23}
    A. Labeaga, J. De Dios, y C. Ruiz, “Polímeros biodegradables. Importancia y potenciales aplicaciones,” \textit{Uned.es}. [En línea]. Disponible en: http://espacio.uned.es/fez/eserv/bibliuned:master-Ciencias-CyTQ-Alabeaga/Labeaga\_Viteri\_Aitziber\_TFM.pdf

    \bibitem{ref15}
    Charles, «3D printer Filament Temperature (+ chart) (2023)», \textit{3D Tech Valley}, 12 de junio de 2022. [En línea]. Disponible en: https://www.3dtechvalley.com/3d-printer-filament-temperature/#google\_vignette

    \bibitem{ref16}
    Z. Hay, «The best 3D printing temperatures for PLA, TPU, ABS, PETG, nylon», \textit{All3DP}, 28 de enero de 2023. Disponible en: https://all3dp.com/2/the-best-printing-temperature-for-different-filaments/#google\_vignette

    \bibitem{ref17}
    F. Arceo, «3D filament glass transition temperatures». \url{https://3dsolved.com/3d-filament-glass-transition-temperatures/}

    \bibitem{joogi2020} 
    K. Jõgi y R. Bhat, 
    ``Valorization of food processing wastes and by-products for bioplastic production,'' 
    \textit{Sustainable Chemistry and Pharmacy}, vol. 18, p. 100326, dic. 2020, 
    doi: 10.1016/j.scp.2020.100326.

    \bibitem{mezcladora}
    J. E. Rincón, "Diseño de una máquina mezcladora, automática, de materias primas para la elaboración de jabones líquidos, suavizantes y desengrasantes industriales, para la empresa químicos Zorel", proyecto integral de grado, Fund. Univ. Am., Bogotá, 2016.

    \bibitem{chocolate}P Pitayachaval and P Watcharamaisakul , "A review of a machine design of chocolate extrusion based co-rotating twin screw extruder", IOP Conf. Ser.: Mater. Sci. Eng. 703 012012, pp 3-6

    \bibitem{motoresDC}
    Yuridia y Yuridia, «Motor de corriente continua», SDI, 23 de junio de 2023. Disponible en: https://sdindustrial.com.mx/blog/motor-de-corriente-continua/

    \bibitem{CPU}
    «CPU - Qué es, concepto, funciones, partes y características», [En línea]. Disponible en: https://concepto.de/cpu/    

    \bibitem{GUIFer}
    D. Urrutia, «Qué es Interfaz Gráfica de Usario (GUI) - definición y ejemplos», Arimetrics, 17 de octubre de 2023. Disponible en: https://www.arimetrics.com/glosario-digital/interfaz-grafica-usuario-gui

    \bibitem{VDI2206}
    Gausemeier, J.; Moehringer, S. (2002). VDI 2206- A New Guideline for the Design of Mechatronic Systems. IFAC Proceedings Volumes, 35(2), 785–790.

    \bibitem{esquemaExtrusora}
    Mariano, «EXTRUSIÓN DE MATERIALES PLÁSTICOS», \textit{Tecnología de los Plásticos} \url{https://tecnologiadelosplasticos.blogspot.com/2011/03/extrusion-de-materiales-plasticos.html}.

    \bibitem{ref29}
    INEGI, "Clima de la Ciudad de México," \textit{CuentaME}, Instituto Nacional de Estadística y Geografía (INEGI). [En línea]. Disponible: \url{https://cuentame.inegi.org.mx/monografias/informacion/df/territorio/clima.aspx?tema=me&e=09}.

    \bibitem{ref30}
    M. Ciulu et al., "Rheological Properties of Honey," en \textit{IntechOpen}, 2017. [En línea]. Disponible: https://www.intechopen.com/chapters/53895.

    \bibitem{Savgorodny}
    V. K. Savgorodny, "Transformación de plásticos". Barcelona: Editorial Gustavo Gili, 1973.

    \bibitem{Ashby}
    M. Ashby, "Material Selection in Mechanical Design", 4th ed. ELSEVIER, 2011.

    \bibitem{goff2009dynisco}
    J. Goff and T. Whelan, \emph{The Dynisco Extrusion Processors Handbook}, 2nd ed. Hanser Gardner Publications, 2009, pp. 92-93.

    \bibitem{moreno2011}
    J. Moreno, J. Maldonado, J. Portocarrero, V. Molina, and M. Marulanda, "Automatización del Perfil de Calentamiento de una Extrusora de Polímeros en UAN Cali," Universidad Antonio Nariño, Colombia, Diciembre 2011.

    \bibitem{beltran}
    M. Beltrán and A. Marcilla, "Tecnología de Polímeros," p. 104.

    \bibitem{rosales2020}
    Rosales-Dávalos, Jaime, Gil-Antonio, Leopoldo, Mastache-Mastache, Jorge Edmundo y López-Ramírez, Roberto. ``Diseño e implementación de un sistema de control a lazo cerrado PID para manipular la temperatura en el proceso de termoformado.'' Revista de Ingeniería Eléctrica, 2020.

    \bibitem{ogata2003}
    Ogata, K., ``Ingeniería de control moderna,'' Pearson Educación, 2003.

    \bibitem{control-pid} ``Control PID,'' \emph{Picuino}, [Online]. Available: \url{https://www.picuino.com/es/control-pid.html}.




    \bibitem{ref24}
    S. Habib, "Density of ABS Material | The Complete Guide," \textit{PlasticRanger}, 29-Nov-2023. [Online]. Available: https://plasticranger.com/density-of-abs-material/. 

    \bibitem{ref25}
    S. Habib, "Density OF PLA Plastic | The Definitive Guide," \textit{PlasticRanger}, 20-Nov-2023. [Online]. Available: https://plasticranger.com/density-of-pla-plastic/. 

    \bibitem{ref26}
    "PETG Material For 3D Printing - Properties Guide," \textit{3DRealize}, 22-Nov-2021. [Online]. Available: https://www.3drealize.com/petg-filament-material-properties-guide/. 

    \bibitem{ref27}
    "Density of Rubber, natural," \textit{Aqua-Calc.com}. [Online]. Available: \url{https://www.aqua-calc.com/page/density-table/substance/rubber-coma-and-blank-natural}. 

    \bibitem{ref28}
    H.S. Hall, S.R. Knight, "Elementary Geometry," London: Macmillan and Co., 1895.

    \bibitem{ref31}
    "Agitadores Industriales," \textit{Fluidmix}. [En línea]. Disponible: www.agitadoresfluidmix.com. 

    \bibitem{FlujoAgitadores}
    "Hélices para agitadores industriales de flujo axial y radial," \textit{Autmix}. [En línea]. Disponible: www.autmix.com. 

    \bibitem{ref33}
    "Elementos Agitadores," \textit{MIXING PROCESS}. [En línea]. Disponible: www.mixing-process.com. 

    \bibitem{ref34}
    Wikipedia contributors, "Rushton turbine," \textit{Wikipedia, The Free Encyclopedia}. [Online]. Available: https://en.wikipedia.org/wiki/Rushton\_turbine. 

    \bibitem{ref35}
    Eppendorf United Kingdom, "A guide to impeller selection," \textit{Eppendorf}. [Online]. Available: https://www.eppendorf.com. 

    \bibitem{ref36}
    Jongia, "Meet Jongia’s Rushton Turbine for Gas-Liquid \&," \textit{Jongia Mixing Technology}. [Online]. Available: https://www.jongia.com. 

    \bibitem{ref37}
    MIXEL, "Rushton turbine," \textit{MIXEL - Manufacturer of industrial agitators}. [Online]. Available: https://www.mixel.fr. 

    \bibitem{ref38}
    Autmix, "Rushton," \textit{Autmix Flow}. [Online]. Available: https://autmix.com/autmix-flow/moviles-helices/rushton. 

    \bibitem{SeleccionMotor}
    ``Article on Motor Selection,'' \textit{XDMotor}. [Online]. Available: \url{https://www.xdmotor.tech/index.php?c=article\&id=1216}. 

    \bibitem{ref40}
    V. Castillo Uribe, "DISEÑO Y CÁLCULO DE UN AGITADOR DE FLUIDOS," \textit{Repositorio Universidad del Bío-Bío}. [Online]. Available: http://repobib.ubiobio.cl/jspui/bitstream/123456789/412/1/Castillo\_Uribe\_Vladimir.pdf. 
    
    \bibitem{PropiedadesAISI304}
    ``Annealed 304 Stainless Steel,'' \textit{MakeItFrom.com}. [Online]. Disponible en: \url{https://www.makeitfrom.com/material-properties/Annealed-304-Stainless-Steel}. 

    \bibitem{solidworks_manual}
    ``Manual de prácticas de SolidWorks,'' \textit{UNAM Facultad de Ingeniería}. [Online]. Available: \url{http://olimpia.cuautitlan2.unam.mx/pagina_ingenieria/mecanica/mat/mat_mec/m9/MANUAL_DE_PRACTICAS_DE_SOLIDWORKS.pdf}.
    
    \bibitem{ref_seleccion_motor}
    A. Hughes and B. Drury, "Electric Motors and Drives: Fundamentals, Types and Applications," Oxford, Newnes, 2006.
    
    \bibitem{ref43}
    "Fibra cerámica | Fórmula, propiedades y aplicación," Material-properties.org. [En línea]. Disponible en: https://www.material-properties.org/fibra-ceramica-formula-propiedades-y-aplicacion/ . 
    
    \bibitem{ref44}
    "Fibra Ceramica - Productos," Insyser Perú SAC. [En línea]. Disponible en: https://www.insyserperu.com/fibra-ceramica . 
    
    \bibitem{ref45}
    "Mantas de Fibra Cerámica - Fibra Cerámica - Aplitémica - Soluciones y servicios para aislaciones térmicas y acústicas," Aplitérmica.com. [En línea]. Disponible en: \url{https://www.aplitermica.com/fibra-ceramica/mantas-de-fibra-ceramica . }
    
    \bibitem{emerson2021}
    Emerson Automation Solutions, ``Catalog and Engineering Information - ASCO,'' 2021. [Online]. Available:\url{https://www.emerson.com/documents/automation/catalog-engineering-information-asco-en-7486760.pdf.}
        
    \bibitem{omega2021}
    Omega Engineering, ``Understanding Solenoid Valves,'' 2021. [Online]. Available: \url{https://www.omega.com/en-us/resources/understanding-solenoid-valves.}
       
    \bibitem{mercadolibre2024}
    Mercado Libre, ``Llave Electroválvula 1 pulg Solenoide 110V Gas Agua Aire,'' [Online].  Avilable: \url{https://articulo.mercadolibre.com.mx/MLM-737296526-llave-electrovalvula-1-pulg-solenoide-110v}

    \bibitem{nasa-manual}
    ``NASA Fastener Torque Design Manual,'' \emph{Engineering Library}, NASA Reference Publication 1228, [Online]. Available: \url{https://engineeringlibrary.org/reference/fastener-torque-nasa-design-manual}.

     \bibitem{ref1}M. F. Abass \& A. Hamdoon, «Studying the effect of using a mixture (synthetic polymer: natural polymer) on the rheological properties of asphalt», \textit{Journal of physics}, vol. 1999, n.o 1, p. 012143, sep. 2021, doi: 10.1088/1742-6596/1999/1/012143.
    
    \bibitem{ref2}Y. Cao, W. Wang \& Q. Wang, «Application of mechanical model for natural fibre reinforced polymer composites», \textit{Materials Research Innovations}, vol. 18, n.o sup2, pp. S2-357, may 2014, doi: 10.1179/1432891714z.000000000431.

    \bibitem{extrusoraPLA}
    A. Cepeda Alarcón y A. Salas Espino, ``Máquina extrusora de filamento termoplástico PLA para la impresión 3D,'' \textit{Instituto Politécnico Nacional, Unidad Profesional Interdisciplinaria en Ingeniería y Tecnologías Avanzadas}, dirigido por Dr. Cuervo Pinto Victor Daŕıo, Diciembre 2022.

\end{thebibliography}
\end{document}

